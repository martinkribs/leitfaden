\chapter{Quellen und Zitierweise}
\label{chap:quellen-zitierweise}

International haben sich unterschiedliche Standards für die Nachweisführung in wissenschaftlichen Texten etabliert. Sie werden üblicherweise nach den akademischen Institutionen benannt, die sie verantworten, z.B. American Psychological Association (APA), Chicago University oder Harvard University.

An der Rheinischen Hochschule Köln sollen aus Gründen der Einfachheit die Regeln angewendet werden, die im jeweils aktuellen Stand des Chicago Manual of Style festgelegt sind. Konkret handelt es sich um die Zitierweise, die aus Fußnoten und Bibliographie (Notes and Bibliography) besteht.

Allein für die Studiengänge Psychologie und Wirtschaftspsychologie gilt die Zitierweise nach APA, weil APA in diesen Wissenschaftsdisziplinen weltweit als zwingender Standard betrachtet wird.

Die Chicago-Regeln für die meisten Anwendungsfälle finden Sie in der separaten Anleitung der Rheinischen Hochschule fürs Zitieren in wissenschaftlichen Arbeiten.

Falls Sie von den Chicago-Regeln abweichen und einen anderen Zitierstil verwenden wollen, müssen Sie sich darüber mit Ihren Prüfern vorweg abstimmen. Die für die Bewertung relevanten Details anderer Zitierstile müssen sie dann auch eigenständig kontrollieren.

\section{Problemadäquate, wissenschaftliche Quellen (Monographien, wissenschaftliche Zeitschriften, Working Paper etc.) werden in angemessenem Umfang berücksichtigt}
\label{sec:wissenschaftliche-quellen}

Achten Sie darauf, dass Sie für Ihre Arbeit nicht nur Online-Veröffentlichungen verwenden. Das heißt nicht, dass gedruckte Quellen prinzipiell zu bevorzugen sind. Für viele renommierte wissenschaftliche Organe ist die Online-Veröffentlichung mittlerweile selbstverständlich. Konkret folgt daraus für Ihre Recherche, dass es zweitrangig ist, ob die Veröffentlichung gedruckt oder online zugänglich ist – entscheidend ist die inhaltliche Qualität, also die Relevanz für Ihr Thema sowie die Verlässlichkeit bzw. Zitierwürdigkeit.

\section{Die Quellenauswahl entspricht dem aktuellen Forschungsstand}
\label{sec:quellenauswahl-forschungsstand}

Zu ihren grundlegenden Leistungen beim wissenschaftlichen Arbeiten zählt, dass Sie für Ihre Arbeit solche Veröffentlichungen verwenden, welche die wichtigsten Positionen des aktuellen internationalen Diskurses zu Ihrem Thema bzw. den Stand der Technik repräsentieren. Konkret bedeutet dies, dass Sie die Texte nicht zufällig ausgewählt haben (die ersten zehn Ergebnisse auf Google; das entsprechende Regal in einer Buchhandlung) oder weil sie gerade verfügbar waren, sondern weil sie zur Standardliteratur zählen bzw. weil sie für Ihr spezifisches Forschungsinteresse relevant sind.

Verfügbarkeit ist kein Ausschlusskriterium. Das heißt, wenn eine bestimmte Veröffentlichung für Ihr Thema erforderlich ist, dann wird erwartet, dass Sie sie kennen, unabhängig davon, ob sie für Sie unmittelbar verfügbar ist oder ob Sie sie aufwändig organisieren müssen (z.B. über Fernleihe).

\section{Alle Quellen sind zitierwürdig}
\label{sec:quellen-zitierwuerdig}

Eine weitere wesentliche Leistung Ihrer Arbeit besteht darin, dass Sie dafür ausschließlich zitierwürdige Veröffentlichungen verwendet haben. Es besteht ein Unterschied zwischen zitierfähig und zitierwürdig. Zitierfähig ist jede verbale Äußerung, sogar Bilder können zitiert werden. Im Verhältnis dazu sind aber nur wenige Veröffentlichung auch würdig, in einer wissenschaftlichen Arbeit zitiert zu werden.

Zitierwürdig ist eine Veröffentlichung, wenn sie sich durch Verlässlichkeit, Vertrauenswürdigkeit und Sorgfalt auszeichnet und wenn sie die Quellen ihrer Erkenntnis nachvollziehbar macht. In der Regel heißt das, dass sie wissenschaftlichen Ansprüchen genügt. Meist ist das Medium bereits ein Indiz für Vertrauenswürdigkeit: Alle Publikationen, die mehrstufige Kontrollprozesse für eine Veröffentlichung voraussetzen (z.B. Peer-Review-Verfahren), sind prinzipiell zitierwürdig.

Aus diesem Grund ist Wikipedia grundsätzlich nicht zitierwürdig: Hier gibt es keine institutionalisierte Kontrollinstanz für die Wahrheit von Aussagen.

Zitierwürdigkeit verwechseln Sie bitte nicht mit Überzeugungskraft oder Korrektheit. Auch zitierwürdige Veröffentlichungen können Fehler enthalten und Irrtümer verbreiten. Sie müssen also die veröffentlichten Aussagen in jedem Fall kritisch prüfen, ob sie Ihnen zutreffend und überzeugend erscheinen.

Auch ein journalistischer Text oder ein Video auf YouTube kann zitierwürdig sein – wenn die getroffenen Aussagen durch Quellenangaben überprüfbar sind. Es versteht sich, dass die Quellenangaben in journalistischen Publikationen und Streamingdiensten nicht den formalen Ansprüchen an einen wissenschaftlichen Text entsprechen. Sie müssen deshalb im Einzelfall kritisch prüfen, ob die Angaben ausreichen, um die Quellen dieser Veröffentlichung zu kontrollieren.

Angenommen, in einem Video heißt es: »Nach Immanuel Kant ist nicht nur der menschliche Verstand die Quelle für Erkenntnis, sondern auch die Sinne dienen dazu.« Wenn im Video als Nachweis lediglich der Verweis auf Kant allgemein oder auf seine Kritik der reinen Vernunft angeführt wird, genügt das nicht. Denn damit Sie diese Behauptung des Videos prüfen können, benötigen Sie den Titel, die Auflage und die Seitenzahl des Textes, auf den sich das Video bezieht.

Zuletzt achten Sie darauf, dass Sie immer und ausschließlich mit der aktuellsten Auflage einer Veröffentlichung arbeiten. Jede Veröffentlichung kann Fehler enthalten. Es ist deshalb wahrscheinlich, dass in der neuesten Auflage eventuelle Fehler aus früheren Auflagen korrigiert sind.

\section{Die Berücksichtigung praxisnaher Informationen (z. B. Firmen- und Branchenspezifika) ist gegeben, falls zusätzlich erforderlich}
\label{sec:praxisnahe-informationen}

In vielen Fällen beschäftigen sich die wissenschaftlichen Arbeiten an der Rheinischen Hochschule mit wirtschaftlichen Zusammenhängen. Wenn das auch für Ihre Arbeit gilt, dann wird erwartet, dass Sie spezifische, aktuelle und verlässliche (also: zitierwürdige) Informationen z.B. über ein Unternehmen und seine Branche, Datenblätter oder Produktinformationen verarbeiten.

Sollten diese Informationen vertraulich sein, können Sie Ihre Arbeit vorab sperren lassen (bitte achten Sie auf eventuelle Fristen). Dann wird Ihre Arbeit nur von Ihren Prüfern gelesen und kann danach erst nach Bewilligung durch den Prüfungsausschuss ausgehändigt werden.

\section{Eine kritische Distanz bei der Auswertung der Quellen ist erkennbar}
\label{sec:kritische-distanz}

Das Selbstverständnis in der Wissenschaft geht davon aus, dass sich alle Menschen irren können, dass allen Menschen versehentliche Fehler unterlaufen und dass niemand die Wahrheit für sich gepachtet hat. Deshalb gibt es in der Wissenschaft per se keine Autorität, also keine Instanz, die für sich in Anspruch nehmen kann, dass sie allein wegen ihrer Position (z.B. Macht, Status, Funktion) Recht hat. – In anderen Zusammhängen kann das anders sein: Beispielsweise ist in der katholischen Kirche festgelegt, dass der Papst immer die Wahrheit spricht. –

Wenn Sie wissenschaftlich arbeiten, bedeutet das, dass Sie grundsätzlich gegenüber jeder Aussage kritisch eingestellt sind – auch gegenüber Ihren eigenen Aussagen! Sie unterstellen keine böse Absicht. Sie gehen lediglich davon aus, dass es sich bei der Aussage, die Sie gerade lesen, um einen Irrtum handeln kann. Sie folgen also einer Veröffentlichung in keinem Fall blind und unkritisch, nur weil Sie z.B. von einem erfolgreichen Menschen, einem Genie oder einer Koryphäe geäußert wurde. Auch Albert Einstein hat sich geirrt. Sie werden staunen, wie oft selbst vielzitierte Standardwerke Fehler enthalten.

Sprachlich bringen Sie Ihre kritische Haltung dadurch zum Ausdruck, dass Sie konsequent unpersönlich formulieren. Was Sie beobachten, können prinzipiell alle Menschen beobachten, deshalb kommt das \glqq Ich\grqq{} in Ihrem Text nicht vor.

Die einzige Ausnahme von dieser Regeln kann die Beschreibung Ihrer persönlichen Motivation für die Themenwahl in der Einleitung sein. Aber das ist keine Pflicht, auch die Motivation lässt sich unpersönlich formulieren.

\section{Die exakte Kenntlichmachung aller fremden Quellen durch korrekte, konsistente Zitiertechnik ist gegeben}
\label{sec:kenntlichmachung-quellen}

Weil Ihre Arbeit auf Erkenntnissen beruht, die Sie sich durch die Lektüre von Veröffentlichungen angeeignet haben, ist der Nachweis dieser fremden Aussagen in einheitlicher Form eine weitere zentrale Leistung beim wissenschaftlichen Arbeiten.

Beim Zitieren geht es im Kern darum, dass Sie zwischen Ihren eigenen Gedanken und denen anderer Menschen unterscheiden; dass Sie diese Unterscheidung erkennbar machen; und dass Sie Ihr Publikum dazu in die Lage versetzen, zu prüfen, ob Sie die fremden Gedanken korrekt wiedergegeben haben (weil auch Sie sich irren oder etwas missverstehen können).

Grundsätzlich müssen Sie jede fremde Aussage, die Sie übernehmen, als Zitat kennzeichnen. Es gibt von dieser Regel keine Ausnahme, wie auch die Gerichte in Deutschland immer wieder betonen.\footnote{Zuletzt z.B. Verwaltungsgerichts Berlin, 3. Kammer, Aktenzeichen 3 K 176/20; Arbeitsgericht Bonn, 2. Kammer, Aktenzeichen 2 Ca 345/23.} Deshalb kann es vorkommen, dass in Ihrem Text über längere Strecken fast jeder Satz einen Nachweis enthält.

Wenn Sie versehentlich ein oder zwei Zitate falsch angegeben haben oder falls Ihnen im Laufe der Bearbeitung ein Nachweis verloren gegangen ist, dann wird dies zwar Auswirkungen auf Ihre Note haben. Dramatisch wird es erst, sobald Ihre Prüfer darin ein System erkennen. Fehlende oder fehlerhafte Nachweise als Muster der Textproduktion können im besten Fall als Unfähigkeit gedeutet werden, die wissenschaftliche Methode anzuwenden – dann wird Ihre Arbeit \glqq nur\grqq{} mit 5 bewertet. Wenn sich aber begründen lässt, dass Sie mit Absicht systematisch die Quellen Ihrer Erkenntnis verschwiegen oder verschleiert haben, dann drohen schwerwiegende Konsequenzen. Dafür ist die Anzahl der relevanten Textstellen irrelevant, wichtiger ist das inhaltliche Gewicht dieser Passagen. Die Bewertung Ihrer Arbeit kann bis zu zehn Jahre nach der Abgabe revidiert werden.

Lassen Sie es nicht darauf ankommen und nehmen Sie sich die Zeit, gründlich jede fremde Aussage als Zitat zu kennzeichnen. Wenn Sie von Anfang an mit einem effizienten System (z.B. Zotero) arbeiten, gewinnen Sie nicht nur Zeit, sondern auch Sicherheit. Am Ende lässt sich aber nicht leugnen, dass Wissenschaft Arbeit bedeutet, die Aufmerksamkeit, Sorgfalt, Genauigkeit und Mühe kostet. Planen Sie deshalb genügend Zeit für Ihre Bearbeitung ein.

\section{Wörtliche Zitate sind kurz und selten}
\label{sec:woertliche-zitate}

Auch wenn Ihre Arbeit wesentlich auf Veröffentlichungen beruht, so ist Ihr Text dennoch keine Zitatsammlung. Es ist eine wesentliche Leistung, dass Sie Ihre Gedanken formulieren, die Sie in der kritischen Auseinandersetzung mit der relevanten Literatur entwickelt haben. Um einzelne wichtige Stellen pointiert herauszuarbeiten, kann ein prägnantes wörtliches Zitat angemessen erscheinen. In den meisten Fällen beziehen Sie sich aber auf fremde Aussagen, indem Sie sie paraphrasieren.

\section{Sinngemäße Zitate sind neu formuliert}
\label{sec:sinngemaesze-zitate}

Das sinngemäße oder indirekte Zitat (Paraphrase) ist eine fremde Aussage, die Sie in Ihren eigenen Worten formulieren. Dadurch können Sie Ihre Argumentationskette flüssig aufbauen.

Wenn Sie fremde Aussagen wesentlich unverändert übernehmen, wird dadurch unter Umständen nicht deutlich, ob Sie deren Gehalt verstanden haben und ob Sie sie überhaupt kritisch geprüft haben. Es kann sogar vorkommen, dass ein falsches indirektes Zitat als Plagiat gewertet wird, und wenn es häufiger vorkommt, dass Ihre Arbeitsweise als systematischer Täuschungsversuch aufgefasst wird.

\section{Zu jeder Referenzierung ist eine Angabe im Verzeichnis vorhanden. Umgekehrt sind keine Verzeichniseinträge vorhanden, zu denen die Referenzierung im Text fehlt}
\label{sec:referenzierung-verzeichnis}

Sie führen den Nachweis über die von Ihnen verwendeten Textquellen an zwei Stellen in Ihrer Arbeit:
\begin{enumerate}
\item im Text unmittelbar nach Ihrem wörtlichen Zitat oder Ihrer Paraphrase in der Form einer nummerierten Fußnote und
\item im Anhang in der Form eines alphabetisch sortierten Literaturverzeichnisses (wenn Sie nicht nur schriftliche, sondern auch mündliche und/oder audiovisuelle Textquellen verwenden, legen Sie dafür separate Verzeichnisse an).
\end{enumerate}

Es ist zwingend erforderlich, dass Sie alle Nachweise an diesen beiden Stellen führen: Es gibt keinen Veröffentlichung in einer Fußnote, die im Literaturverzeichnis fehlt, und es gibt keine Eintrag im Literaturverzeichnis, der in keiner Fußnote aufgeführt wird.