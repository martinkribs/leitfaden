\chapter{Thema und Forschungsfrage bzw. Aufgabenstellung, Hypothesen}
\label{chap:thema-forschungsfrage}

In Ihrer Arbeit widmen Sie sich einem Thema. Sie formulieren es als Titel mit höchstens 200 Zeichen, am Ende steht kein Satzzeichen (insbesondere kein Fragezeichen). Ihre Formulierung soll aussagefähig, selbsterklärend und nicht zu allgemein gehalten sein. Firmen- und Produktnamen vermeiden Sie. – Bitte denken Sie daran, dass der Titel nicht identisch ist mit der Forschungsfrage, die Sie untersuchen. Falls Sie keine Forschungsfrage untersuchen, sondern eine Aufgabenstellung bearbeiten, ist es zulässig, dass ihr Wortlaut identisch ist mit dem Titel. Dazu stimmen Sie sich mit Ihren Prüfern ab.

\section{Das Thema ist aktuell und für Wissenschaft sowie Praxis relevant}
\label{sec:thema-aktuell-relevant}

Weil Wissenschaft danach strebt, neue Erkenntnisse zu gewinnen und neue Ergebnisse im Sinne der Aufgabenstellung hervozubringen, können Sie sich nicht mit einer Frage beschäftigen, die bereits beantwortet ist – es sei denn, dass neue Aspekte aufgetreten sind, die bisher noch nicht berücksichtigt wurden.

In Ihrer Einleitung begründen Sie Ihre Themenwahl. Seine Aktualität und Relevanz stehen dabei im Vordergrund.

\section{Das Thema ist anspruchsvoll und herausfordernd}
\label{sec:thema-anspruchsvoll}

Es gibt unzählige Fragen, die zwar nicht beantwortet, aber auch simpel oder sogar banal sind. Wissenschaftliche Zusammenhänge zeichnen sich dadurch aus, dass sie prinzipiell komplex sind und tendenziell grenzenlos erscheinen. Sie erfordern eine differenzierte und präzise Bearbeitung. Einfache Antworten sind dafür unangemessen. In Ihrer Arbeit widmen Sie sich solchen Aufgaben. Inwiefern Ihre Forschungsfrage bzw. Aufgabenstellung diesen Ansprüchen genügt, besprechen Sie mit Ihren Prüfern.

\section{Die Situation bzw. das Problem ist klar und eindeutig formuliert}
\label{sec:problem-klar-formuliert}

Sie formulieren das Problem, das Sie untersuchen, bzw. die Situation, die verändert werden soll, so eindeutig und konkret wie möglich. Wenn Sie den Eindruck haben, dass Ihre Aussage an dieser Stelle weitschweifig, abstrakt und kompliziert ist, dann beherrschen Sie Ihr Thema wahrscheinlich noch nicht. Sie erkennen, dass Sie Ihr Thema im Griff haben, wenn Sie sich klar und verständlich ausdrücken können.

\section{Themenrelevante Forschungslücken werden identifiziert}
\label{sec:forschungsluecken}

Ihre Arbeit beruht auf einer Darstellung des aktuellen Stands der Wissenschaft für Ihr Thema. Dabei referieren Sie nicht nur die gegenwärtig wesentlichen Positionen und Argumente. Sie identifzieren auch die Lücken in der Literatur, die für Ihr Thema relevant sind.

\section{Die Zielsetzung der Arbeit wird dargestellt}
\label{sec:zielsetzung}

Das allgemeine Ziel jeder wissenschaftlichen Arbeit ist Erkenntnisgewinn. Sie müssen deutlich machen, worin das Ziel Ihrer konkreten Arbeit besteht und inwiefern es dazu geeignet ist, den Stand der Erkenntnis über Ihr Thema zu erweitern.

\section{Die Forschungsfrage(n) bzw. Aufgabenstellung und Hypothesen sind eindeutig formuliert und begründet}
\label{sec:forschungsfrage-hypothesen}

Mit Ihrer Forschungsfrage bzw. mit der Vorgehensweise zur Bearbeitung Ihre Aufgabenstellung formulieren Sie Ihr konkretes Vorhaben. Grundsätzlich sind zum Thema Ihrer Arbeit (ausgedrückt im Titel) unendlich viele Forschungsfragen denkbar. Sie widmen sich aber nur einer einzigen.

Weil Ihre Arbeit auf dem aktuellen Stand der Wissenschaft beruhen (und bei Master-Thesen: eine Forschungslücke schließen) muss, können Sie Ihre Forschungsfrage erst formulieren, wenn Sie die relevante Literatur bzw. den aktuellen Stand der Technik kennen: Solange Sie nicht genau wissen, welche Erkenntnisse bereits veröffentlicht wurden, können Sie nicht wissen, welche Erkenntnisse noch fehlen. Es ist deshalb unwahrscheinlich, dass Sie Ihre Forschungsfrage direkt zu Beginn der Bearbeitung präzise formulieren können.

Das gilt auch für die Aufsstellung Ihrer Hypothesen oder Annahmen. Sie entwickeln sie auf der Grundlage der theoretischen Grundlagen in Bezug auf Ihr spezifisches Thema. Hypothesen oder Annahmen funktionieren wie ein Scharnier zwischen der allgemeinen Theorie und der konkreten Wirklichkeit nach dem Muster: Wenn allgemein angenommen wird, dass ..., dann ist in dieser konkreten Situation davon auszugehen, dass...

\section{Methoden und Konzepte sind unabhängig von einem Unternehmen bzw. einem spezifischen Problem}
\label{sec:methoden-unabhaengig}

An jede wissenschaftliche Arbeit richtet sich der Anspruch, dass ihre Erkenntnisse verallgemeinerbar sind. Das bedeutet nicht unbedingt, dass sie allgemeingültig im Sinne statistischer Repräsentativität sind: Dieser Anspruch richtet sich vor allem an Dissertationen und Habilitationen.

Verallgemeinerbar bedeutet: Sie führen Ihre Untersuchung methodisch so durch, dass Sie zu repräsentativen Ergebnissen gelangen könnten, wenn Ihnen dafür die Ressourcen (v.a. Zeit) zur Verfügung stünden. Sie führen z.B. eine Online-Umfrage mit einer Stichprobe von 100 Teilnehmern, deren soziodemographische Merkmale nicht repräsentativ für die deutsche Bevölkerung sind. Ihre Methode passt und ist auch korrekt angewandt, Ihnen fehlen \glqq nur\grqq{} die passenden Teilnehmer, damit Sie eine allgemeingültige Aussage treffen können.

Dieser Anspruch schließt auch jede Untersuchung aus, die methodisch nur auf eine einzige Situation zugeschnitten ist. Viele Unternehmen interessieren sich lediglich für Antworten, die ihre eigene Situation betreffen. Die Übertragbarkeit auf vergleichbare Situationen von anderen Unternehmen ist für sie nicht von Belang. In der Wissenschaft beschäftigen wir uns aber nur mit solchen Fragen, die sich auf vergleichbare Situationen übertragen lassen.