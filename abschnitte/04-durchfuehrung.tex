\chapter{Durchführung}
\label{chap:durchfuehrung}

Dieser Bereich stellt den Kern Ihrer wissenschaftlichen Arbeit dar. Um es mit einem Vergleich zu sagen: Die übrigen Kapitel bilden den Rahmen, während dieser Textteil das eigentliche Bild ist. Sie liefern hier die tatsächliche Umsetzung Ihres Untersuchungsvorhabens, das Sie zuvor vorgestellt haben und nachfolgend auswerten. Ausschlaggebend ist insbesondere die inhaltliche Qualität: Welche Signifikanz haben Ihre Ergebnisse, welchen Wert haben Ihre Lösungen, was ist bei Ihren Untersuchungen herausgekommen, welchen Beitrag zur allgemeinen Erkenntnis konnten Sie leisten bzw. welchen praktischen Nutzen konnten Sie für einen konkreten Anwendungsfall erzeugen?

\section{Eine konkrete Problemstellung ggf. aus der Praxis wird identifiziert, analysiert und strukturiert beschrieben}
\label{sec:problemstellung-beschreibung}

Die Grundlage Ihrer Leistung besteht darin, dass Sie eine konkrete Situation aus der Praxis erkennen, durchdringen und darstellen können. Sie analysieren systematisch die Ausgangssituation, definieren randscharf das Problem – damit ist gemeint: eine Situation, die nicht gewünscht ist und deshalb verändert werden soll – und leiten daraus eine klares Ziel ab. In Ihrer Analyse beleuchten Sie dabei mit angemessenen Anteilen sowohl die theoretischen als auch die praktischen Aspekte des Problems.

\section{Die recherchierten wissenschaftlichen Methoden werden auf den Untersuchungsgegenstand angewandt}
\label{sec:methoden-anwendung}

Um Ihr Ziel zu erreichen, wählen Sie eine etablierte wissenschaftliche Methode oder sie kombinieren mehrere. Sie begründen nachvollziehbar, warum diese Wahl am besten geeignet ist für Ihren Untersuchungsgegenstand und Ihr Ziel. Durch die korrekte Anwendung der Methode(n) führen Sie vor, dass Sie diese sicher beherrschen.

\section{Mittels einer geeigneten, systematischen und strukturierten Vorgehensweise wird für die Problemstellung eine Lösung entwickelt}
\label{sec:loesung-entwicklung}

Das Ziel Ihrer Arbeit besteht darin, eine Antwort auf eine Frage zu formulieren, oder anders gesagt: eine Lösung für ein Problem zu finden. Diese Antwort bzw. Lösung enthält immer eine Erkenntnis. Es ist auch denkbar, dass das Problem ganz anders gelöst wird: durch Intuition, Zufall oder Magie. Aber dann wäre es eben keine wissenschaftliche Vorgehensweise. Ihre Arbeit ist deshalb im Grunde nichts anderes als eine systematische Dokumentation aller einzelnen Entwicklungsschritte, die sie auf dem Weg zu Ihrer Antwort vollzogen haben bzw. ein Arbeitsbericht mit dem Nachweise Ihrer Methode und der Dokumentation Ihrer Ergebnisse.

\section{Problemanalyse, Methodenauswahl und -anwendung sowie Entwicklung des Lösungsansatzes erfolgen korrekt und eigenständig}
\label{sec:korrekte-eigenstaendige-durchfuehrung}

Mit dieser Anforderung werden zwei Selbstverständlichkeiten für die Bewertung Ihrer Leistung zum Ausdruck gebracht: Je weniger Fehler Ihnen bei der Durchführung Ihrer Arbeit unterlaufen, desto besser. Darüber hinaus müssen Sie Ihre Arbeit eigenständig erbracht haben (dafür stehen Ihnen Hilfsmittel zur Verfügung, wie z.B. Veröffentlichungen oder KI, aber diese müssen Sie vollständig angeben), andernfalls handelt es sich um einen schwerwiegenden Täuschungsversuch, der zum Nichtbestehen führt.

\section{Der entwickelte Ansatz ist zur Lösung der vorliegenden Problemstellung geeignet und umsetzbar}
\label{sec:ansatz-geeignet}

Es kommt bisweilen vor, dass die Bearbeitung zwar ein Ergebnis hervorbringt, aber dies ist keine Lösung für das von Ihnen identifizierte Problem (sondern für ein anderes Problem). Stellen Sie deshalb sicher, dass Ihr entwickelter Ansatz auch tatsächlich die von Ihnen definierte Situation effektiv addressiert. Unter Umständen kann es erforderlich sein, dass dazu auch eine kritische Prüfung der praktischen Umsetzbarkeit in der Praxis sowie die Diskussion möglicher Limitationen Ihres Vorschlags zählt. Im Zweifel stimmen Sie sich darüber mit Ihren Prüfern ab.

\section{Vorgehensweise und Problemlösung werden in strukturierter Form schriftlich und für den kundigen Fachleser verständlich dokumentiert}
\label{sec:dokumentation}

Sie müssen davon ausgehen, dass Sie sich mit Ihrer Arbeit an ein Fachpublikum richten. Deshalb wird von Ihnen erwartet, dass Sie Ihre Arbeit in einer klaren, nüchternen und präzisen Fachsprache verfassen. Außerdem wählen Sie eine für Fachleute angemessene Darstellungstiefe. Eine logische und angemessene Struktur Ihrer Ausführungen hilft Ihrem Publikum dabei, Ihre Argumentation effizient nachzuvollziehen: Sie möchten ihm eine zeitraubende Lektüre umständlicher Gedankengänge ersparen. Weitere Hinweise zu diesem Punkt finden Sie auch unter Abschnitt 6.

\section{Abbildungen, Diagramme, Grafiken und Schaubilder sind in ausreichender Zahl enthalten. Sie sind formal und inhaltlich korrekt, mit korrekter Quellenangabe versehen und ergänzen bzw. illustrieren den Text in angemessener Weise}
\label{sec:abbildungen}

Weitere Informationen hierzu finden Sie unter Punkt 6.7.

\section{Zusätzlich bei Masterthesis: Gegenüber einer Bachelorarbeit wird ein deutlich höherer Abstraktionsgrad erreicht}
\label{sec:masterthesis-abstraktionsgrad}

In einer Bachelorthesis führen Sie vor, dass Sie den Stand der Forschung kennen und das wissenschaftliche Arbeiten beherrschen. Inhaltlich bewegen Sie sich auf einem deskriptiven Niveau. Das bedeutet: Sie referieren Erkenntnisse aus dem relevanten wissenschaftlichen Diskurs.

In einer Masterthesis liefern Sie darüber hinaus auch analytische Anteile. Sie zeigen darin, dass Sie die allgemeinen Erkenntnisse auf einen neuen Sachverhalt anwenden können. Darin besteht im wesentlichen die höhere Abstraktionsleistung einer Masterthesis: Sie sind dazu in der Lage, die relevanten allgemeingültigen Aussagen in einem konkreten Kontext sinnvoll und passend zu verorten und daraus neue Schlüsse zu ziehen. – Im Sinne der Statistik können sowohl deskriptive als auch analytische Anteile sowohl in einer Bachelor- als auch in einer Masterthesis enthalten sein.

\section{Zusätzlich bei Masterthesis: Die anhand einer konkreten Problemstellung erarbeiteten Methoden, Verfahren oder Konzepte sind zumindest in Teilaspekten unabhängig vom Anwendungsfall und über diesen hinaus gültig. Sie sind auf ähnlich gelagerte Fälle übertragbar}
\label{sec:masterthesis-uebertragbarkeit}

Der Anspruch an eine Masterthesis besteht nicht darin, dass Sie allgemeingültige, d.h. repräsentative Aussagen treffen. Es wird aber erwartet, dass Ihre Erkenntnisse und Antworten bzw. Lösungen über den von Ihnen untersuchten Einzelfall hinaus auch in anderen Situationen angewandt werden können, welche mit Ihrer Situation vergleichbar sind. So könnte etwa der Vorschlag für das Unternehmen, dessen Fall Sie bearbeitet haben, auch für unmittelbare Wettbewerber gültig sein (wenn die Unternehmen in den für Sie relevanten Aspekten vergleichbar sind). Was das für Ihre Masterthesis konkret bedeutet, hängt von der Abstimmung mit Ihren Prüfern ab.

\section{Zusätzlich bei Masterthesis: Sie werden am konkreten Anwendungsfall validiert}
\label{sec:masterthesis-validierung}

Die Entwicklung einer Lösung für das von Ihnen untersuchte Problem bedeutet auch, dass Sie die Anwendung der Lösung in Ihre Arbeit einbeziehen und untersuchend oder zumindest gedanklich deren Validität überprüfen und die Ergebnisse kritisch bewerten.