\chapter{Konzeptionell-theoretische Grundlagen}
\label{chap:konzeptionell-theoretische-grundlagen}

Dieser Bereich Ihrer Leistungen beeinflusst die Bewertung stärker als die beiden vorigen. Hier zeigt sich besonders deutlich, ob Sie das wissenschaftliche Arbeiten beherrschen. Deshalb können hierbei Fehler gravierende Folgen haben.

Die theoretischen Grundlagen nehmen etwa die Hälfte Ihres Textes ein. Abweichungen sind, nach Absprache mit Ihre Prüfern, möglich.

\section{Ein konzeptionell-theoretischer Anteil ist vorhanden}
\label{sec:konzeptionell-theoretischer-anteil}

Es ist nicht möglich, mit einer wissenschaftlichen Arbeit zu bestehen, die ohne konzeptionell-theoretische Fundierung auskommt. Wenn dieser Teil Ihrer Arbeit erhebliche inhaltliche und/oder formale Mängel aufweist, ist die Wahrscheinlichkeit hoch, dass sie mit der Note 5 bewertet wird.

\section{Definitionen zentraler Begriffe sind vorhanden}
\label{sec:definitionen-begriffe}

Definitionen sind Abgrenzungen des Gehalts von Begriffen. Mit Definitionen wird geklärt, was gemeint ist. Deshalb sind Definitionen fürs wissenschaftliche Arbeiten unerlässlich. Kategorien, Kriterien und Konzepte müssen definiert werden, bevor eine Analyse stattfinden kann.

Diese Definitionen formulieren Sie in der Regel nicht selbst (Ausnahmen klären Sie mit Ihren Prüfern): Sie beziehen sich auf bestehende Definitionen. Dadurch, dass Sie die maßgeblichen Definitionen in ihrem Zusammenhang darstellen, machen Sie für Ihr Publikum nachvollziehbar, inwiefern Sie den Diskurs zu Ihrem Thema beherrschen.

\section{Eine Darlegung der für die Untersuchung benötigten Grundlagen und Forschungsstände ist vorhanden}
\label{sec:darlegung-grundlagen}

Höchstwahrscheinlich liegen auch zu Ihrem Thema konkurrierende Theorien vor (mit unterschiedlichen Definitionen), die sich teilweise ergänzen oder widersprechen und unterschiedliche Schwerpunkte setzen. Oder es gibt technische Rahmenbedingungen, die für Ihre Bearbeitung grundlegend sind. Sie stellen diesen aktuellen Stand der Forschung bzw. der Technik dar und entscheiden sich danach für eine oder mehrere Theorien bzw. Techniken als Grundlage Ihrer folgenden Analyse, empirischen Studie oder Bearbeitung.

\section{Die Theoriewahl bzw. Vorgehensweise wird inhaltlich sinnvoll begründet}
\label{sec:theoriewahl-begruendung}

Ihre Entscheidung für eine oder mehrere Theorien, der bzw. denen Sie in Ihrem weiteren Text folgen, begründen Sie überzeugend. Respektive: Sie begründen Ihre Vorgehensweise bzw. Ihr Programm zur Bearbeitung der Aufgabenstellung. Ihre Wahl darf also weder willkürlich ausfallen, noch dürfen dafür praktische oder Bequemlichkeitsgründe ausschlaggebend sein. Den logisch zwingenden Zusammenhang zwischen Ihrer Entscheidung und Ihrem Thema sowie Ihrer Forschungsfrage müssen Sie nachvollziehbar darlegen.

\section{Die wesentlichen Themenaspekte werden berücksichtigt, eine Reduktion auf bestimmte Teilaspekte wird begründet}
\label{sec:themenaspekte-begruendung}

Jedes Thema kann prinzipiell uferlos ausgeführt werden, wenn Sie sich davon leiten lassen, dass mittelbar alles mit allem zusammenhängt. In Ihrer wissenschaftlichen Arbeit setzen Sie aber Ihrem Thema Grenzen (indem Sie nur ausgewählten Theorien und deren Definitionen folgen). Daraus ergibt sich zwangsläufig, dass Sie nicht alle Aspekte Ihres Themas bearbeiten können. Ihre Entscheidung, was für Ihr Thema wesentlich ist, stellt einen wichtigen Teil Ihrer Prüfungsleistung dar. Sie müssen deshalb Ihre Auswahl überzeugend begründen.

\section{Zusätzlich bei Master-Thesis: Die Arbeit identifiziert Lücken im Stand der Forschung, die im Verlauf geschlossen werden sollen}
\label{sec:masterthesis-forschungsluecken}

Der höhere Anspruch an eine Master-Thesis (verglichen mit einer Bachelor-Thesis) drückt sich u.a. darin aus, dass Sie darin eine eigenständige Forschung durchführen. Auch wenn diese Untersuchung eng begrenzt ist, so beantworten Sie damit doch eine Frage, die bisher in der wissenschaftlichen Literatur noch nicht beantwortet wurde. – In einer Bachelor-Thesis ist diese Leistung nicht erforderlich. Die deskriptiven Anteile sind darin größer und die analytischen Teile sind kleiner als bei einer Master-Thesis. – Im Sinne der Statistik können sowohl deskriptive als auch analytische Anteile sowohl in einer Bachelor- als auch in einer Masterthesis enthalten sein.