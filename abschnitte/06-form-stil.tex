\chapter{Form und Stil}
\label{chap:form-stil}

Was Sie sagen und was Ihr Publikum versteht, hängt untrennbar damit zusammen, wie Sie es sagen. Es genügt deshalb nicht, dass Sie Ihre Aussagen nur in zutreffende Worte fassen. Ebenso wichtig ist, dass die gesamte äußere Form der Arbeit Ihrer Sorgfalt und Genauigkeit entspricht sowie der Bedeutung, die Sie dieser Prüfungsleistung beimessen.

\section{Die Ausdrücke sind eindeutig und präzise, es werden keine umgangssprachlichen Formulierungen verwendet}
\label{sec:ausdruecke-praezise}

Weil Sprache nicht so eindeutig ist wie mathematische Formeln, sollten Sie Ihre Aufmerksamkeit darauf richten, Ihre Aussagen so klar und verständlich wie möglich zu formulieren. Das gelingt Ihnen, wenn Sie unter anderem diese Richtlinien berücksichtigen:
\begin{itemize}[label={--}]
\item Verwenden Sie aussagekräftige und aktive Verben.
\item Wägen Sie bei jeder Substantivierung ab, ob sie Ihre Aussage schärft oder verunklart.
\item Bevorzugen Sie kurze Hauptsätze.
\item Vermeiden Sie Konstruktionen mit vielen Nebensätzen.
\item Lösen Sie umständliche Passivkonstruktionen auf in mehrere Hauptsätze mit aktiven Aussagen.
\item Formulieren Sie konkret.
\item Lassen Sie die handelnden Akteure deutlich hervortreten.
\item Verzichten Sie auf umgangssprachliche Formulierungen (\glqq Klamotten\grqq{} anstelle von \glqq Kleidung\grqq{}) und Redewendungen (\glqq In der Nacht sind alle Katzen grau\grqq{}).
\end{itemize}

Grundsätzlich schreiben Sie im Präsens. Das gilt auch, wenn Sie veröffentlichte Aussagen zitieren: \glqq Immanuel Kant plädiert dafür, die eigene Meinung auf der Basis guter Gründe zu vertreten.\grqq{}

Es kann vorkommen, dass es Ihnen angemessen erscheint, eine historische Entwicklung nachzuzeichnen, z.B. die Geschichte einer Theorie. In diesem Fall verwenden Sie die Vergangenheitsform: \glqq Demokrit beschrieb im 4. Jh. v. Chr. ein Teilchenmodell. Daran knüpfte Aristoteles mit der Lehre von den vier Elementen an.\grqq{}

Sie formulieren allerdings auf keinen Fall im sog. historischen Präsens: \glqq Sigmund Freund erlebt viele Jahre lang starke Ablehnung seiner Ideen, bis sich allmählich ein Kreis aus Anhängern um ihn bildet.\grqq{}

\section{Die Spracheffizienz ist gegeben}
\label{sec:spracheffizienz}

In wissenschaftlichen Texten ist Ökonomie eine Tugend: Sie verzichten auf alles, was überflüssig ist. Das gelingt Ihnen, indem Sie Ihre verfassten Passagen schon während der Bearbeitung immer wieder darauf kritisch prüfen, ob Sie sie kürzen und durch treffendere Begriffe schärfen können.

Achten Sie darauf, keine Phrasen zu verwenden. Das sind Formulierungen, deren Wahrheitsgehalt nicht bewiesen werden muss, weil er offensichtlich ist: \glqq Wer ohne Schirm durch den Regen geht, wird nass.\grqq{} Der Gehalt dieser Aussage ist ohne jeglichen Erkenntnisgewinn. Weil Sie aber danach streben, zum Erkenntnisgewinn beizutragen, verzichten Sie auf Phrasen. – Oft erscheint es verlockend, solche Allgemeinplätze in einleitenden Passagen zu verwenden. Geben Sie dieser Versuchung nicht nach. Vielleicht finden Sie hierfür eine aktuelle Information, die zu Ihrem Thema passt (Zitierwürdigkeit beachten): \glqq Eine aktuelle repräsentative Studie zum Medienkonsum in Deutschland verdeutlicht die Bedeutung von TikTok für die politische Meinungsbildung.\grqq{}

Sie vermeiden Wiederholungen. Sie dürfen und müssen erwarten, dass Ihr Publikum fachkundig ist und sich der Lektüre Ihres Textes aufmerksam und konzentriert zuwendet. Wiederholungen lassen deshalb vermuten, dass Sie Ihre Argumentationskette nicht sorgfältig genug konstruiert haben oder dass Sie Ihrem Publikum mangelnde Aufmerksamkeit oder Vergesslichkeit unterstellen.

\section{Eine klare Gedankenführung ist erkennbar}
\label{sec:gedankenfuehrung}

Das Ziel Ihrer Textproduktion besteht darin, eine schlüssige Argumentationskette vorzulegen, um Ihr Publikum zu überzeugen. Ein Argument ist eine Aussage. Eine Argumentationskette ist also eine Aneinanderreihung von Aussagen. Überzeugend wirkt diese nicht nur durch die angeführten Beweise für Ihre Behauptungen, sondern auch dadurch, dass jede Aussage logisch nahtlos an die vorige Aussage anschließt (schlüssig).

Die Faustregel \glqq Eine Aussage, ein Satz\grqq{} hilft Ihnen bei der Formulierung Ihrer Aussagen und Ihrem Publikum bei deren Lektüre.

Eine zweite Faustregel lautet: \glqq Ein Gedanke, ein Absatz.\grqq{} Es ist unwahrscheinlich, dass Sie für die Formulierung eines Gedankens mehr als ein paar Sätze benötigen. Wenn Sie also eine Seite geschrieben haben ohne trennende Absätze, dann prüfen Sie bitte, an welchen Stellen Sie durch einen harten Zeilenumbruch deutlich machen sollten, dass hier ein neuer Gedanke beginnt.

Vermeiden Sie Verweise auf frühere oder spätere Stellen in Ihrem Text: \glqq Wie bereits in Kapitel 3 ausgeführt, ...\grqq{} oder \glqq Zur Analyse vgl. das folgende Kapitel 7.\grqq{} Mit Verweisen auf frühere Textstellen unterstellen Sie Ihrem Publikum Unaufmerksamkeit. Verweise auf folgende Stellen deuten auf Mängel in der Stringenz Ihrer Argumentationskette hin und lassen vermuten, dass es Ihnen noch nicht gelungen ist, jede Aussage an die passende Stelle zu platzieren.

\section{Eine angemessene Nutzung von Fremdwörtern und einschlägigen Fachausdrücken ist vorhanden}
\label{sec:fremdwoerter-fachausdruecke}

Fachsprache ist für wissenschaftliche Texte unverzichtbar. Sie hilft Ihnen dabei, kurz und genau zu formulieren.

Lassen Sie sich nicht davon irritieren, wenn Sie die gleichen Fachbegriffe häufig wiederholen. Abwechslung im Sprachgebrauch ist in einem wissenschaftlichen Text keine Tugend (im Unterschied zur journalistischen Reportage oder zum Roman). Sie verwenden bitte immer die gleichen Fachbegriffe für die gleichen Sachverhalte, um eindeutige und unmissverständliche Aussagen zu treffen.

Sie formulieren sachlich und nüchtern. Sie verzichten prinzipiell auf wertende Adjektive (\glqq die renommierte Forscherin\grqq{}, \glqq das erfolgreiche Unternehmen\grqq{}) und Steigerungen (\glqq die ganz besondere Marke\grqq{}).

Ihr Text muss nicht unterhalten. Deshalb sind entsprechende stilistische Empfehlungen, die für andere Textsorten gelten, fürs wissenschaftliche Schreiben nicht hilfreich. Sie konstruieren in keinem Fall einen Spannungsbogen. Ihr Publikum braucht keine Überraschung. Und Sie enthalten sich jeder Ironie.

Sie trennen strikt zwischen Darstellung und Bewertung. Ihr gesamter Text ist eine distanzierte und nüchterne Darstellung. Erst am Ende, nach der Beschreibung Ihrer Untersuchungsergebnisse, können Sie diese in einem separaten Abschnitt bewerten. Diese Aussagen kennzeichnen Sie dann ausdrücklich mit der Überschrift: Bewertung der Ergebnisse.

\section{Eine korrekte Anwendung der Regeln der Rechtschreibung, Grammatik und Interpunktion ist gegeben}
\label{sec:rechtschreibung}

Zur formalen Korrektheit Ihres Textes zählt sprachliche Fehlerfreiheit. Dies ist ein weiterer Aspekt, unter dem Ihr Publikum die Qualität Ihrer Arbeit bewertet.

Künstliche Intelligenz hilft Ihnen dabei, Ihren Text schnell und einfach auf Orthographie, Grammatik und Interpunktion zu prüfen. Denken Sie daran, die entsprechenden Prompts nachzuweisen, Hinweise dazu finden Sie in der Praktische(n) Anleitung: Zitieren in wissenschaftlichen Arbeiten der Rheinischen Hochschule Köln. Gleichzeitig muss Ihnen klar sein, dass die Ergebnisse der KI nur Vorschläge sind, auf die Sie sich nicht verlassen können.

\section{Eine korrekte äußere Form ist vorhanden}
\label{sec:aeuszere-form}

Sie drucken Ihre Arbeit auf weißes Papier, 80 bis 110 g, DIN A4 (hochformatig) einseitig auf der rechten Seite. Der Text steht im einspaltigen Blocksatz. Große Lücken, die das Schriftbild stören, vermeiden Sie, indem Sie die letzte Zeile eines Absatzes nicht bündig zum rechten Blockrand setzen, sondern sie auslaufen (flattern) lassen. Den Seitenrändern geben Sie folgende Breiten:
\begin{itemize}[label={--}]
\item links: 3,5 cm,
\item rechts: 2,5 cm,
\item oben: 3 cm,
\item unten: 2 cm.
\end{itemize}

Sie wählen eine sachliche Groteskschrift wie z.B. Helvetica oder Arial oder eine Antiquaschrift wie z.B. Times mit folgenden Auszeichnungen:
\begin{itemize}[label={--}]
\item Fließtext: regulärer oder normaler Schriftschnitt, Schriftgröße 11 pt, Zeilenabstand 1,5 Linien, 6 pt Abstand nach dem Absatz;
\item Überschriften: linksbündig, fetter Schriftschnitt, Schriftgröße 11 pt, Zeilenabstand 1,5 Linien, 6 pt Abstand nach dem Absatz;
\item Kopfzeilen mit Seitenzahl und Fußnoten: regulärer oder normaler Schriftschnitt, Schriftgröße 9 pt, Zeilenabstand 1,5 Linien, kein Abstand nach dem Absatz.
\end{itemize}

Nach einem Absatz folgt eine Freizeile als Abstand zum nächsten Absatz oder zum nächsten Unterabschnitt. Jedes Kapitel beginnt auf einer neuen Seite.

Ihr Text hat keine Überschrift als letzte Zeile auf einer Seite. Ihr Text hat auch keine einzelne Zeile als erste oder als letzte Zeile auf einer Seite. Sollten Sie am Ende Ihrer Bearbeitung, nachdem Sie alle anderen Kontrollen durchgeführt haben, solche einzelnen Zeilen bemerken, dann sorgen Sie mit harten Zeilenumbrüchen dafür, dass dieser typographische Fauxpas behoben wird.

Akademische Grade und Titel nennen Sie an keiner Stelle, weder im Text noch in den Fußnoten oder im Literaturverzeichnis. Ob die von Ihnen interviewte Expertin eine Professorin ist oder der Verfasser eines zitierten Textes einen Doktorgrad führt, ist völlig unerheblich. Die einzigen beiden Ausnahmen für diese Regel: Falls Sie bereits einen akademischen Abschluss erworben haben, nennen Sie diesen auf dem Titel, in der Eigenständigkeitserklärung und in Ihrem Lebenslauf. Das gilt auch für Ihre Prüfer auf den Titeln.

Der Schmutztitel (Vorderdeckel, auf Karton gedruckt) enthält folgende Informationen:
\begin{itemize}[label={--}]
\item Logo und Schriftzug der Rheinischen Hochschule Köln als Abbildung,
\item Bezeichnung des Fachbereichs (fetter Schriftschnitt, 14 pt)
\item Bezeichnung des Studiengangs (regulärer Schriftschnitt, 14 pt), danach eine Freizeile
\item Hausarbeit, Projektarbeit, Bachelor-Thesis oder Master-Thesis (fetter Schriftschnitt, 14 pt)
\item Thema der Arbeit (regulärer Schriftschnitt, 14 pt), danach eine Freizeile
\item vorgelegt von (regulärer Schriftschnitt, 11 pt)
\item Ihr Vor- und Nachname (regulärer Schriftschnitt, 14 pt)
\item Ihre Matrikelnummer (regulärer Schriftschnitt, 11 pt), danach eine Freizeile
\item Sommer- oder Wintersemester
\end{itemize}

\begin{figure}[h]
\centering
\caption[Beispielhafter Schmutztitel und Haupttitel für Haus- und Abschlussarbeiten an der Rheinischen Hochschule Köln,]{Beispielhafter Schmutztitel (li.) und Haupttitel (re.) für Haus- und Abschlussarbeiten an der Rheinischen Hochschule Köln, eigene Darstellung.}
\label{fig:schmutztitel-haupttitel}
\end{figure}

Der Haupttitel (auf weißem Papier wie der gesamte Inhalt gedruckt) enthält die gleichen Informationen wie der Schmutztitel mit einem Unterschied: Auf dem Haupttitel nennen Sie nach der Matrikelnummer noch die Prüfer (regulärer Schriftschnitt, 11 pt, danach eine Freizeile).

Ihre Arbeit lassen Sie durch eine Klebebindung mit unbedrucktem Textilrücken binden.

Die Farbe des Kartons für den Vorder- und Rückdeckel richtet sich nach Ihrem Studiengang. Sie finden die korrekte Farbe für Ihre Arbeit in der folgenden Tabelle \ref{tab:einbandfarben}.

\begin{table}[H]
\label{tab:einbandfarben}
\begin{tabular}{|l|l|l|}
\hline
\textbf{Fachbereich} & \textbf{Studiengang} & \textbf{Einbandfarbe} \\
\hline
\multirow{5}{*}{Ingenieurwesen} & Elektrotechnik (B.Eng.) & Schwarz \\
\cline{2-3}
 & Informatik (B.Sc.) & Hellgrün \\
\cline{2-3}
 & Maschinenbau (B.Eng.), alle Fachrichtungen & Dunkelblau \\
\cline{2-3}
 & Prozesstechnik (B.Sc.) & Hellgrün \\
\cline{2-3}
 & Wirtschaftsingenieurwesen (B.Eng.) & Dunkelgrün \\
\cline{2-3}
 & Alle Masterstudiengänge & Dunkelblau \\
\hline
\multirow{17}{*}{\begin{tabular}[c]{@{}l@{}}Wirtschaft,\\ Psychologie \&\\ Recht\end{tabular}} & Betriebswirtschaftslehre (B.A.) & Gelb \\
\cline{2-3}
 & Compliance \& Corporate Security (LL.M.) & Rot \\
\cline{2-3}
 & Digital Transformation Management (M.A.) & Gelb \\
\cline{2-3}
 & Entrepreneurship (M.A.) & Gelb \\
\cline{2-3}
 & General Management (M.A.) & Gelb \\
\cline{2-3}
 & International Business Management (M.A.) & Gelb \\
\cline{2-3}
 & MBA International Business (MBA) & Gelb \\
\cline{2-3}
 & Nachhaltigkeitsmanagement (B.A.) & Gelb \\
\cline{2-3}
 & Psychologie (B.Sc.) & Orange \\
\cline{2-3}
 & Steuerrecht (LL.M.) & Rot \\
\cline{2-3}
 & Unternehmensmanagement (B.A.) & Gelb \\
\cline{2-3}
 & Werteorientierte Unternehmensführung (M.Sc.) & Gelb \\
\cline{2-3}
 & Wirtschaftsinformatik (B.Sc.) & Grau \\
\cline{2-3}
 & Wirtschaftsinformatik (M.Sc.) & Grau \\
\cline{2-3}
 & Wirtschaftspsychologie (B.Sc.) & Orange \\
\cline{2-3}
 & Wirtschaftspsychologie (M.Sc.) & Orange \\
\cline{2-3}
 & Wirtschaftsrecht (LL.B.) & Rot \\
\hline
\multirow{6}{*}{\begin{tabular}[c]{@{}l@{}}Medien, Marketing \&\\ Innovation\end{tabular}} & Digital Business Management (M.A.) & Weiß \\
\cline{2-3}
 & International Marketing \& Media Management (M.A.) & Weiß \\
\cline{2-3}
 & Media \& Marketing Management (B.A.) & Weiß \\
\cline{2-3}
 & Mediendesign (B.A.) & Braun oder eigene \\
\cline{2-3}
 & User Experience Design (M.A.) & Weiß oder eigene \\
\hline
\multirow{5}{*}{\begin{tabular}[c]{@{}l@{}}Medizinökonomie \&\\ Gesundheit\end{tabular}} & Erweiterte Pflegepraxis (B.Sc.) & Magenta \\
\cline{2-3}
 & Gesundheitsökonomie (M.Sc.) & Magenta \\
\cline{2-3}
 & \begin{tabular}[c]{@{}l@{}}Medizinökonomie \& Digitales Management (B.Sc.)\\ (ehem. Medizinökonomie)\end{tabular} & Magenta \\
\cline{2-3}
 & Molekulare Biomedizin (B.Sc.) & Magenta \\
\cline{2-3}
 & Physiotherapie (B.Sc.) & Magenta \\
\hline
\end{tabular}
\caption{Übersicht der Einbandfarben für Haus- und Abschlussarbeiten an der Rheinischen Hochschule Köln.}
\end{table}

\section{Die Tabellen und Abbildungen sind gut lesbar}
\label{sec:tabellen-abbildungen}

Abbildungen (z.B. Fotos, Illustrationen, Schemata, Diagramme, Zeichnungen, Pläne oder Screenshots) und Tabellen sind erwünscht, weil sie verbale Aussagen veranschaulichen bzw. einen kompakten Überblick bieten können. Abbildungen sind sogar erforderlich, wenn Ihre Aussagen ohne sie nicht völlig nachvollzogen werden können. Schemata und Grafiken können Abläufe und Zusammenhänge sichtbar machen. Es gibt keine Abbildung ohne Bezug zu Ihrem Text.

Die Größe der einzelnen Abbildung hängt davon ab, ob sie einen überwiegend illustrativen oder informativen Charakter hat.
\begin{itemize}[label={--}]
\item Mit illustrativ ist gemeint, dass das Bild vor allem verbale Aussagen anschaulich macht und das Textverständnis dadurch fördert. Eine typische illustrative Abbildung ist ein Schaubild mit einer Vielzahl aktueller Online-Medien (Blogs, Websites, Streamingplattformen, Gamingplattformen, Social-Media-Kanäle etc.). Sie wird oft zur Veranschaulichung der Aussage gezeigt, dass es gegenwärtig unübersichtlich viele Online-Medien gibt. Diese Aussage ist auch ohne die Abbildung nachvollziehbar. In einem solchen Fall sollte die Abbildung nicht größer sein als ein Viertel der Seite. Sie platzieren sie mittig auf der Seite. Abbildung \ref{fig:libelle} ist dafür ein Beispiel. – Wenn Sie feststellen, dass Texte innerhalb des Bildes dann nicht mehr lesbar wären, verzichten Sie auf diese Abbildung.
\item Mit informativ ist gemeint, dass das Bild fürs Textverständnis wichtig oder sogar unverzichtbar ist. Das kann der Fall sein, wenn Sie eine Bildanalyse durchführen oder wenn Sie Details in einem Bild zum Gegenstand Ihrer Aussagen machen. Dann zeigen Sie die Abbildung in einer passenden Größe, eventuell sogar seitenfüllend.
\end{itemize}

Ihr Publikum freut sich über einen einheitlichen gestalterischen Duktus bei Tabellen und allen Formen von Abbildungen. Einheitlichkeit ist wichtiger als Finesse im Detail, es genügt eine schlichte Darstellung mit Grauschattierungen für Flächen oder Linien.

Unmittelbar unter jeder Tabelle und Abbildung steht eine nummerierte Unterschrift. Achten Sie darauf, dass diese Bild- oder Tabellenunterschrift nicht erst auf der folgenden Seite steht. Bitte kontrollieren Sie mit der Praktische(n) Anleitung: Zitieren in wissenschaftlichen Arbeiten der Rheinischen Hochschule oder dem Chicago Manual of Style, dem Reference Guide des IEEE bzw. dem Publication Manual der APA, ob Sie Ihre Bildunterschriften korrekt geschrieben haben.

\begin{figure}[h]
\centering
\caption[Veranschaulichung des Panorama-Sichtfelds einer Libelle, die visuelle Reize ihrer Umwelt mit zwei komplexen Facettenaugen und drei einfachen Linsenaugen aufnimmt]{Veranschaulichung des Panorama-Sichtfelds einer Libelle, die visuelle Reize ihrer Umwelt mit zwei komplexen Facettenaugen und drei einfachen Linsenaugen aufnimmt, Quelle: Journal of Neuroscience 28, Nr. 11 (12. März 2008), Titelbild, Quelle: https://www.jneurosci.org/content/28/11.cover-expansion.}
\label{fig:libelle}
\end{figure}

\section{Die erforderlichen Verzeichnisse sind formal korrekt}
\label{sec:verzeichnisse}

Das Inhaltsverzeichnis enthält die Überschriften aller Bestandteile Ihrer Arbeit (evtl. Vorbemerkung, Einleitung, alle Kapitel, Abschnitte und Unterabschnitte sowie Verzeichnisse und evtl. Anhänge) mit der Zahl der Seite, an der diese beginnen. Sie können als Überschrift dafür Inhaltsverzeichnis oder kurz Inhalt verwenden. Achten Sie darauf, dass das Inhaltsverzeichnis selbst kein Eintrag im Inhaltsverzeichnis ist.

Das Abkürzungsverzeichnis enthält nur Abkürzungen jenseits des allgemeinen Sprachgebrauchs (wie Abb., etc., evtl., sog., Tab., z.B.) sowie deren Auflösung.

Wie Sie Ihr Literaturverzeichnis korrekt aufbauen, finden Sie in der Praktische(n) Anleitung: Zitieren in wissenschaftlichen Arbeiten der Rheinischen Hochschule Köln.

Alle anderen Verzeichnisse sind einfache tabellarische Listen, die Sie nach dem Muster erstellen, welches Sie in Abbildung \ref{fig:verzeichnisse} finden:

\begin{figure}[h]
\centering
\caption[Muster für die Darstellung von Abbildungs-, Tabellen-, Formel- und Symbolverzeichnissen]{Muster für die Darstellung von Abbildungs-, Tabellen-, Formel- und Symbolverzeichnissen, eigene Darstellung.}
\label{fig:verzeichnisse}
\end{figure}

\section{Die Regeln zum Umfang werden eingehalten}
\label{sec:umfang}

Der Umfang Ihrer Arbeit ergibt sich bei Haus- und Projektarbeiten aus den Vorgaben Ihrer Prüfer.
\begin{itemize}[label={--}]
\item Ihre Bachelorthesis hat einen Umfang von 135.000 Zeichen inkl. Leerzeichen (das entspricht ungefähr 60 Seiten je 2.300 Zeichen inkl. Leerzeichen).
\item Ihre Masterthesis hat einen Umfang von 185.000 Zeichen inkl. Leerzeichen (ca. 80 Seiten Text).
\end{itemize}

Damit ist der Text von der Einleitung bis zum Schluss gemeint. Titel, Verzeichnisse und Anhänge werden hierbei nicht mitgezählt. Inwiefern eine Toleranz von zehn Prozent Minder- oder Mehrumfang akzeptabel sein kann, besprechen Sie mit Ihren Prüfern.