\chapter{Strukturierung}
\label{chap:strukturierung}

Die Struktur erfüllt zwei Funktionen. Sie hilft Ihnen im während der Bearbeitung dabei, den Überblick über Ihre Argumentationskette zu behalten: Befinden sich alle Aussagen an der richtigen Stelle, damit Ihre Argumentationskette schlüssig und stringent ist? Und Ihrem Publikum hilft sie dabei, Ihrer Argumentation zu folgen: Ihr roter Faden dient beim ersten Blick ins Inhaltsverzeichnis und bei der fortlaufenden Lektüre als Orientierungshilfe.

Grundsätzlich ist Ihre Arbeit in dieser Reihenfolge strukturiert:
\begin{itemize}[label={--}]
\item Vorderes Deckblatt (Fachbegriff: Schmutztitel, auf Karton gedruckt)
\item leeres weißes Blatt
\item Titelblatt (Fachbegriff: Haupttitel)
\item Zusammenfassung (deutsch) und Abstract (englisch), zusammen eine Seite, mit römischer Seitenzahl in Kleinbuchstaben
\item evtl. Vorbemerkung mit fortlaufender römischer Seitenzahl in Kleinbuchstaben
\item Inhaltsverzeichnis mit fortlaufender römischer Seitenzahl in Kleinbuchstaben
\end{itemize}

\begin{enumerate}
\item Einleitung als erstes Kapitel mit arabischen Seitenzahlen, beginnend bei 1. Darin führen Sie Ihr Publikum ins Thema ein und beschreiben folgende Punkte:
  \begin{enumerate}[label*=\arabic*.]
  \item Der Gegenstand Ihrer Arbeit, Abgrenzung von angrenzenden Themen.
  \item Die Relevanz Ihres Themas (evtl. auch Ihre persönliche Motivation).
  \item Das ungelöste Problem oder die ungeklärte Frage bzw. die Aufgabenstellung Ihrer Arbeit.
  \item Das Ziel Ihrer Untersuchung bzw. Bearbeitung.
  \item Eventuell die Forschungsfrage, die dazu dient, Ihr Ziel zu erreichen.
  \item Die Vorgehensweise bzw. Methode, die Sie dafür gewählt haben.
  \end{enumerate}

\item Darauf folgt ein Kapitel, in dem Sie die theoretischen Grundlagen Ihrer Arbeit nachvollziehen. Wenn Sie sich auf mehrere theoretische Grundlagen beziehen, sollten Sie jedem Diskurs ein eigenes Kapitel widmen. Es ist also meist hilfreich, wenn dieses Kapitel nicht die Überschrift trägt: \glqq 2. Theoretische Grundlagen\grqq{}. Statt dessen benennen Sie direkt das entsprechende Thema, z.B.: \glqq 2. Persönlichkeitspsychologie\grqq{}, oder: \glqq 2. Branchenstrukturanalyse\grqq{}.

In einer Hausarbeit ist es möglich, dass Sie sich auf nur eine theoretische Grundlage beziehen. In einer Abschlussarbeit sind es in der Regel zwei oder drei theoretische Grundlagen, daraus ergeben sich die Kapitelnummern 2, 3 und 4. – Diese Entscheidung treffen Sie nach Absprache mit Ihren Prüfern.

Die Reihenfolge der Kapitel ergibt sich aus der Allgemeinheit oder Breite der Themen. Sie beginnen mit dem allgemeinsten, z.B.: 2. Aufmerksamkeit und die Wahrnehmung von Sinnesreizen; 3. Verarbeitung visueller Reize; 4. Prinzipien für die Gestaltung visueller Elemente. – Die Regel für diese Anordnung ist selbstverständlich nicht eindeutig und muss von Ihnen gedeutet werden. Damit stecken Sie mitten in der Konstruktion einer schlüssigen Argumentationskette, und das ist ein wesentlicher Teil Ihrer Prüfungsleistung.

\item Insbesondere bei Abschlussarbeiten wenden Sie die allgemeinen Erkenntnisse aus den theoretischen Grundlagen auf einen konkreten Fall z.B. aus der Wirtschaft an. Dann beschreiben Sie auch diesen Zusammenhang (Branche, Unternehmen oder Organisation, Produkt oder Dienstleistung...) in einem eigenen Kapitel, z.B.: \glqq 5. Automobilbranche\grqq{}, oder: \glqq 5. Der Automobilzulieferer Coroplast«. Im engeren Sinn handelt es sich dabei nicht um theoretische Grundlagen. Gleichzeitig handelt es sich auch nicht um die Anwendung einer empirischen oder naturwissenschaftlichen Methode. Deshalb befindet sich dieses Kapitel zwischen den theoretischen und analytischen Kapiteln wie ein Scharnier.

\item Anschließend führen Sie Ihre Analyse durch, z.B. mit einer empirischen oder ingenieurwissenschaftlichen Methode. Es könnte den Titel tragen: \glqq 6. Empirische Untersuchung\grqq{} und enthält folgende Abschnitte:
  \begin{itemize}
  \item Darstellung der gewählten Methoden(n), z.B. Experteninterview.
  \item Schilderung Ihrer konkreten Untersuchung, z.B. Fragebogen, Datensammlung, Auswertung.
  \item Beschreibung der Untersuchungsergebnisse.
  \item Beantwortung der Forschungsfrage (wichtig: wenn Sie Ihre eingangs formulierte Forschungsfrage nicht explizit beantworten, ist das ein gravierender Mangel).
  \end{itemize}

\item Ihr Text endet mit der Zusammenfassung der Ergebnisse. In diesem Kapitel
  \begin{itemize}
  \item stellen Sie die wesentlichen Erkenntnisse Ihrer Arbeit kompakt zusammen, diskutieren diese und verorten Sie evtl. im Diskurs (wichtig: Sie formulieren hier keine neuen Argumente),
  \item reflektieren Sie selbstkritisch Ihre Anwendung der wissenschaftlichen Methode (was hat gefehlt, was würden Sie beim nächsten Mal berücksichtigen...)
  \item und formulieren Sie evtl. einen Ausblick auf Konsequenzen für künftige Forschung oder weitere Perspektiven auf das Thema.
  \end{itemize}
\end{enumerate}

\begin{itemize}[label={--}]
\item Der Textteil Ihrer Arbeit ist damit abgeschlossen. Es folgen die Verzeichnisse in dieser Reihenfolge ohne Kapitelnummern:
  \begin{itemize}[label={\bullet}]
  \item evtl. Abkürzungsverzeichnis,
  \item evtl. Formelverzeichnis,
  \item evtl. Abbildungsverzeichnis,
  \item evtl. Tabellenverzeichnis,
  \item evtl. Promptverzeichnis,
  \item evtl. Quellenverzeichnis oder Verzeichnis unveröffentlichter Quellen (falls Sie zwischen Quellen und Literatur unterscheiden müssen, das besprechen Sie bitte mit Ihren Prüfern),
  \item Literaturverzeichnis und
  \item evtl. Verzeichnis der Anhänge.
  \end{itemize}

\item Daran schließen sich evtl. Anhänge an (für Dokumente, Tabelle, Interviews, Fragebögen etc.), die mit Großbuchstaben sortiert werden, z.B.:
  \begin{itemize}[leftmargin=*,label={}]
  \item Anhang A: Fragebogen
  \item Anhang B: Vollständige Transkriptionen
  \item Anhang C: Häufigkeitstabellen
  \item Anhang D: Standbilder
  \end{itemize}

\item Dann folgt Ihre Eigenständigkeitserklärung. Sie unterschreiben sie bitte eigenhändig mit Datum. Ganz wichtig: Wenn Ihre Eigenständigkeitserklärung fehlt, wird Ihre Arbeit zwangsläufig mit 5 bewertet. Eine Vorlage für den Wortlaut Ihrer Eigenständigkeitserklärung finden Sie in Anhang B.
\item Ihre Abschlussarbeit enthält zuletzt noch einen tabellarischen Lebenslauf (ohne Portraitfoto).
\item Der Rückdeckel (Karton) ist unbedruckt.
\end{itemize}

\section{Die Gliederung ist formal korrekt}
\label{sec:gliederung-formal-korrekt}

Die Gliederung erfasst den Textteil Ihrer Arbeit (Einleitung bis Schluss) und zeigt ihren logischen Aufbau übersichtlich und aussagekräftig.

Damit Ihre Gliederung formal korrekt ist, achten Sie auf folgende Punkte:
\begin{itemize}[label={--}]
\item Alle Überschriften von Kapiteln (1. Erste Hierarchie), Abschnitten (1.1. Zweite Hierarchie) und Unterabschnitten (1.1.3. Dritte Hierarchie) sind beim Text und beim Inhaltsverzeichnis identisch.
\item Die Überschriften sind aussagekräftig und spezifisch. Es gibt keine Wiederholungen von übergeordneten Überschriften.
\item Die Überschriften sind keine vollständigen Sätze und enden deshalb nicht mit einem Satzzeichen (Punkt, Fragezeichen, Ausrufezeichen).
\item Anstelle von finiten Verbformen (\glqq Hier stelle ich die Methode dar\grqq{}) verwenden Sie Nominalisierungen (\glqq Darstellung der Methode\grqq{}).
\item Sie verwenden höchstens drei Hierarchieebenen (1.1.3. Dritte Hierarchieebene). Es gibt keine weitere Unterteilung Ihres Textes, also auch keine Überschriften ohne vorangestellte Nummerierung.
\item Kein Abschnitt oder Unterabschnitt steht alleine: Wenn Sie z.B. einen Abschnitt 3.1. haben, benötigen Sie auch zwingend einen Abschnitt 3.2.
\item Unter jeder Überschrift steht ein Textabschnitt. Bei Kapiteln können dies z.B. einführende und methodisch erläuternde Aussagen sein.
\end{itemize}

\section{Die Gliederung ist folgerichtig formuliert und aussagekräftig}
\label{sec:gliederung-folgerichtig}

Sie erkennen, ob Ihre Gliederung folgerichtig und aussagekräftig ist, wenn allein durch die Lektüre des Inhaltsverzeichnisses eine recht klare Vorstellung davon entsteht, worum es in Ihrer Arbeit geht und wie sich Ihre Argumentation darstellt. Die Kernaussagen lassen sich zwar nicht daran ablesen, aber es wird deutlich, an welchen Stellen Sie welche Arten von Aussagen treffen.

\section{Die Gliederung verfügt über eine der Themenstellung angemessene Tiefe}
\label{sec:gliederung-tiefe}

Sie haben Ihre Gliederung ausgewogen eingeteilt, wenn sich ein Unterabschnitt (z.B. 3.2.4.) über ca. eine halbe bis zwei Seiten erstreckt. Falls Sie sehen, dass Ihr Text zu einem Abschnitt oder Unterabschnitt sehr kurz oder sehr lang ist, dann prüfen Sie, ob Sie diese Passage sinnvollerweise mit einer anderen zusammenführen bzw. in kleinere Einheiten aufteilen sollten.

Vermeiden Sie mehr als neun Kapitel in Ihrer Gliederung.

\section{Die Strukturierung der Argumentation folgt stringent einer wissenschaftlichen Methode}
\label{sec:strukturierung-methode}

Wissenschaft beruht auf der Anwendung spezifischer, etablierter Methoden. In Ihrem Studium weisen Sie durch Ihre Haus- und Abschlussarbeiten nach, dass Sie sie beherrschen.

Nach Absprache mit Ihren Prüfern entscheiden Sie sich in der Regel für eine dieser Methoden:
\begin{itemize}[label={--}]
\item Deduktive Methode: Schlussfolgerung vom allgemeinen Modell auf spezifische Aussagen im konkreten Einzelfall.
\item Induktive Methode: Ableitung allgemeiner Aussagen auf der Grundlage spezifischer (z.B. empirischer) Daten.
\item Kausale Methode: Untersuchung von Ursachen und Wirkungen.
\item Dialektische Methode: Aufstellen von These, Gegenthese und abschließender Synthese.
\item Vergleichende Methode: Vergleich nach Objekten und Kriterien als Grundlage für Folgerungen.
\end{itemize}