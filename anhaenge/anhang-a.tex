\chapter*{Anhang A: Praktische Hinweise zum Gendern}
\addcontentsline{toc}{section}{Anhang A: Praktische Hinweise zum Gendern}
\label{app:gendern}

Eine häufig verwendete Möglichkeit ist das substantivierte Partizip Präsens. Es zeigt im Plural kein Genus:
\begin{itemize}
\item die Studierenden,
\item die Lehrenden.
\end{itemize}

Bitte beachten Sie, dass das substantivierte Partizip Präsens im Singular ein Genus zeigt:
\begin{itemize}
\item der Studierende ist eindeutig maskulin,
\item die Lehrende ist eindeutig feminin.
\end{itemize}

Abstrakte Rollenbezeichnungen und Begriffe für Kollektive, Institutionen und Positionen sind eine weitere Möglichkeit für sexusindifferente Formulierungen:
\begin{itemize}
\item die Vertretung,
\item die Studiengangsleitung,
\item das Kollegium,
\item das Publikum,
\item die Lehrkräfte,
\item das Team,
\item das Präsidium.
\end{itemize}

Die Pronomina alle und niemand zeigen ebenfalls kein Genus:
\begin{itemize}
\item Alle können am Kurs teilnehmen.
\item Niemand muss sich wegen der Prüfung Sorgen machen.
\end{itemize}

Gender-Zeichen, meist ein Doppelpunkt oder Sternchen (Asterisk), stehen zwischen dem Wortstamm der männlichen Form und der weiblichen Endung:
\begin{itemize}
\item Bewerber:in
\item Teilnehmer:innen
\item Student:in
\item Expert:innen
\end{itemize}

Die Beidnennung ist eine weitere Möglichkeit, also die Nennung von maskulinen und femininen Bezeichnungen:
\begin{itemize}
\item Studentinnen und Studenten,
\item Dozentinnen und Dozenten,
\item Prüferin bzw. Prüfer,
\item Expertin bzw. Experte.
\end{itemize}