\chapter*{Anhang C: Ergänzende Hinweise für Abschlussarbeiten in den Studiengängen Wirtschaftspsychologie und Mediendesign}
\addcontentsline{toc}{section}{Anhang C: Ergänzende Hinweise für Abschlussarbeiten in den Studiengängen Wirtschaftspsychologie und Mediendesign}
\label{app:wirtschaftspsychologie-mediendesign}

In diesen beiden Studiengängen können die Abschlussarbeit unterschiedlich ausgerichtet werden. Möglich sind:
\begin{enumerate}
\item theoretisch-konzeptionelle Abschlussarbeiten,
\item empirische Arbeiten (in der Wirtschaftspsychologie verpflichtend): qualitative oder quantitative Untersuchungen,
\item konzeptionell-gestalterische Arbeiten (nur im Studiengang Mediendesign).
\end{enumerate}

Eine eindeutige Zuordnung zu einem dieser Schwerpunkte ist nicht immer möglich.

In jedem Fall ist eine umfassende und systematische Darstellung des Stands der Forschung anhand der aktuellen wissenschaftlichen Literatur (Diskurs) erforderlich. Auf diesen Status Quo bezieht sich Ihre weitere Bearbeitung, bzw. Sie wenden ihn dafür an.

Alle Arbeiten können in Kooperation mit Unternehmen geschrieben werden. Bitte stimmen Sie sich dazu mit Ihren Prüfern ab.

\section*{1. Theoretisch-konzeptionelle Arbeiten}

Wenn Sie eine theoretisch-konzeptionelle Arbeit vorlegen, analysieren Sie darin ein Thema, ohne dass Sie hierzu eigene Berechnungen durchführen.

Ihr Ausgangspunkt ist eine möglichst vollständige Erfassung der relevanten Literatur und die systematische Aufbereitung der gewonnenen Informationen. Wie Sie dieses Material strukturieren und aufbereiten (Tiefe, Differenzierung, Präzision, Prägnanz, Gewichtung), bestimmt wesentlich die Qualität Ihrer Arbeit.

Typischerweise wird von Ihnen erwartet, dass Sie die wesentliche Literatur durchdringen und sich mit ihr kritisch auseinandersetzen. Dazu zählen insbesondere Definitionen (die Sie evtl. selbst entwickeln), Klassifikationsschemata oder Typologien. Zudem wird erwartet, dass sie Begründungszusammenhänge strukturierte darstellen.

Ein zentrales Ziel der theoretisch-konzeptionellen Arbeit ist die Verbesserung, Vereinfachung und vor allem Systematisierung der Darstellung bereits bekannter, aber bisher nicht in einer einzigen Arbeit zusammengefasster Fakten oder Argumente.

Den Titel Ihrer theoretisch-konzeptionellen Arbeit können Sie beispielsweise nach diesem Schema formulieren:
\begin{itemize}
\item Entwicklung einer strategisch-konzeptionellen Empfehlung
\item Formulierung von Handlungsempfehlungen für bestimmte Fälle
\item Eingrenzung von potenziellen Entwicklungsszenarien
\item Konzeption von bestimmten Maßnahmenempfehlungen für besondere Zwecke (beispielsweise aus dem konkreten Arbeitsumfeld).
\end{itemize}

Ihre Strukturierungsleistung soll zu einem Ziel führen, das bereits im Titel anklingt, z.B.: \glqq Strategische Markt- und Wettbewerbsanalyse der Videospielbranche - unter besonderer Berücksichtigung der Neuprodukteinführung eines Konsolenherstellers\grqq{}.

\section*{2. Empirische Arbeiten}

Bei Ihrer empirischen Arbeit kann es sich entweder um eine qualitative oder quantitative Untersuchung handeln.

Qualitative Untersuchungen werden durchgeführt, wenn ein wesentliches Verständnis für eine bestimmte Forschungsfrage erlangt werden soll. Ziel der qualitativen Forschung ist das Erkennen, Beschreiben und Verstehen von Zusammenhängen. Im Vordergrund stehen die vollständige Erfassung und Interpretation aller problemrelevanten Aspekte hinsichtlich des Themas der Abschlussarbeit. Hierbei bedient man sich offener, nicht standardisierter Erhebungsverfahren (z.B. Interviews, Expertengespräche, Gruppendiskussionen, qualitative Beobachtung, qualitative Experimente), deskriptiver Aufbereitungsverfahren (z.B. Gesprächsprotokoll, Transkription) und interpretativer Auswertungsverfahren (z.B. qualitative Inhaltsanalyse, Cognitive Mapping, objektive Hermeneutik). In der Regel kommen im Rahmen der qualitativen Untersuchung Methoden zum Einsatz, die sich auf kleine Fallzahlen beschränken, keine statistischen Analysen (z.B. Signifikanztests) implizieren, relative weiche Daten produzieren und ihre Erkenntnisse auf einem verhältnismäßig niedrigen Abstraktionsniveau mittels subjektiver Interpretation gewinnen. Die gewonnenen Ergebnisse sind zwar nicht repräsentativ, sie dienen aber zur ausführlichen Stoffsammlung, um ihrerseits wieder Hypothesen quantifizieren zu können.

Quantitative Untersuchungen haben die Messung bestimmter Sachverhalte bzw. die Entdeckung von Gesetzmäßigkeiten zum Gegenstand. Es handelt sich hierbei um einen Ansatz, der theoriegeleitet ist und der sich standardisierter Erhebungsmethoden (i.d.R. schriftliche Fragebögen) bedient. Ziel dieser Standardisierung ist es, die Antworten einer Vielzahl von Befragten unmittelbar vergleichen zu können. Der Vorteil quantitativer Untersuchungen liegt darin, dass sich die Messergebnisse mit statistischen Methoden (z.B. Korrelations- und Regressionsanalysen, Varianzanalysen, Faktoren- und Clusteranalysen, Metaanalysen, Diskriminanzanalysen, MDS) unter Nutzung statistischer Kennzahlen (Mittelwert, Median, Standardabweichung, Varianz etc.) verdichten und weiterverarbeiten lassen. Die Auswertung und Analyse der Daten, die im Rahmen der quantitativen Forschung generiert werden, erfolgt mit Hilfe von statistischen Programmen (z.B. SPSS, AMOS, PLS, R). Ziel ist es, Zusammenhänge zu erkennen und daraus allgemein gültige Aussagen abzuleiten. Rückschlüsse auf die tatsächlichen Verhältnisse in der Grundgesamtheit sind möglich.

In der Wirtschaftspsychologie (B.Sc.) werden ausschließlich empirische Abschlussarbeiten durchgeführt, wobei quantitativen Ansätzen der Vorzug zu geben ist.

\section*{3. Konzeptionell-gestalterische Arbeiten}

Diese Arbeit können Sie nur im Studiengang Mediendesign vorlegen. Hierbei handelt es sich um eine innovative Lösung für eine komplexe gestalterische Aufgabe.

Ihr Ausgangspunkt ist auch hier die Durchdringung relevanter wissenschaftlicher Grundlagen (insbesondere Theorien und Beispiele aus der Designgeschichte).

Daran schließen sich evtl. strategische sowie Ihre konzeptionellen und methodischen Darstellungen an, in denen Sie den Kontext der Aufgabe und Ihren Designprozess nachvollziehbar machen.

Sowohl die wissenschaftlichen als auch die methodischen und konzeptionellen Aussagen führen zu Ihrer praktischen Umsetzung im Sinn einer Lösungsentwicklung mit gestalterischen und technischen Mitteln.