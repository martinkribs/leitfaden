\chapter*{Anhang D: Gestalterische Hinweise für die Studiengänge Mediendesign (B.A.) und User Experience Design (M.A.)}
\addcontentsline{toc}{chapter}{Anhang D: Gestalterische Hinweise für die Studiengänge Mediendesign (B.A.) und User Experience Design (M.A.)}
\appendixentry{Gestalterische Hinweise (Mediendesign und User Experience Design)}
\label{app:gestalterische-hinweise}

\section*{Schmutztitel und Haupttitel}

Auf dem Schmutztitel und dem Haupttitel stehen die gleichen Inhalte wie bei allen anderen Studiengängen (vgl. Abschnitt 6.6 in diesem Leitfaden). Ansonsten sind Sie in Ihrer Gestaltung der Titel frei. Es ist auch ausdrücklich erwünscht, dass Sie von dieser Freiheit Gebrauch machen. Falls Sie sich dafür entscheiden, Ihre Arbeit nicht zu gestalten, verwenden Sie einen braunen (Mediendesign) bzw. weißen (User Experience Design) Einband. Bitte beachten Sie, dass Sie keinen schlichten andersfarbigen Einband verwenden, weil diese bereits anderen Studiengängen vorbehalten sind.

\section*{Schriften}

Beachten Sie bei der Wahl Ihrer Schrift die Angemessenheit. Neben der akademischen Ausstrahlung ist es wichtig, dass Sie sich später als Bewerber mit Ihrer Arbeit gegen Konkurrenten durchsetzten wollen. Alle Schriften für sogenannte Mengentexte sind aufgrund ihrer Lesefreundlichkeit geeignet, beispielsweise die Thesis, Info Text oder Corporate oder Schriften nach dem dynamischen Formprinzip, wie z.B. Garamond, Minion, Weidemann, Lexicon, Optima, Officina, Frutiger, Today oder Meta.

Beachten Sie bitte auch, dass die Prüfer Ihre Arbeit innerhalb kurzer Zeit lesen werden. Die Lektüre darf auch Freude bereiten, deshalb ist die Lesbarkeit bei der Auswahl ein wichtiges Entscheidungskriterium. Eine schmal laufende Schrift (condensed) ist also nicht geeignet. Serifenschriften bieten in der Regel Vorteile, sind aber kein Muss.

\section*{Raster und Layout}

Die Bachelor-Thesis soll angemessene Seitenränder haben und nur eine Textspalte, sowohl einen toten als auch einen lebenden Kolumnentitel integrieren sowie zwingend über eine Paginierung verfügen. Abbildungen sollt angemessen groß sein. Komplexe Schaubilder oder die Herleitung Ihrer Entwürfe dürfen, sofern notwendig, seitenfüllend sein. Für Infografiken entwickeln Sie idealerweise einen durchgängigen einheitlichen Stil.

\section*{Feintypografie}

Beachten Sie bitte alle feintypografischen Regeln. Halten Sie bitte sauber definierte Abstände vor und nach Überschriften oder Absätzen usw. ein. Ein Ausgleich des Flatter- oder Blocksatzes nach Fertigstellen des Textes ist Pflicht. Tipp: Beginnen Sie damit von hinten.