% Dokumentklasse
\documentclass[11pt,titlepage]{report} % Für längere Arbeiten mit Kapiteln und Abschnitten, geeignet für Berichte und Dokumentationen.

% Sprachunterstützung
\usepackage[ngerman]{babel} % Sorgt für sprachspezifische Regeln wie Silbentrennung und deutsche Bezeichnungen.

% Symbole und Icons
\usepackage{fontawesome} % Ermöglicht die Nutzung von Font Awesome Icons.

% Seitenlayout
\usepackage[a4paper,lmargin={3.5cm},rmargin={2.5cm},tmargin={3cm},bmargin={2cm},headsep=0.5cm,headheight=1.5cm]{geometry} % Einstellung von Seitenrändern für A4-Papier.

% Mathematische Pakete
\usepackage{amssymb} % Enthält zusätzliche mathematische Symbole.
\usepackage{amsthm} % Unterstützung für Theoreme, Definitionen und Beweise.
\usepackage{amsmath} % Verbesserte Darstellung von mathematischen Formeln.

% Grafiken und Bilder
\usepackage{graphicx} % Ermöglicht das Einfügen von Bildern und Grafiken.
\usepackage{svg} % Ermöglicht das Einfügen von SVG-Grafiken.

% Typografie
\usepackage{microtype} % Verbessert die Lesbarkeit und Silbentrennung.
\usepackage{fontspec} % Für die Verwendung von TrueType- und OpenType-Schriften.

% Hyperlinks
\usepackage[hidelinks]{hyperref}% Fügt klickbare Hyperlinks ein, ohne sie farblich hervorzuheben.

% Abkürzungen
\usepackage[footnote,printonlyused]{acronym} % Verwaltung von Abkürzungen mit Fußnotenoption.

% Kopf- und Fußzeilen
\usepackage{fancyhdr} % Ermöglicht die Gestaltung von Kopf- und Fußzeilen.

% Farben
\usepackage{xcolor} % Definiert und verwendet Farben im Dokument.

% Weitere mathematische Symbole
\usepackage{amsfonts} % Zusätzliche Schriftarten für mathematische Symbole.

% Tabellen
\usepackage{booktabs} % Verbesserte Gestaltung von Tabellen.
\usepackage{multirow} % Ermöglicht Zellen über mehrere Zeilen in Tabellen.
\usepackage{enumitem} % Anpassung von Listen und Aufzählungen.

% Literaturverwaltung
\usepackage[style=chicago-notes, backend=biber, sorting=nyt, dashed=false, giveninits=true]{biblatex} % Für Literaturverzeichnisse nach Chicago-Stil mit Fußnoten.

% Zitate
\usepackage{csquotes} % Korrekte Darstellung von Zitaten, z. B. Anführungszeichen.

% Fußnoten-Formatierung
\usepackage[hang,flushmargin,bottom,multiple]{footmisc} % Keine Einrückung, mehr Abstand zwischen Fußnoten
\setlength{\footnotesep}{12pt} % Abstand zwischen Fußnoten

% Fußnotenzahlen mit Punkt danach
\makeatletter
\renewcommand\@makefnmark{\mbox{\normalfont\@thefnmark.}}
\makeatother

% Positionierung
\usepackage{float} % Präzise Platzierung von Tabellen und Abbildungen.

% Zeilenabstand
\usepackage[onehalfspacing]{setspace} % Erzeugt 1,5-fachen Zeilenabstand.

% Glossare
\usepackage[toc=false]{glossaries-extra} % Verwaltung von Glossaren ohne Inhaltsverzeichnis-Eintrag.

% Tabellenfarben
\usepackage{colortbl} % Ermöglicht das Einfärben von Tabellenzellen.

% Tabellen- und Spaltenformatierung
\usepackage{array} % Zusätzliche Optionen für Tabellen und Spalten.

% Diagramme und Visualisierungen
\usepackage{tikz} % Erstellung von Vektorgraphiken und Diagrammen.
\usepackage{pgfplots} % Erstellung von Plots und Grafiken.

\usetikzlibrary{shapes,arrows,positioning,fit,backgrounds,calc,matrix,decorations.pathreplacing,decorations.markings,shapes.geometric,mindmap}

% PDF-Integration
\usepackage{pdfpages} % Einfügen von ganzen PDF-Dokumenten.

% Querformat
\usepackage{pdflscape} % Ermöglicht die Nutzung von Seiten im Querformat.

% Nummerierungsanpassung
\usepackage{chngcntr} % Ermöglicht Änderungen der Nummerierung von Kapiteln und Abschnitten.

% Inhaltsverzeichnis
\usepackage{tocloft} % Anpassung des Layouts des Inhaltsverzeichnisses.

% Titelgestaltung
\usepackage{titlesec} % Anpassung von Kapitel- und Abschnittstiteln.

% Ausrichtung
\usepackage{ragged2e} % Ermöglicht den Blocksatz oder Flattersatz.

\pgfplotsset{compat=1.18}

% Pfade für Bilder und andere Ressourcen
\graphicspath{{bilder/}}

% Arial als Hauptschrift (11pt)
\setmainfont{Arial}

% Bibliographie
\addbibresource{quellen.bib}

% Farben
\definecolor{link}{HTML}{0000EE}

% Absatzformatierung
\setlength{\parindent}{0pt}           % Keine Einrückung am Absatzbeginn
\setlength{\parskip}{6pt}             % 6pt Abstand nach jedem Absatz

% Anpassung \listoffigures
\renewcommand{\cftloftitlefont}{\Large\bfseries}
\renewcommand{\cftbeforeloftitleskip}{10pt}
\renewcommand{\cftafterloftitleskip}{10pt}

% Anpassung \listoftables
\renewcommand{\cftlottitlefont}{\Large\bfseries}
\renewcommand{\cftbeforelottitleskip}{10pt}
\renewcommand{\cftafterlottitleskip}{10pt}

% Anpassung \tableofcontents
\renewcommand{\cfttoctitlefont}{\Large\bfseries}
\renewcommand{\cftbeforetoctitleskip}{10pt}
\renewcommand{\cftaftertoctitleskip}{10pt}

% Füge Punkte für Kapitel im Inhaltsverzeichnis hinzu
\renewcommand{\cftchapaftersnum}{.}
\renewcommand{\cftchapdotsep}{\cftdotsep}
\renewcommand{\cftsecaftersnum}{.}

% Abstand nach Nummern im Abbildungs- und Tabellenverzeichnis
\setlength{\cftfignumwidth}{3em}
\setlength{\cfttabnumwidth}{3em}

% Überschriftenformatierung
\titleformat{\chapter}
  {\fontsize{11}{16.5}\bfseries\raggedright}
  {\thechapter.}{1em}{}
\titlespacing*{\chapter}{0pt}{0pt}{6pt}

\titleformat{\section}
  {\fontsize{11}{16.5}\bfseries\raggedright}
  {\thesection.}{1em}{}
\titlespacing*{\section}{0pt}{0pt}{6pt}

\titleformat{\subsection}
  {\fontsize{11}{16.5}\bfseries\raggedright}
  {\thesubsection.}{1em}{}
\titlespacing*{\subsection}{0pt}{0pt}{6pt}

\titleformat{\subsubsection}
  {\fontsize{11}{16.5}\bfseries\raggedright}
  {\thesubsubsection.}{1em}{}
\titlespacing*{\subsubsection}{0pt}{0pt}{6pt}

% Blocksatz für Fließtext
\justifying

% Bibliographie-Einstellungen für Chicago-Stil
\DeclareNameAlias{sortname}{family-given}
\DeclareDelimFormat{nameyeardelim}{\addcomma\space}
\DeclareDelimFormat{multinamedelim}{\space\addcomma\space}
\DeclareDelimFormat{finalnamedelim}{\space\&\space}
\DefineBibliographyStrings{german}{andothers = {{et\,al\adddot}},}
\renewcommand{\mkbibnamefamily}[1]{\textsc{#1}}

% Benutzerdefinierte Zitatbefehle
\newcommand{\customcite}[2][]{%
  \ifblank{#1}{%
    \footcite{#2}%
  }{%
    \footcite[#1]{#2}%
  }%
}

% Fix lange URLs
\newcommand*\oldurlbreaks{}
\let\oldurlbreaks=\UrlBreaks
\renewcommand{\UrlBreaks}{\oldurlbreaks
  \do\a\do\b\do\c\do\d\do\e\do\f\do\g\do\h\do\i\do\j\do\k\do\l\do\m\do\n\do\o\do\p\do\q\do\r\do\s\do\t\do\u\do\v
  %\do\w%
  \do\x\do\y\do\z
  \do?\do&\do_
  \do\0\do\1\do\2\do\3\do\4\do\5\do\6\do\7\do\8\do\9
  \do:\do/
  }
\urlstyle{same}

% Befehle für Links und Verweise
\newcommand{\sct}[2]{\textcolor{link}{\hyperref[#1]{#2}}}
\newcommand{\gl}[1]{\gls{#1}}

% Promptverzeichnis erstellen
\newcommand{\listpromptname}{Promptverzeichnis}
\newlistof{prompt}{prm}{\listpromptname}
% Formatierung des Promptverzeichnis-Titels (wie bei anderen Verzeichnissen)
\renewcommand{\cftprmtitlefont}{\Large\bfseries}
\renewcommand{\cftbeforeprmtitleskip}{10pt}
\renewcommand{\cftafterprmtitleskip}{10pt}

% Befehl für Prompt-Referenzen
\newcounter{promptcounter}
\newcommand{\promptentry}[1]{%
  \refstepcounter{promptcounter}%
  \addcontentsline{prm}{prompt}{\protect\numberline{\thepromptcounter}#1}%
  #1%
}

% Inhaltsverzeichnis-Einrückung
\setlength{\cftsubsecindent}{1.5em}
\setlength{\cftsubsubsecindent}{3.0em}

% Anhangsverzeichnis erstellen
\newcommand{\listappendixname}{Verzeichnis der Anhänge}
\newlistof{appendix}{app}{\listappendixname}

% Formatierung des Anhangsverzeichnis-Titels (wie bei anderen Verzeichnissen)
\renewcommand{\cftapptitlefont}{\Large\bfseries}
\renewcommand{\cftbeforeapptitleskip}{10pt}
\renewcommand{\cftafterapptitleskip}{10pt}

\newcommand{\appendixentry}[1]{%
  \refstepcounter{appendix}%
  \addcontentsline{app}{appendix}{\protect\numberline{\theappendix}#1}%
}

% Glossar/Abkürzungsverzeichnis
\makenoidxglossaries
\newglossaryentry{ba}{
  name={B.A.},
  description={Bachelor of Arts}
}
\newglossaryentry{beng}{
  name={B.Eng.},
  description={Bachelor of Engineering}
}
\newglossaryentry{bsc}{
  name={B.Sc.},
  description={Bachelor of Science}
}
\newglossaryentry{cmos}{
  name={CMOS},
  description={The Chicago Manual of Style}
}
\newglossaryentry{llb}{
  name={LL.B.},
  description={Bachelor of Laws}
}
\newglossaryentry{llm}{
  name={LL.M.},
  description={Master of Laws}
}
\newglossaryentry{ma}{
  name={M.A.},
  description={Master of Arts}
}
\newglossaryentry{mba}{
  name={MBA},
  description={Master of Business Administration}
}

\glsaddall
% Anpassung der Kopfzeilen
% Behält die Nummer bei nummerierten Kapiteln, entfernt nur das Wort "Kapitel"
\renewcommand{\chaptermark}[1]{\markboth{\thechapter. #1}{}}

% Seitenstil für alle Seiten, einschließlich der ersten Seite eines Kapitels
\fancypagestyle{plain}{%
  \fancyhf{} % Alle Kopf- und Fußzeilenfelder löschen
  \lhead{\nouppercase{\leftmark}} % Kapitelüberschrift oben links
  \rhead{\thepage} % Seitenzahl oben rechts
}

% Seitenstil für normale Seiten
\pagestyle{fancy}
\fancyhf{} % Alle Kopf- und Fußzeilenfelder löschen
\lhead{\nouppercase{\leftmark}} % Kapitelüberschrift oben links
\rhead{\fontsize{9}{13.5}\selectfont\thepage} % Seitenzahl oben rechts, 9pt
\lhead{\fancyplain{}{\nouppercase{\leftmark}}} % Kapitelüberschrift oben links

% Anpassung der Verzeichnis-Überschriften
\renewcommand{\contentsname}{Inhalt}
\renewcommand{\listfigurename}{Abbildungsverzeichnis}
\renewcommand{\listtablename}{Tabellenverzeichnis}
\renewcommand{\bibname}{Quellen- und Literaturverzeichnis}

% Setze PDF-Metadaten
\hypersetup{
  pdftitle={Titel},
  pdfauthor={Mustermensch},
  pdfsubject={Subjekt},
}

\begin{document}

% Schmutztitel (Vorderdeckel)
%\begin{titlepage}
    \thispagestyle{empty}
    \hspace{-1cm}\includegraphics[width=0.55\textwidth]{rh-logo}\\[1.9cm]
    
    \vspace{2cm}
    \begin{minipage}{\textwidth}
    \hspace{2.5cm}\begin{minipage}{\dimexpr\textwidth-2.5cm\relax}
    {\Large\textbf{Fachbereich Ingenieurwesen}}
    
    {\Large Studiengang Informatik (B.Sc.)}\\
    \\
    \\
    
    
    {\Large\textbf{Subjekt}}
    
    \parbox{\textwidth}{\Large\raggedright Titel}\\
    \\
    
    {\normalsize\fontsize{11pt}{16.5pt}\selectfont vorgelegt von}
    
    {\Large Mustermensch}
    
    {\normalsize\fontsize{11pt}{16.5pt}\selectfont Matrikelnr. 111111111}\\
    \\
    
    {\normalsize Sommersemester 2025}
    \end{minipage}
    \end{minipage}
    
    \vfill
    
\end{titlepage}


% Leeres weißes Blatt
%\newpage
%\thispagestyle{empty}
%\quad 
%\newpage

% Haupttitel
%\input{zusatz/haupttitel}

% Leitfaden Titelseite (ohne Kopfzeile)
\thispagestyle{empty}
\begin{titlepage}
    \thispagestyle{empty}
    \hspace{-1cm}\includegraphics[width=0.55\textwidth]{rh-logo}\\[1.9cm]
    
    \vspace{2cm}
    \begin{minipage}{\textwidth}
    \hspace{2.5cm}\begin{minipage}{\dimexpr\textwidth-2.5cm\relax}
    {\Large\textbf{Leitfaden zum Anfertigen \\ wissenschaftlicher Arbeiten}}\\

    
    {\normalsize\fontsize{11pt}{16.5pt}\selectfont Version vom 31. März 2025}\\
    

    \end{minipage}
    \end{minipage}
    
    \vfill
    
\end{titlepage}

% Deaktiviere Einrückungen
\setlength\parindent{0pt}

\begin{onehalfspace}
    % Römische Seitennummerierung für Verzeichnisse
    \pagenumbering{roman}
    
    % Zusammenfassung (deutsch) und Abstract (englisch)
    %\chapter*{Zusammenfassung}
\addcontentsline{toc}{section}{Zusammenfassung}
\label{sec:zusammenfassung}

Der vorliegende Leitfaden zum Anfertigen wissenschaftlicher Arbeiten bietet Studierenden der Rheinischen Hochschule Köln eine umfassende Orientierung für die Erstellung von Haus- und Abschlussarbeiten. Er folgt dem Bewertungsschema der Hochschule und erläutert detailliert die sechs Leistungsabschnitte: Thema und Forschungsfrage, Strukturierung, konzeptionell-theoretische Grundlagen, Durchführung, Quellen und Zitierweise sowie Form und Stil. Der Leitfaden gibt konkrete Hinweise zur formalen Gestaltung, zur inhaltlichen Strukturierung und zur wissenschaftlichen Arbeitsweise. Ergänzende Anhänge bieten spezifische Informationen für bestimmte Studiengänge sowie praktische Hilfestellungen zu Themen wie gendergerechte Sprache. Das Dokument dient als verbindliche Grundlage für die Erstellung wissenschaftlicher Arbeiten an der Rheinischen Hochschule und ergänzt die Prüfungsordnungen.

\vspace{0.5cm}

\begin{center}
\textbf{Abstract}
\end{center}

This guide for preparing academic papers provides students at Rheinische Hochschule Köln with comprehensive orientation for creating term papers and theses. It follows the university's assessment scheme and explains in detail the six performance sections: topic and research question, structuring, conceptual-theoretical foundations, implementation, sources and citation style, as well as form and style. The guide provides specific instructions on formal design, content structuring, and scientific methodology. Supplementary appendices offer specific information for certain degree programs as well as practical assistance on topics such as gender-inclusive language. The document serves as a binding basis for the preparation of academic papers at Rheinische Hochschule and complements the examination regulations.
    %\newpage
    
    % Inhaltsverzeichnis
    \phantomsection
    \tableofcontents
    \newpage

    % Optionale Vorbemerkung
    % Auskommentieren, wenn nicht benötigt
    \chapter*{Vorbemerkung zur Schreibweise in diesem Leitfaden}
\addcontentsline{toc}{chapter}{Vorbemerkung zur Schreibweise in diesem Leitfaden}
\label{chap:vorbemerkung}

Die Debatte um das Gendern in der deutschen Sprache wird momentan engagiert geführt. Dabei überlagern sich politische, moralische und sprachwissenschaftliche Ebenen.\footnote{Der gesamte Text wurde mit folgendem Prompt bearbeitet: \promptentry{\glqq Versetze dich in die Rolle eines Lektors. Kontrolliere den folgenden Text in Bezug auf die aktuell verbindlichen Regeln der deutschen Orthographie, Grammatik und Interpunktion. Achte auch auf typographisch korrekte Anführungszeichen sowie darauf, dass Gedankenstriche anstelle von Bindestrichen (Divis) verwendet werden. Kontrolliere außerdem die Ausdrucksweise: Sie soll sachlich, nüchtern, klar, präzise und eindeutig sein. Erhalte den fachsprachlichen Duktus. Markiere alle Textstellen, die dir als Lektor unter diesen Gesichtspunkten auffallen.\grqq{}, ChatGPT-3.5, Open AI, 8.11.2024, stilistisch und inhaltlich korrigiert.}}

Unter linguistischer Perspektive geht es beim Gendern im Kern um die Frage, wie Funktionsbezeichnungen vor allem im Plural zum Ausdruck gebracht werden sollen. Die Befürworter des Genderns gehen davon aus, dass die generische Pluralform immer nur spezifisch als Ausdruck für Männer verstanden wird. Die Kritiker dieser Position gehen hingegen davon aus, dass mit der generischen Pluralform eine Aussage getroffen wird, die unabhängig ist von den konkreten Personen, welche eine Funktion ausüben.

In den beiden vorigen Sätzen wird der generische Plural \glqq Befürworter\grqq{} und \glqq Kritiker\grqq{} verwendet. Die ausschlaggebende Grundlage für diese Entscheidung, die für den gesamten Leitfaden gilt, ist der gegenwärtige Sachstand, dass das Gendern mit einem Binnen-I (\glqq VerfasserInnen\grqq{}) oder mit Sonderzeichen (z.B. \glqq Befürworter*innen\grqq{} oder \glqq Kritiker:innen\grqq{}) kein Bestandteil der deutschen Rechtschreibregelung ist.

Gleichzeitig lautet die Empfehlung dieses Leitfadens, dass Sie als diejenigen, die eine wissenschaftliche Arbeit verfassen, selbst darüber entscheiden sollen, ob Sie Ihren Text gendern oder nicht. Auf die Benotung soll Ihre Entscheidung keinen Einfluss haben.

Wofür Sie sich auch entscheiden: Es in jedem Fall erforderlich, dass Sie einheitlich schreiben, denn Uneinheitlichkeit führt zu Punktabzug.

Falls Sie möchten, können Sie Ihrer Arbeit einen spezifischen Hinweis voranstellen, zum Beispiel in einer Vorbemerkung oder in einer Fußnote, weil Sie damit möglichen Fehldeutungen vorbeugen können. Zwingend notwendig ist ein solcher Hinweis nicht.\footnote{In dieser Arbeit wird das generische Maskulinum verwendet, wenn im Mittelpunkt der Aussagen keine spezifischen Personen stehen, sondern die Funktion, die Personen ausüben. Sämtliche biologischen Geschlechter und subjektiv empfundenen Zuordnungen sind damit gleichermaßen gemeint.}

Falls Sie in Ihrem Text gendern möchten, finden Sie dazu Hinweise in Anhang A.
    \newpage

    % Hauptteil mit arabischer Seitennummerierung
    \pagenumbering{arabic}

    % Optionale Einleitung und Struktur
    % Auskommentieren, wenn nicht benötigt
    \chapter*{Einleitung}
\addcontentsline{toc}{chapter}{Einleitung}
\label{chap:einleitung}

Die scientific community – also die Gemeinschaft der Menschen, die forschen und ihre Erkenntnisse mit anderen teilen wollen – steht allen offen. Es gibt keine formale, exkludierende Zugangsbarriere.

Spätestens seit Ihrer Immatrikulation an der Rheinischen Hochschule sind Sie Teil dieser scientifc community. Herzlich willkommen!

Ihr Studium bereitet Sie nicht erst aufs Forschen vor. Es steht Ihnen frei, vom ersten Tag an Ihre Zeit an der Rheinischen Hochschule forschend zu nutzen. Das gilt nicht nur für Seminare oder Projekte. Sie können auch schon Ihre Haus- und Abschlussarbeiten als Forschung konzipieren. Es spielt dafür keine Rolle, ob Sie Ihr Forschungsinteresse auf ein äußerst kleines Feld eingrenzen. Jeder Erkenntnisgewinn ist ein Dienst an der Wissenschaft, mag er Ihnen auch gering erscheinen.

Mit der Lektüre dieses Leitfaden ersetzen Sie selbstverständlich nicht Ihre aktive Teilnahme am Modul Wissenschaftliches Arbeiten. Es handelt sich hier lediglich um einige grundlegende Hinweise von übergeordnetem, die gesamte Hochschule betreffenden Charakter. Damit sollen insbesondere solche Aspekte des wissenschaftlichen Arbeitens geklärt werden, die sich verallgemeinern lassen.

Am Ende ist die Absprache mit Ihren Prüfern verbindlich, weil die individuelle Durchführung von Forschung und Lehre durch Artikel 5 Absatz 3 des Grundgesetzes geschützt ist.

Wissenschaftliches Arbeiten ist keine Tätigkeit, der Sie sich zuwenden, wenn Sie alles andere im Studium erledigt haben. Im Kern ist das Schreiben eines wissenschaftlichen Textes die Forschung selbst. Denn Forschung zielt auf Erkenntnisgewinn, den Sie nicht für sich behalten, sondern durch Veröffentlichung mit Ihrer Fachöffentlichkeit teilen.

Dafür müssen Sie Ihre Erkenntnisse in eine schriftliche Form bringen. Und um Ihr Publikum zu überzeugen, müssen Sie Ihre Aussagen intersubjektiv nachvollziehbar formulieren und sich auf vertrauenswürdige Quellen stützen. Beide Aspekte, die schriftliche Form und Ihre Nachweisführung, werden in diesem vorliegenden Leitfaden sowie in der separaten Praktische[n] Anleitung: Zitieren in wissenschaftlichen Arbeiten der Rheinischen Hochschule behandelt.

Was für Wissenschaft grundsätzlich gilt, das trifft auch auf diesen Leitfaden zu: Wir sind alle nur Menschen, die sich irren können und denen versehentliche Fehler unterlaufen. Falls Ihnen also etwas auffällt, das Ihnen fragwürdig oder offensichtlich falsch erscheint und deshalb geändert werden sollte, freuen wir uns jederzeit über Ihren Hinweis an unser Qualitätsmanagement: qm@rh-koeln.de. Vielen Dank für Ihre Hilfe!

\chapter*{Struktur und Gültigkeit dieses Leitfadens}
\addcontentsline{toc}{chapter}{Struktur und Gültigkeit dieses Leitfadens}
\label{chap:leitfaden-struktur}

Mit jeder Hausarbeit und Abschlussarbeit führen Sie vor, dass Sie das Wissen, welches Sie sich im Studium angeeignet haben, anwenden und praxisrelevante Fragen sachgerecht in einer wissenschaftlichen Methode beantworten können. Inwieweit Ihnen dies gelungen ist, unterliegt der Bewertung durch Ihre Prüfer.

Für die Bewertung wissenschaftlicher Arbeiten verwendet die Rheinische Hochschule ein einheitliches Schema. Es enthält sechs Abschnitte mit detaillierten Hinweisen und bestimmt, wieviele Punkte Sie für welche Leistungen erhalten.

\begin{table}[h]
\label{tab:leistungsabschnitte}
\begin{tabular}{|l|l|c|}
\hline
\textbf{Nr.} & \textbf{Leistungsabschnitt} & \textbf{Punkte} \\
\hline
1. & Thema und Forschungsfrage bzw. Aufgabenstellung, Hypothesen & 10 \\
\hline
2. & Strukturierung & 10 \\
\hline
3. & Konzeptionell-theoretische Grundlagen & 20 \\
\hline
4. & Durchführung & 40 \\
\hline
5. & Quellen und Zitierweise & 10 \\
\hline
6. & Form und Stil & 10 \\
\hline
\end{tabular}
\caption{Übersicht der Leistungsabschnitte für die Bewertung von Haus- und Abschlussarbeiten an der Rheinischen Hochschule Köln.}
\end{table}

Damit Sie genau nachvollziehen können, welche Leistungen von Ihnen erwartet werden (Kategorien) und wie diese bewertet werden (Kriterien), folgt dieser Leitfaden dem Bewertungsschema.

In Anhang C finden Sie ergänzende Hinweise für die Abschlussarbeiten in den Studiengängen Mediendesign und Wirtschaftspsychologie.

Der vorliegende Leitfaden ergänzt die Prüfungsordnungen (RSPO und SPO), die im Zweifelsfall verbindlich sind.
    \newpage

    % Kapitel einbinden
   \chapter{Thema und Forschungsfrage bzw. Aufgabenstellung, Hypothesen}
\label{chap:thema-forschungsfrage}

In Ihrer Arbeit widmen Sie sich einem Thema. Sie formulieren es als Titel mit höchstens 200 Zeichen, am Ende steht kein Satzzeichen (insbesondere kein Fragezeichen). Ihre Formulierung soll aussagefähig, selbsterklärend und nicht zu allgemein gehalten sein. Firmen- und Produktnamen vermeiden Sie. – Bitte denken Sie daran, dass der Titel nicht identisch ist mit der Forschungsfrage, die Sie untersuchen. Falls Sie keine Forschungsfrage untersuchen, sondern eine Aufgabenstellung bearbeiten, ist es zulässig, dass ihr Wortlaut identisch ist mit dem Titel. Dazu stimmen Sie sich mit Ihren Prüfern ab.

\section{Das Thema ist aktuell und für Wissenschaft sowie Praxis relevant}
\label{sec:thema-aktuell-relevant}

Weil Wissenschaft danach strebt, neue Erkenntnisse zu gewinnen und neue Ergebnisse im Sinne der Aufgabenstellung hervozubringen, können Sie sich nicht mit einer Frage beschäftigen, die bereits beantwortet ist – es sei denn, dass neue Aspekte aufgetreten sind, die bisher noch nicht berücksichtigt wurden.

In Ihrer Einleitung begründen Sie Ihre Themenwahl. Seine Aktualität und Relevanz stehen dabei im Vordergrund.

\section{Das Thema ist anspruchsvoll und herausfordernd}
\label{sec:thema-anspruchsvoll}

Es gibt unzählige Fragen, die zwar nicht beantwortet, aber auch simpel oder sogar banal sind. Wissenschaftliche Zusammenhänge zeichnen sich dadurch aus, dass sie prinzipiell komplex sind und tendenziell grenzenlos erscheinen. Sie erfordern eine differenzierte und präzise Bearbeitung. Einfache Antworten sind dafür unangemessen. In Ihrer Arbeit widmen Sie sich solchen Aufgaben. Inwiefern Ihre Forschungsfrage bzw. Aufgabenstellung diesen Ansprüchen genügt, besprechen Sie mit Ihren Prüfern.

\section{Die Situation bzw. das Problem ist klar und eindeutig formuliert}
\label{sec:problem-klar-formuliert}

Sie formulieren das Problem, das Sie untersuchen, bzw. die Situation, die verändert werden soll, so eindeutig und konkret wie möglich. Wenn Sie den Eindruck haben, dass Ihre Aussage an dieser Stelle weitschweifig, abstrakt und kompliziert ist, dann beherrschen Sie Ihr Thema wahrscheinlich noch nicht. Sie erkennen, dass Sie Ihr Thema im Griff haben, wenn Sie sich klar und verständlich ausdrücken können.

\section{Themenrelevante Forschungslücken werden identifiziert}
\label{sec:forschungsluecken}

Ihre Arbeit beruht auf einer Darstellung des aktuellen Stands der Wissenschaft für Ihr Thema. Dabei referieren Sie nicht nur die gegenwärtig wesentlichen Positionen und Argumente. Sie identifzieren auch die Lücken in der Literatur, die für Ihr Thema relevant sind.

\section{Die Zielsetzung der Arbeit wird dargestellt}
\label{sec:zielsetzung}

Das allgemeine Ziel jeder wissenschaftlichen Arbeit ist Erkenntnisgewinn. Sie müssen deutlich machen, worin das Ziel Ihrer konkreten Arbeit besteht und inwiefern es dazu geeignet ist, den Stand der Erkenntnis über Ihr Thema zu erweitern.

\section{Die Forschungsfrage(n) bzw. Aufgabenstellung und Hypothesen sind eindeutig formuliert und begründet}
\label{sec:forschungsfrage-hypothesen}

Mit Ihrer Forschungsfrage bzw. mit der Vorgehensweise zur Bearbeitung Ihre Aufgabenstellung formulieren Sie Ihr konkretes Vorhaben. Grundsätzlich sind zum Thema Ihrer Arbeit (ausgedrückt im Titel) unendlich viele Forschungsfragen denkbar. Sie widmen sich aber nur einer einzigen.

Weil Ihre Arbeit auf dem aktuellen Stand der Wissenschaft beruhen (und bei Master-Thesen: eine Forschungslücke schließen) muss, können Sie Ihre Forschungsfrage erst formulieren, wenn Sie die relevante Literatur bzw. den aktuellen Stand der Technik kennen: Solange Sie nicht genau wissen, welche Erkenntnisse bereits veröffentlicht wurden, können Sie nicht wissen, welche Erkenntnisse noch fehlen. Es ist deshalb unwahrscheinlich, dass Sie Ihre Forschungsfrage direkt zu Beginn der Bearbeitung präzise formulieren können.

Das gilt auch für die Aufsstellung Ihrer Hypothesen oder Annahmen. Sie entwickeln sie auf der Grundlage der theoretischen Grundlagen in Bezug auf Ihr spezifisches Thema. Hypothesen oder Annahmen funktionieren wie ein Scharnier zwischen der allgemeinen Theorie und der konkreten Wirklichkeit nach dem Muster: Wenn allgemein angenommen wird, dass ..., dann ist in dieser konkreten Situation davon auszugehen, dass...

\section{Methoden und Konzepte sind unabhängig von einem Unternehmen bzw. einem spezifischen Problem}
\label{sec:methoden-unabhaengig}

An jede wissenschaftliche Arbeit richtet sich der Anspruch, dass ihre Erkenntnisse verallgemeinerbar sind. Das bedeutet nicht unbedingt, dass sie allgemeingültig im Sinne statistischer Repräsentativität sind: Dieser Anspruch richtet sich vor allem an Dissertationen und Habilitationen.

Verallgemeinerbar bedeutet: Sie führen Ihre Untersuchung methodisch so durch, dass Sie zu repräsentativen Ergebnissen gelangen könnten, wenn Ihnen dafür die Ressourcen (v.a. Zeit) zur Verfügung stünden. Sie führen z.B. eine Online-Umfrage mit einer Stichprobe von 100 Teilnehmern, deren soziodemographische Merkmale nicht repräsentativ für die deutsche Bevölkerung sind. Ihre Methode passt und ist auch korrekt angewandt, Ihnen fehlen \glqq nur\grqq{} die passenden Teilnehmer, damit Sie eine allgemeingültige Aussage treffen können.

Dieser Anspruch schließt auch jede Untersuchung aus, die methodisch nur auf eine einzige Situation zugeschnitten ist. Viele Unternehmen interessieren sich lediglich für Antworten, die ihre eigene Situation betreffen. Die Übertragbarkeit auf vergleichbare Situationen von anderen Unternehmen ist für sie nicht von Belang. In der Wissenschaft beschäftigen wir uns aber nur mit solchen Fragen, die sich auf vergleichbare Situationen übertragen lassen.
   \chapter{Strukturierung}
\label{chap:strukturierung}

Die Struktur erfüllt zwei Funktionen. Sie hilft Ihnen im während der Bearbeitung dabei, den Überblick über Ihre Argumentationskette zu behalten: Befinden sich alle Aussagen an der richtigen Stelle, damit Ihre Argumentationskette schlüssig und stringent ist? Und Ihrem Publikum hilft sie dabei, Ihrer Argumentation zu folgen: Ihr roter Faden dient beim ersten Blick ins Inhaltsverzeichnis und bei der fortlaufenden Lektüre als Orientierungshilfe.

Grundsätzlich ist Ihre Arbeit in dieser Reihenfolge strukturiert:
\begin{itemize}[label={--}]
\item Vorderes Deckblatt (Fachbegriff: Schmutztitel, auf Karton gedruckt)
\item leeres weißes Blatt
\item Titelblatt (Fachbegriff: Haupttitel)
\item Zusammenfassung (deutsch) und Abstract (englisch), zusammen eine Seite, mit römischer Seitenzahl in Kleinbuchstaben
\item evtl. Vorbemerkung mit fortlaufender römischer Seitenzahl in Kleinbuchstaben
\item Inhaltsverzeichnis mit fortlaufender römischer Seitenzahl in Kleinbuchstaben
\end{itemize}

\begin{enumerate}
\item Einleitung als erstes Kapitel mit arabischen Seitenzahlen, beginnend bei 1. Darin führen Sie Ihr Publikum ins Thema ein und beschreiben folgende Punkte:
  \begin{enumerate}[label*=\arabic*.]
  \item Der Gegenstand Ihrer Arbeit, Abgrenzung von angrenzenden Themen.
  \item Die Relevanz Ihres Themas (evtl. auch Ihre persönliche Motivation).
  \item Das ungelöste Problem oder die ungeklärte Frage bzw. die Aufgabenstellung Ihrer Arbeit.
  \item Das Ziel Ihrer Untersuchung bzw. Bearbeitung.
  \item Eventuell die Forschungsfrage, die dazu dient, Ihr Ziel zu erreichen.
  \item Die Vorgehensweise bzw. Methode, die Sie dafür gewählt haben.
  \end{enumerate}

\item Darauf folgt ein Kapitel, in dem Sie die theoretischen Grundlagen Ihrer Arbeit nachvollziehen. Wenn Sie sich auf mehrere theoretische Grundlagen beziehen, sollten Sie jedem Diskurs ein eigenes Kapitel widmen. Es ist also meist hilfreich, wenn dieses Kapitel nicht die Überschrift trägt: \glqq 2. Theoretische Grundlagen\grqq{}. Statt dessen benennen Sie direkt das entsprechende Thema, z.B.: \glqq 2. Persönlichkeitspsychologie\grqq{}, oder: \glqq 2. Branchenstrukturanalyse\grqq{}.

In einer Hausarbeit ist es möglich, dass Sie sich auf nur eine theoretische Grundlage beziehen. In einer Abschlussarbeit sind es in der Regel zwei oder drei theoretische Grundlagen, daraus ergeben sich die Kapitelnummern 2, 3 und 4. – Diese Entscheidung treffen Sie nach Absprache mit Ihren Prüfern.

Die Reihenfolge der Kapitel ergibt sich aus der Allgemeinheit oder Breite der Themen. Sie beginnen mit dem allgemeinsten, z.B.: 2. Aufmerksamkeit und die Wahrnehmung von Sinnesreizen; 3. Verarbeitung visueller Reize; 4. Prinzipien für die Gestaltung visueller Elemente. – Die Regel für diese Anordnung ist selbstverständlich nicht eindeutig und muss von Ihnen gedeutet werden. Damit stecken Sie mitten in der Konstruktion einer schlüssigen Argumentationskette, und das ist ein wesentlicher Teil Ihrer Prüfungsleistung.

\item Insbesondere bei Abschlussarbeiten wenden Sie die allgemeinen Erkenntnisse aus den theoretischen Grundlagen auf einen konkreten Fall z.B. aus der Wirtschaft an. Dann beschreiben Sie auch diesen Zusammenhang (Branche, Unternehmen oder Organisation, Produkt oder Dienstleistung...) in einem eigenen Kapitel, z.B.: \glqq 5. Automobilbranche\grqq{}, oder: \glqq 5. Der Automobilzulieferer Coroplast«. Im engeren Sinn handelt es sich dabei nicht um theoretische Grundlagen. Gleichzeitig handelt es sich auch nicht um die Anwendung einer empirischen oder naturwissenschaftlichen Methode. Deshalb befindet sich dieses Kapitel zwischen den theoretischen und analytischen Kapiteln wie ein Scharnier.

\item Anschließend führen Sie Ihre Analyse durch, z.B. mit einer empirischen oder ingenieurwissenschaftlichen Methode. Es könnte den Titel tragen: \glqq 6. Empirische Untersuchung\grqq{} und enthält folgende Abschnitte:
  \begin{itemize}
  \item Darstellung der gewählten Methoden(n), z.B. Experteninterview.
  \item Schilderung Ihrer konkreten Untersuchung, z.B. Fragebogen, Datensammlung, Auswertung.
  \item Beschreibung der Untersuchungsergebnisse.
  \item Beantwortung der Forschungsfrage (wichtig: wenn Sie Ihre eingangs formulierte Forschungsfrage nicht explizit beantworten, ist das ein gravierender Mangel).
  \end{itemize}

\item Ihr Text endet mit der Zusammenfassung der Ergebnisse. In diesem Kapitel
  \begin{itemize}
  \item stellen Sie die wesentlichen Erkenntnisse Ihrer Arbeit kompakt zusammen, diskutieren diese und verorten Sie evtl. im Diskurs (wichtig: Sie formulieren hier keine neuen Argumente),
  \item reflektieren Sie selbstkritisch Ihre Anwendung der wissenschaftlichen Methode (was hat gefehlt, was würden Sie beim nächsten Mal berücksichtigen...)
  \item und formulieren Sie evtl. einen Ausblick auf Konsequenzen für künftige Forschung oder weitere Perspektiven auf das Thema.
  \end{itemize}
\end{enumerate}

\begin{itemize}[label={--}]
\item Der Textteil Ihrer Arbeit ist damit abgeschlossen. Es folgen die Verzeichnisse in dieser Reihenfolge ohne Kapitelnummern:
  \begin{itemize}[label={\bullet}]
  \item evtl. Abkürzungsverzeichnis,
  \item evtl. Formelverzeichnis,
  \item evtl. Abbildungsverzeichnis,
  \item evtl. Tabellenverzeichnis,
  \item evtl. Promptverzeichnis,
  \item evtl. Quellenverzeichnis oder Verzeichnis unveröffentlichter Quellen (falls Sie zwischen Quellen und Literatur unterscheiden müssen, das besprechen Sie bitte mit Ihren Prüfern),
  \item Literaturverzeichnis und
  \item evtl. Verzeichnis der Anhänge.
  \end{itemize}

\item Daran schließen sich evtl. Anhänge an (für Dokumente, Tabelle, Interviews, Fragebögen etc.), die mit Großbuchstaben sortiert werden, z.B.:
  \begin{itemize}[leftmargin=*,label={}]
  \item Anhang A: Fragebogen
  \item Anhang B: Vollständige Transkriptionen
  \item Anhang C: Häufigkeitstabellen
  \item Anhang D: Standbilder
  \end{itemize}

\item Dann folgt Ihre Eigenständigkeitserklärung. Sie unterschreiben sie bitte eigenhändig mit Datum. Ganz wichtig: Wenn Ihre Eigenständigkeitserklärung fehlt, wird Ihre Arbeit zwangsläufig mit 5 bewertet. Eine Vorlage für den Wortlaut Ihrer Eigenständigkeitserklärung finden Sie in Anhang B.
\item Ihre Abschlussarbeit enthält zuletzt noch einen tabellarischen Lebenslauf (ohne Portraitfoto).
\item Der Rückdeckel (Karton) ist unbedruckt.
\end{itemize}

\section{Die Gliederung ist formal korrekt}
\label{sec:gliederung-formal-korrekt}

Die Gliederung erfasst den Textteil Ihrer Arbeit (Einleitung bis Schluss) und zeigt ihren logischen Aufbau übersichtlich und aussagekräftig.

Damit Ihre Gliederung formal korrekt ist, achten Sie auf folgende Punkte:
\begin{itemize}[label={--}]
\item Alle Überschriften von Kapiteln (1. Erste Hierarchie), Abschnitten (1.1. Zweite Hierarchie) und Unterabschnitten (1.1.3. Dritte Hierarchie) sind beim Text und beim Inhaltsverzeichnis identisch.
\item Die Überschriften sind aussagekräftig und spezifisch. Es gibt keine Wiederholungen von übergeordneten Überschriften.
\item Die Überschriften sind keine vollständigen Sätze und enden deshalb nicht mit einem Satzzeichen (Punkt, Fragezeichen, Ausrufezeichen).
\item Anstelle von finiten Verbformen (\glqq Hier stelle ich die Methode dar\grqq{}) verwenden Sie Nominalisierungen (\glqq Darstellung der Methode\grqq{}).
\item Sie verwenden höchstens drei Hierarchieebenen (1.1.3. Dritte Hierarchieebene). Es gibt keine weitere Unterteilung Ihres Textes, also auch keine Überschriften ohne vorangestellte Nummerierung.
\item Kein Abschnitt oder Unterabschnitt steht alleine: Wenn Sie z.B. einen Abschnitt 3.1. haben, benötigen Sie auch zwingend einen Abschnitt 3.2.
\item Unter jeder Überschrift steht ein Textabschnitt. Bei Kapiteln können dies z.B. einführende und methodisch erläuternde Aussagen sein.
\end{itemize}

\section{Die Gliederung ist folgerichtig formuliert und aussagekräftig}
\label{sec:gliederung-folgerichtig}

Sie erkennen, ob Ihre Gliederung folgerichtig und aussagekräftig ist, wenn allein durch die Lektüre des Inhaltsverzeichnisses eine recht klare Vorstellung davon entsteht, worum es in Ihrer Arbeit geht und wie sich Ihre Argumentation darstellt. Die Kernaussagen lassen sich zwar nicht daran ablesen, aber es wird deutlich, an welchen Stellen Sie welche Arten von Aussagen treffen.

\section{Die Gliederung verfügt über eine der Themenstellung angemessene Tiefe}
\label{sec:gliederung-tiefe}

Sie haben Ihre Gliederung ausgewogen eingeteilt, wenn sich ein Unterabschnitt (z.B. 3.2.4.) über ca. eine halbe bis zwei Seiten erstreckt. Falls Sie sehen, dass Ihr Text zu einem Abschnitt oder Unterabschnitt sehr kurz oder sehr lang ist, dann prüfen Sie, ob Sie diese Passage sinnvollerweise mit einer anderen zusammenführen bzw. in kleinere Einheiten aufteilen sollten.

Vermeiden Sie mehr als neun Kapitel in Ihrer Gliederung.

\section{Die Strukturierung der Argumentation folgt stringent einer wissenschaftlichen Methode}
\label{sec:strukturierung-methode}

Wissenschaft beruht auf der Anwendung spezifischer, etablierter Methoden. In Ihrem Studium weisen Sie durch Ihre Haus- und Abschlussarbeiten nach, dass Sie sie beherrschen.

Nach Absprache mit Ihren Prüfern entscheiden Sie sich in der Regel für eine dieser Methoden:
\begin{itemize}[label={--}]
\item Deduktive Methode: Schlussfolgerung vom allgemeinen Modell auf spezifische Aussagen im konkreten Einzelfall.
\item Induktive Methode: Ableitung allgemeiner Aussagen auf der Grundlage spezifischer (z.B. empirischer) Daten.
\item Kausale Methode: Untersuchung von Ursachen und Wirkungen.
\item Dialektische Methode: Aufstellen von These, Gegenthese und abschließender Synthese.
\item Vergleichende Methode: Vergleich nach Objekten und Kriterien als Grundlage für Folgerungen.
\end{itemize}
   \chapter{Konzeptionell-theoretische Grundlagen}
\label{chap:konzeptionell-theoretische-grundlagen}

Dieser Bereich Ihrer Leistungen beeinflusst die Bewertung stärker als die beiden vorigen. Hier zeigt sich besonders deutlich, ob Sie das wissenschaftliche Arbeiten beherrschen. Deshalb können hierbei Fehler gravierende Folgen haben.

Die theoretischen Grundlagen nehmen etwa die Hälfte Ihres Textes ein. Abweichungen sind, nach Absprache mit Ihre Prüfern, möglich.

\section{Ein konzeptionell-theoretischer Anteil ist vorhanden}
\label{sec:konzeptionell-theoretischer-anteil}

Es ist nicht möglich, mit einer wissenschaftlichen Arbeit zu bestehen, die ohne konzeptionell-theoretische Fundierung auskommt. Wenn dieser Teil Ihrer Arbeit erhebliche inhaltliche und/oder formale Mängel aufweist, ist die Wahrscheinlichkeit hoch, dass sie mit der Note 5 bewertet wird.

\section{Definitionen zentraler Begriffe sind vorhanden}
\label{sec:definitionen-begriffe}

Definitionen sind Abgrenzungen des Gehalts von Begriffen. Mit Definitionen wird geklärt, was gemeint ist. Deshalb sind Definitionen fürs wissenschaftliche Arbeiten unerlässlich. Kategorien, Kriterien und Konzepte müssen definiert werden, bevor eine Analyse stattfinden kann.

Diese Definitionen formulieren Sie in der Regel nicht selbst (Ausnahmen klären Sie mit Ihren Prüfern): Sie beziehen sich auf bestehende Definitionen. Dadurch, dass Sie die maßgeblichen Definitionen in ihrem Zusammenhang darstellen, machen Sie für Ihr Publikum nachvollziehbar, inwiefern Sie den Diskurs zu Ihrem Thema beherrschen.

\section{Eine Darlegung der für die Untersuchung benötigten Grundlagen und Forschungsstände ist vorhanden}
\label{sec:darlegung-grundlagen}

Höchstwahrscheinlich liegen auch zu Ihrem Thema konkurrierende Theorien vor (mit unterschiedlichen Definitionen), die sich teilweise ergänzen oder widersprechen und unterschiedliche Schwerpunkte setzen. Oder es gibt technische Rahmenbedingungen, die für Ihre Bearbeitung grundlegend sind. Sie stellen diesen aktuellen Stand der Forschung bzw. der Technik dar und entscheiden sich danach für eine oder mehrere Theorien bzw. Techniken als Grundlage Ihrer folgenden Analyse, empirischen Studie oder Bearbeitung.

\section{Die Theoriewahl bzw. Vorgehensweise wird inhaltlich sinnvoll begründet}
\label{sec:theoriewahl-begruendung}

Ihre Entscheidung für eine oder mehrere Theorien, der bzw. denen Sie in Ihrem weiteren Text folgen, begründen Sie überzeugend. Respektive: Sie begründen Ihre Vorgehensweise bzw. Ihr Programm zur Bearbeitung der Aufgabenstellung. Ihre Wahl darf also weder willkürlich ausfallen, noch dürfen dafür praktische oder Bequemlichkeitsgründe ausschlaggebend sein. Den logisch zwingenden Zusammenhang zwischen Ihrer Entscheidung und Ihrem Thema sowie Ihrer Forschungsfrage müssen Sie nachvollziehbar darlegen.

\section{Die wesentlichen Themenaspekte werden berücksichtigt, eine Reduktion auf bestimmte Teilaspekte wird begründet}
\label{sec:themenaspekte-begruendung}

Jedes Thema kann prinzipiell uferlos ausgeführt werden, wenn Sie sich davon leiten lassen, dass mittelbar alles mit allem zusammenhängt. In Ihrer wissenschaftlichen Arbeit setzen Sie aber Ihrem Thema Grenzen (indem Sie nur ausgewählten Theorien und deren Definitionen folgen). Daraus ergibt sich zwangsläufig, dass Sie nicht alle Aspekte Ihres Themas bearbeiten können. Ihre Entscheidung, was für Ihr Thema wesentlich ist, stellt einen wichtigen Teil Ihrer Prüfungsleistung dar. Sie müssen deshalb Ihre Auswahl überzeugend begründen.

\section{Zusätzlich bei Master-Thesis: Die Arbeit identifiziert Lücken im Stand der Forschung, die im Verlauf geschlossen werden sollen}
\label{sec:masterthesis-forschungsluecken}

Der höhere Anspruch an eine Master-Thesis (verglichen mit einer Bachelor-Thesis) drückt sich u.a. darin aus, dass Sie darin eine eigenständige Forschung durchführen. Auch wenn diese Untersuchung eng begrenzt ist, so beantworten Sie damit doch eine Frage, die bisher in der wissenschaftlichen Literatur noch nicht beantwortet wurde. – In einer Bachelor-Thesis ist diese Leistung nicht erforderlich. Die deskriptiven Anteile sind darin größer und die analytischen Teile sind kleiner als bei einer Master-Thesis. – Im Sinne der Statistik können sowohl deskriptive als auch analytische Anteile sowohl in einer Bachelor- als auch in einer Masterthesis enthalten sein.
   \chapter{Durchführung}
\label{chap:durchfuehrung}

Dieser Bereich stellt den Kern Ihrer wissenschaftlichen Arbeit dar. Um es mit einem Vergleich zu sagen: Die übrigen Kapitel bilden den Rahmen, während dieser Textteil das eigentliche Bild ist. Sie liefern hier die tatsächliche Umsetzung Ihres Untersuchungsvorhabens, das Sie zuvor vorgestellt haben und nachfolgend auswerten. Ausschlaggebend ist insbesondere die inhaltliche Qualität: Welche Signifikanz haben Ihre Ergebnisse, welchen Wert haben Ihre Lösungen, was ist bei Ihren Untersuchungen herausgekommen, welchen Beitrag zur allgemeinen Erkenntnis konnten Sie leisten bzw. welchen praktischen Nutzen konnten Sie für einen konkreten Anwendungsfall erzeugen?

\section{Eine konkrete Problemstellung ggf. aus der Praxis wird identifiziert, analysiert und strukturiert beschrieben}
\label{sec:problemstellung-beschreibung}

Die Grundlage Ihrer Leistung besteht darin, dass Sie eine konkrete Situation aus der Praxis erkennen, durchdringen und darstellen können. Sie analysieren systematisch die Ausgangssituation, definieren randscharf das Problem – damit ist gemeint: eine Situation, die nicht gewünscht ist und deshalb verändert werden soll – und leiten daraus eine klares Ziel ab. In Ihrer Analyse beleuchten Sie dabei mit angemessenen Anteilen sowohl die theoretischen als auch die praktischen Aspekte des Problems.

\section{Die recherchierten wissenschaftlichen Methoden werden auf den Untersuchungsgegenstand angewandt}
\label{sec:methoden-anwendung}

Um Ihr Ziel zu erreichen, wählen Sie eine etablierte wissenschaftliche Methode oder sie kombinieren mehrere. Sie begründen nachvollziehbar, warum diese Wahl am besten geeignet ist für Ihren Untersuchungsgegenstand und Ihr Ziel. Durch die korrekte Anwendung der Methode(n) führen Sie vor, dass Sie diese sicher beherrschen.

\section{Mittels einer geeigneten, systematischen und strukturierten Vorgehensweise wird für die Problemstellung eine Lösung entwickelt}
\label{sec:loesung-entwicklung}

Das Ziel Ihrer Arbeit besteht darin, eine Antwort auf eine Frage zu formulieren, oder anders gesagt: eine Lösung für ein Problem zu finden. Diese Antwort bzw. Lösung enthält immer eine Erkenntnis. Es ist auch denkbar, dass das Problem ganz anders gelöst wird: durch Intuition, Zufall oder Magie. Aber dann wäre es eben keine wissenschaftliche Vorgehensweise. Ihre Arbeit ist deshalb im Grunde nichts anderes als eine systematische Dokumentation aller einzelnen Entwicklungsschritte, die sie auf dem Weg zu Ihrer Antwort vollzogen haben bzw. ein Arbeitsbericht mit dem Nachweise Ihrer Methode und der Dokumentation Ihrer Ergebnisse.

\section{Problemanalyse, Methodenauswahl und -anwendung sowie Entwicklung des Lösungsansatzes erfolgen korrekt und eigenständig}
\label{sec:korrekte-eigenstaendige-durchfuehrung}

Mit dieser Anforderung werden zwei Selbstverständlichkeiten für die Bewertung Ihrer Leistung zum Ausdruck gebracht: Je weniger Fehler Ihnen bei der Durchführung Ihrer Arbeit unterlaufen, desto besser. Darüber hinaus müssen Sie Ihre Arbeit eigenständig erbracht haben (dafür stehen Ihnen Hilfsmittel zur Verfügung, wie z.B. Veröffentlichungen oder KI, aber diese müssen Sie vollständig angeben), andernfalls handelt es sich um einen schwerwiegenden Täuschungsversuch, der zum Nichtbestehen führt.

\section{Der entwickelte Ansatz ist zur Lösung der vorliegenden Problemstellung geeignet und umsetzbar}
\label{sec:ansatz-geeignet}

Es kommt bisweilen vor, dass die Bearbeitung zwar ein Ergebnis hervorbringt, aber dies ist keine Lösung für das von Ihnen identifizierte Problem (sondern für ein anderes Problem). Stellen Sie deshalb sicher, dass Ihr entwickelter Ansatz auch tatsächlich die von Ihnen definierte Situation effektiv addressiert. Unter Umständen kann es erforderlich sein, dass dazu auch eine kritische Prüfung der praktischen Umsetzbarkeit in der Praxis sowie die Diskussion möglicher Limitationen Ihres Vorschlags zählt. Im Zweifel stimmen Sie sich darüber mit Ihren Prüfern ab.

\section{Vorgehensweise und Problemlösung werden in strukturierter Form schriftlich und für den kundigen Fachleser verständlich dokumentiert}
\label{sec:dokumentation}

Sie müssen davon ausgehen, dass Sie sich mit Ihrer Arbeit an ein Fachpublikum richten. Deshalb wird von Ihnen erwartet, dass Sie Ihre Arbeit in einer klaren, nüchternen und präzisen Fachsprache verfassen. Außerdem wählen Sie eine für Fachleute angemessene Darstellungstiefe. Eine logische und angemessene Struktur Ihrer Ausführungen hilft Ihrem Publikum dabei, Ihre Argumentation effizient nachzuvollziehen: Sie möchten ihm eine zeitraubende Lektüre umständlicher Gedankengänge ersparen. Weitere Hinweise zu diesem Punkt finden Sie auch unter Abschnitt 6.

\section{Abbildungen, Diagramme, Grafiken und Schaubilder sind in ausreichender Zahl enthalten. Sie sind formal und inhaltlich korrekt, mit korrekter Quellenangabe versehen und ergänzen bzw. illustrieren den Text in angemessener Weise}
\label{sec:abbildungen}

Weitere Informationen hierzu finden Sie unter Punkt 6.7.

\section{Zusätzlich bei Masterthesis: Gegenüber einer Bachelorarbeit wird ein deutlich höherer Abstraktionsgrad erreicht}
\label{sec:masterthesis-abstraktionsgrad}

In einer Bachelorthesis führen Sie vor, dass Sie den Stand der Forschung kennen und das wissenschaftliche Arbeiten beherrschen. Inhaltlich bewegen Sie sich auf einem deskriptiven Niveau. Das bedeutet: Sie referieren Erkenntnisse aus dem relevanten wissenschaftlichen Diskurs.

In einer Masterthesis liefern Sie darüber hinaus auch analytische Anteile. Sie zeigen darin, dass Sie die allgemeinen Erkenntnisse auf einen neuen Sachverhalt anwenden können. Darin besteht im wesentlichen die höhere Abstraktionsleistung einer Masterthesis: Sie sind dazu in der Lage, die relevanten allgemeingültigen Aussagen in einem konkreten Kontext sinnvoll und passend zu verorten und daraus neue Schlüsse zu ziehen. – Im Sinne der Statistik können sowohl deskriptive als auch analytische Anteile sowohl in einer Bachelor- als auch in einer Masterthesis enthalten sein.

\section{Zusätzlich bei Masterthesis: Die anhand einer konkreten Problemstellung erarbeiteten Methoden, Verfahren oder Konzepte sind zumindest in Teilaspekten unabhängig vom Anwendungsfall und über diesen hinaus gültig. Sie sind auf ähnlich gelagerte Fälle übertragbar}
\label{sec:masterthesis-uebertragbarkeit}

Der Anspruch an eine Masterthesis besteht nicht darin, dass Sie allgemeingültige, d.h. repräsentative Aussagen treffen. Es wird aber erwartet, dass Ihre Erkenntnisse und Antworten bzw. Lösungen über den von Ihnen untersuchten Einzelfall hinaus auch in anderen Situationen angewandt werden können, welche mit Ihrer Situation vergleichbar sind. So könnte etwa der Vorschlag für das Unternehmen, dessen Fall Sie bearbeitet haben, auch für unmittelbare Wettbewerber gültig sein (wenn die Unternehmen in den für Sie relevanten Aspekten vergleichbar sind). Was das für Ihre Masterthesis konkret bedeutet, hängt von der Abstimmung mit Ihren Prüfern ab.

\section{Zusätzlich bei Masterthesis: Sie werden am konkreten Anwendungsfall validiert}
\label{sec:masterthesis-validierung}

Die Entwicklung einer Lösung für das von Ihnen untersuchte Problem bedeutet auch, dass Sie die Anwendung der Lösung in Ihre Arbeit einbeziehen und untersuchend oder zumindest gedanklich deren Validität überprüfen und die Ergebnisse kritisch bewerten.
   \chapter{Quellen und Zitierweise}
\label{chap:quellen-zitierweise}

International haben sich unterschiedliche Standards für die Nachweisführung in wissenschaftlichen Texten etabliert. Sie werden üblicherweise nach den akademischen Institutionen benannt, die sie verantworten, z.B. American Psychological Association (APA), Chicago University oder Harvard University.

An der Rheinischen Hochschule Köln sollen aus Gründen der Einfachheit die Regeln angewendet werden, die im jeweils aktuellen Stand des Chicago Manual of Style festgelegt sind. Konkret handelt es sich um die Zitierweise, die aus Fußnoten und Bibliographie (Notes and Bibliography) besteht.

Allein für die Studiengänge Psychologie und Wirtschaftspsychologie gilt die Zitierweise nach APA, weil APA in diesen Wissenschaftsdisziplinen weltweit als zwingender Standard betrachtet wird.

Die Chicago-Regeln für die meisten Anwendungsfälle finden Sie in der separaten Anleitung der Rheinischen Hochschule fürs Zitieren in wissenschaftlichen Arbeiten.

Falls Sie von den Chicago-Regeln abweichen und einen anderen Zitierstil verwenden wollen, müssen Sie sich darüber mit Ihren Prüfern vorweg abstimmen. Die für die Bewertung relevanten Details anderer Zitierstile müssen sie dann auch eigenständig kontrollieren.

\section{Problemadäquate, wissenschaftliche Quellen (Monographien, wissenschaftliche Zeitschriften, Working Paper etc.) werden in angemessenem Umfang berücksichtigt}
\label{sec:wissenschaftliche-quellen}

Achten Sie darauf, dass Sie für Ihre Arbeit nicht nur Online-Veröffentlichungen verwenden. Das heißt nicht, dass gedruckte Quellen prinzipiell zu bevorzugen sind. Für viele renommierte wissenschaftliche Organe ist die Online-Veröffentlichung mittlerweile selbstverständlich. Konkret folgt daraus für Ihre Recherche, dass es zweitrangig ist, ob die Veröffentlichung gedruckt oder online zugänglich ist – entscheidend ist die inhaltliche Qualität, also die Relevanz für Ihr Thema sowie die Verlässlichkeit bzw. Zitierwürdigkeit.

\section{Die Quellenauswahl entspricht dem aktuellen Forschungsstand}
\label{sec:quellenauswahl-forschungsstand}

Zu ihren grundlegenden Leistungen beim wissenschaftlichen Arbeiten zählt, dass Sie für Ihre Arbeit solche Veröffentlichungen verwenden, welche die wichtigsten Positionen des aktuellen internationalen Diskurses zu Ihrem Thema bzw. den Stand der Technik repräsentieren. Konkret bedeutet dies, dass Sie die Texte nicht zufällig ausgewählt haben (die ersten zehn Ergebnisse auf Google; das entsprechende Regal in einer Buchhandlung) oder weil sie gerade verfügbar waren, sondern weil sie zur Standardliteratur zählen bzw. weil sie für Ihr spezifisches Forschungsinteresse relevant sind.

Verfügbarkeit ist kein Ausschlusskriterium. Das heißt, wenn eine bestimmte Veröffentlichung für Ihr Thema erforderlich ist, dann wird erwartet, dass Sie sie kennen, unabhängig davon, ob sie für Sie unmittelbar verfügbar ist oder ob Sie sie aufwändig organisieren müssen (z.B. über Fernleihe).

\section{Alle Quellen sind zitierwürdig}
\label{sec:quellen-zitierwuerdig}

Eine weitere wesentliche Leistung Ihrer Arbeit besteht darin, dass Sie dafür ausschließlich zitierwürdige Veröffentlichungen verwendet haben. Es besteht ein Unterschied zwischen zitierfähig und zitierwürdig. Zitierfähig ist jede verbale Äußerung, sogar Bilder können zitiert werden. Im Verhältnis dazu sind aber nur wenige Veröffentlichung auch würdig, in einer wissenschaftlichen Arbeit zitiert zu werden.

Zitierwürdig ist eine Veröffentlichung, wenn sie sich durch Verlässlichkeit, Vertrauenswürdigkeit und Sorgfalt auszeichnet und wenn sie die Quellen ihrer Erkenntnis nachvollziehbar macht. In der Regel heißt das, dass sie wissenschaftlichen Ansprüchen genügt. Meist ist das Medium bereits ein Indiz für Vertrauenswürdigkeit: Alle Publikationen, die mehrstufige Kontrollprozesse für eine Veröffentlichung voraussetzen (z.B. Peer-Review-Verfahren), sind prinzipiell zitierwürdig.

Aus diesem Grund ist Wikipedia grundsätzlich nicht zitierwürdig: Hier gibt es keine institutionalisierte Kontrollinstanz für die Wahrheit von Aussagen.

Zitierwürdigkeit verwechseln Sie bitte nicht mit Überzeugungskraft oder Korrektheit. Auch zitierwürdige Veröffentlichungen können Fehler enthalten und Irrtümer verbreiten. Sie müssen also die veröffentlichten Aussagen in jedem Fall kritisch prüfen, ob sie Ihnen zutreffend und überzeugend erscheinen.

Auch ein journalistischer Text oder ein Video auf YouTube kann zitierwürdig sein – wenn die getroffenen Aussagen durch Quellenangaben überprüfbar sind. Es versteht sich, dass die Quellenangaben in journalistischen Publikationen und Streamingdiensten nicht den formalen Ansprüchen an einen wissenschaftlichen Text entsprechen. Sie müssen deshalb im Einzelfall kritisch prüfen, ob die Angaben ausreichen, um die Quellen dieser Veröffentlichung zu kontrollieren.

Angenommen, in einem Video heißt es: »Nach Immanuel Kant ist nicht nur der menschliche Verstand die Quelle für Erkenntnis, sondern auch die Sinne dienen dazu.« Wenn im Video als Nachweis lediglich der Verweis auf Kant allgemein oder auf seine Kritik der reinen Vernunft angeführt wird, genügt das nicht. Denn damit Sie diese Behauptung des Videos prüfen können, benötigen Sie den Titel, die Auflage und die Seitenzahl des Textes, auf den sich das Video bezieht.

Zuletzt achten Sie darauf, dass Sie immer und ausschließlich mit der aktuellsten Auflage einer Veröffentlichung arbeiten. Jede Veröffentlichung kann Fehler enthalten. Es ist deshalb wahrscheinlich, dass in der neuesten Auflage eventuelle Fehler aus früheren Auflagen korrigiert sind.

\section{Die Berücksichtigung praxisnaher Informationen (z. B. Firmen- und Branchenspezifika) ist gegeben, falls zusätzlich erforderlich}
\label{sec:praxisnahe-informationen}

In vielen Fällen beschäftigen sich die wissenschaftlichen Arbeiten an der Rheinischen Hochschule mit wirtschaftlichen Zusammenhängen. Wenn das auch für Ihre Arbeit gilt, dann wird erwartet, dass Sie spezifische, aktuelle und verlässliche (also: zitierwürdige) Informationen z.B. über ein Unternehmen und seine Branche, Datenblätter oder Produktinformationen verarbeiten.

Sollten diese Informationen vertraulich sein, können Sie Ihre Arbeit vorab sperren lassen (bitte achten Sie auf eventuelle Fristen). Dann wird Ihre Arbeit nur von Ihren Prüfern gelesen und kann danach erst nach Bewilligung durch den Prüfungsausschuss ausgehändigt werden.

\section{Eine kritische Distanz bei der Auswertung der Quellen ist erkennbar}
\label{sec:kritische-distanz}

Das Selbstverständnis in der Wissenschaft geht davon aus, dass sich alle Menschen irren können, dass allen Menschen versehentliche Fehler unterlaufen und dass niemand die Wahrheit für sich gepachtet hat. Deshalb gibt es in der Wissenschaft per se keine Autorität, also keine Instanz, die für sich in Anspruch nehmen kann, dass sie allein wegen ihrer Position (z.B. Macht, Status, Funktion) Recht hat. – In anderen Zusammhängen kann das anders sein: Beispielsweise ist in der katholischen Kirche festgelegt, dass der Papst immer die Wahrheit spricht. –

Wenn Sie wissenschaftlich arbeiten, bedeutet das, dass Sie grundsätzlich gegenüber jeder Aussage kritisch eingestellt sind – auch gegenüber Ihren eigenen Aussagen! Sie unterstellen keine böse Absicht. Sie gehen lediglich davon aus, dass es sich bei der Aussage, die Sie gerade lesen, um einen Irrtum handeln kann. Sie folgen also einer Veröffentlichung in keinem Fall blind und unkritisch, nur weil Sie z.B. von einem erfolgreichen Menschen, einem Genie oder einer Koryphäe geäußert wurde. Auch Albert Einstein hat sich geirrt. Sie werden staunen, wie oft selbst vielzitierte Standardwerke Fehler enthalten.

Sprachlich bringen Sie Ihre kritische Haltung dadurch zum Ausdruck, dass Sie konsequent unpersönlich formulieren. Was Sie beobachten, können prinzipiell alle Menschen beobachten, deshalb kommt das \glqq Ich\grqq{} in Ihrem Text nicht vor.

Die einzige Ausnahme von dieser Regeln kann die Beschreibung Ihrer persönlichen Motivation für die Themenwahl in der Einleitung sein. Aber das ist keine Pflicht, auch die Motivation lässt sich unpersönlich formulieren.

\section{Die exakte Kenntlichmachung aller fremden Quellen durch korrekte, konsistente Zitiertechnik ist gegeben}
\label{sec:kenntlichmachung-quellen}

Weil Ihre Arbeit auf Erkenntnissen beruht, die Sie sich durch die Lektüre von Veröffentlichungen angeeignet haben, ist der Nachweis dieser fremden Aussagen in einheitlicher Form eine weitere zentrale Leistung beim wissenschaftlichen Arbeiten.

Beim Zitieren geht es im Kern darum, dass Sie zwischen Ihren eigenen Gedanken und denen anderer Menschen unterscheiden; dass Sie diese Unterscheidung erkennbar machen; und dass Sie Ihr Publikum dazu in die Lage versetzen, zu prüfen, ob Sie die fremden Gedanken korrekt wiedergegeben haben (weil auch Sie sich irren oder etwas missverstehen können).

Grundsätzlich müssen Sie jede fremde Aussage, die Sie übernehmen, als Zitat kennzeichnen. Es gibt von dieser Regel keine Ausnahme, wie auch die Gerichte in Deutschland immer wieder betonen.\footnote{Zuletzt z.B. Verwaltungsgerichts Berlin, 3. Kammer, Aktenzeichen 3 K 176/20; Arbeitsgericht Bonn, 2. Kammer, Aktenzeichen 2 Ca 345/23.} Deshalb kann es vorkommen, dass in Ihrem Text über längere Strecken fast jeder Satz einen Nachweis enthält.

Wenn Sie versehentlich ein oder zwei Zitate falsch angegeben haben oder falls Ihnen im Laufe der Bearbeitung ein Nachweis verloren gegangen ist, dann wird dies zwar Auswirkungen auf Ihre Note haben. Dramatisch wird es erst, sobald Ihre Prüfer darin ein System erkennen. Fehlende oder fehlerhafte Nachweise als Muster der Textproduktion können im besten Fall als Unfähigkeit gedeutet werden, die wissenschaftliche Methode anzuwenden – dann wird Ihre Arbeit \glqq nur\grqq{} mit 5 bewertet. Wenn sich aber begründen lässt, dass Sie mit Absicht systematisch die Quellen Ihrer Erkenntnis verschwiegen oder verschleiert haben, dann drohen schwerwiegende Konsequenzen. Dafür ist die Anzahl der relevanten Textstellen irrelevant, wichtiger ist das inhaltliche Gewicht dieser Passagen. Die Bewertung Ihrer Arbeit kann bis zu zehn Jahre nach der Abgabe revidiert werden.

Lassen Sie es nicht darauf ankommen und nehmen Sie sich die Zeit, gründlich jede fremde Aussage als Zitat zu kennzeichnen. Wenn Sie von Anfang an mit einem effizienten System (z.B. Zotero) arbeiten, gewinnen Sie nicht nur Zeit, sondern auch Sicherheit. Am Ende lässt sich aber nicht leugnen, dass Wissenschaft Arbeit bedeutet, die Aufmerksamkeit, Sorgfalt, Genauigkeit und Mühe kostet. Planen Sie deshalb genügend Zeit für Ihre Bearbeitung ein.

\section{Wörtliche Zitate sind kurz und selten}
\label{sec:woertliche-zitate}

Auch wenn Ihre Arbeit wesentlich auf Veröffentlichungen beruht, so ist Ihr Text dennoch keine Zitatsammlung. Es ist eine wesentliche Leistung, dass Sie Ihre Gedanken formulieren, die Sie in der kritischen Auseinandersetzung mit der relevanten Literatur entwickelt haben. Um einzelne wichtige Stellen pointiert herauszuarbeiten, kann ein prägnantes wörtliches Zitat angemessen erscheinen. In den meisten Fällen beziehen Sie sich aber auf fremde Aussagen, indem Sie sie paraphrasieren.

\section{Sinngemäße Zitate sind neu formuliert}
\label{sec:sinngemaesze-zitate}

Das sinngemäße oder indirekte Zitat (Paraphrase) ist eine fremde Aussage, die Sie in Ihren eigenen Worten formulieren. Dadurch können Sie Ihre Argumentationskette flüssig aufbauen.

Wenn Sie fremde Aussagen wesentlich unverändert übernehmen, wird dadurch unter Umständen nicht deutlich, ob Sie deren Gehalt verstanden haben und ob Sie sie überhaupt kritisch geprüft haben. Es kann sogar vorkommen, dass ein falsches indirektes Zitat als Plagiat gewertet wird, und wenn es häufiger vorkommt, dass Ihre Arbeitsweise als systematischer Täuschungsversuch aufgefasst wird.

\section{Zu jeder Referenzierung ist eine Angabe im Verzeichnis vorhanden. Umgekehrt sind keine Verzeichniseinträge vorhanden, zu denen die Referenzierung im Text fehlt}
\label{sec:referenzierung-verzeichnis}

Sie führen den Nachweis über die von Ihnen verwendeten Textquellen an zwei Stellen in Ihrer Arbeit:
\begin{enumerate}
\item im Text unmittelbar nach Ihrem wörtlichen Zitat oder Ihrer Paraphrase in der Form einer nummerierten Fußnote und
\item im Anhang in der Form eines alphabetisch sortierten Literaturverzeichnisses (wenn Sie nicht nur schriftliche, sondern auch mündliche und/oder audiovisuelle Textquellen verwenden, legen Sie dafür separate Verzeichnisse an).
\end{enumerate}

Es ist zwingend erforderlich, dass Sie alle Nachweise an diesen beiden Stellen führen: Es gibt keinen Veröffentlichung in einer Fußnote, die im Literaturverzeichnis fehlt, und es gibt keine Eintrag im Literaturverzeichnis, der in keiner Fußnote aufgeführt wird.
   \chapter{Form und Stil}
\label{chap:form-stil}

Was Sie sagen und was Ihr Publikum versteht, hängt untrennbar damit zusammen, wie Sie es sagen. Es genügt deshalb nicht, dass Sie Ihre Aussagen nur in zutreffende Worte fassen. Ebenso wichtig ist, dass die gesamte äußere Form der Arbeit Ihrer Sorgfalt und Genauigkeit entspricht sowie der Bedeutung, die Sie dieser Prüfungsleistung beimessen.

\section{Die Ausdrücke sind eindeutig und präzise, es werden keine umgangssprachlichen Formulierungen verwendet}
\label{sec:ausdruecke-praezise}

Weil Sprache nicht so eindeutig ist wie mathematische Formeln, sollten Sie Ihre Aufmerksamkeit darauf richten, Ihre Aussagen so klar und verständlich wie möglich zu formulieren. Das gelingt Ihnen, wenn Sie unter anderem diese Richtlinien berücksichtigen:
\begin{itemize}[label={--}]
\item Verwenden Sie aussagekräftige und aktive Verben.
\item Wägen Sie bei jeder Substantivierung ab, ob sie Ihre Aussage schärft oder verunklart.
\item Bevorzugen Sie kurze Hauptsätze.
\item Vermeiden Sie Konstruktionen mit vielen Nebensätzen.
\item Lösen Sie umständliche Passivkonstruktionen auf in mehrere Hauptsätze mit aktiven Aussagen.
\item Formulieren Sie konkret.
\item Lassen Sie die handelnden Akteure deutlich hervortreten.
\item Verzichten Sie auf umgangssprachliche Formulierungen (\glqq Klamotten\grqq{} anstelle von \glqq Kleidung\grqq{}) und Redewendungen (\glqq In der Nacht sind alle Katzen grau\grqq{}).
\end{itemize}

Grundsätzlich schreiben Sie im Präsens. Das gilt auch, wenn Sie veröffentlichte Aussagen zitieren: \glqq Immanuel Kant plädiert dafür, die eigene Meinung auf der Basis guter Gründe zu vertreten.\grqq{}

Es kann vorkommen, dass es Ihnen angemessen erscheint, eine historische Entwicklung nachzuzeichnen, z.B. die Geschichte einer Theorie. In diesem Fall verwenden Sie die Vergangenheitsform: \glqq Demokrit beschrieb im 4. Jh. v. Chr. ein Teilchenmodell. Daran knüpfte Aristoteles mit der Lehre von den vier Elementen an.\grqq{}

Sie formulieren allerdings auf keinen Fall im sog. historischen Präsens: \glqq Sigmund Freund erlebt viele Jahre lang starke Ablehnung seiner Ideen, bis sich allmählich ein Kreis aus Anhängern um ihn bildet.\grqq{}

\section{Die Spracheffizienz ist gegeben}
\label{sec:spracheffizienz}

In wissenschaftlichen Texten ist Ökonomie eine Tugend: Sie verzichten auf alles, was überflüssig ist. Das gelingt Ihnen, indem Sie Ihre verfassten Passagen schon während der Bearbeitung immer wieder darauf kritisch prüfen, ob Sie sie kürzen und durch treffendere Begriffe schärfen können.

Achten Sie darauf, keine Phrasen zu verwenden. Das sind Formulierungen, deren Wahrheitsgehalt nicht bewiesen werden muss, weil er offensichtlich ist: \glqq Wer ohne Schirm durch den Regen geht, wird nass.\grqq{} Der Gehalt dieser Aussage ist ohne jeglichen Erkenntnisgewinn. Weil Sie aber danach streben, zum Erkenntnisgewinn beizutragen, verzichten Sie auf Phrasen. – Oft erscheint es verlockend, solche Allgemeinplätze in einleitenden Passagen zu verwenden. Geben Sie dieser Versuchung nicht nach. Vielleicht finden Sie hierfür eine aktuelle Information, die zu Ihrem Thema passt (Zitierwürdigkeit beachten): \glqq Eine aktuelle repräsentative Studie zum Medienkonsum in Deutschland verdeutlicht die Bedeutung von TikTok für die politische Meinungsbildung.\grqq{}

Sie vermeiden Wiederholungen. Sie dürfen und müssen erwarten, dass Ihr Publikum fachkundig ist und sich der Lektüre Ihres Textes aufmerksam und konzentriert zuwendet. Wiederholungen lassen deshalb vermuten, dass Sie Ihre Argumentationskette nicht sorgfältig genug konstruiert haben oder dass Sie Ihrem Publikum mangelnde Aufmerksamkeit oder Vergesslichkeit unterstellen.

\section{Eine klare Gedankenführung ist erkennbar}
\label{sec:gedankenfuehrung}

Das Ziel Ihrer Textproduktion besteht darin, eine schlüssige Argumentationskette vorzulegen, um Ihr Publikum zu überzeugen. Ein Argument ist eine Aussage. Eine Argumentationskette ist also eine Aneinanderreihung von Aussagen. Überzeugend wirkt diese nicht nur durch die angeführten Beweise für Ihre Behauptungen, sondern auch dadurch, dass jede Aussage logisch nahtlos an die vorige Aussage anschließt (schlüssig).

Die Faustregel \glqq Eine Aussage, ein Satz\grqq{} hilft Ihnen bei der Formulierung Ihrer Aussagen und Ihrem Publikum bei deren Lektüre.

Eine zweite Faustregel lautet: \glqq Ein Gedanke, ein Absatz.\grqq{} Es ist unwahrscheinlich, dass Sie für die Formulierung eines Gedankens mehr als ein paar Sätze benötigen. Wenn Sie also eine Seite geschrieben haben ohne trennende Absätze, dann prüfen Sie bitte, an welchen Stellen Sie durch einen harten Zeilenumbruch deutlich machen sollten, dass hier ein neuer Gedanke beginnt.

Vermeiden Sie Verweise auf frühere oder spätere Stellen in Ihrem Text: \glqq Wie bereits in Kapitel 3 ausgeführt, ...\grqq{} oder \glqq Zur Analyse vgl. das folgende Kapitel 7.\grqq{} Mit Verweisen auf frühere Textstellen unterstellen Sie Ihrem Publikum Unaufmerksamkeit. Verweise auf folgende Stellen deuten auf Mängel in der Stringenz Ihrer Argumentationskette hin und lassen vermuten, dass es Ihnen noch nicht gelungen ist, jede Aussage an die passende Stelle zu platzieren.

\section{Eine angemessene Nutzung von Fremdwörtern und einschlägigen Fachausdrücken ist vorhanden}
\label{sec:fremdwoerter-fachausdruecke}

Fachsprache ist für wissenschaftliche Texte unverzichtbar. Sie hilft Ihnen dabei, kurz und genau zu formulieren.

Lassen Sie sich nicht davon irritieren, wenn Sie die gleichen Fachbegriffe häufig wiederholen. Abwechslung im Sprachgebrauch ist in einem wissenschaftlichen Text keine Tugend (im Unterschied zur journalistischen Reportage oder zum Roman). Sie verwenden bitte immer die gleichen Fachbegriffe für die gleichen Sachverhalte, um eindeutige und unmissverständliche Aussagen zu treffen.

Sie formulieren sachlich und nüchtern. Sie verzichten prinzipiell auf wertende Adjektive (\glqq die renommierte Forscherin\grqq{}, \glqq das erfolgreiche Unternehmen\grqq{}) und Steigerungen (\glqq die ganz besondere Marke\grqq{}).

Ihr Text muss nicht unterhalten. Deshalb sind entsprechende stilistische Empfehlungen, die für andere Textsorten gelten, fürs wissenschaftliche Schreiben nicht hilfreich. Sie konstruieren in keinem Fall einen Spannungsbogen. Ihr Publikum braucht keine Überraschung. Und Sie enthalten sich jeder Ironie.

Sie trennen strikt zwischen Darstellung und Bewertung. Ihr gesamter Text ist eine distanzierte und nüchterne Darstellung. Erst am Ende, nach der Beschreibung Ihrer Untersuchungsergebnisse, können Sie diese in einem separaten Abschnitt bewerten. Diese Aussagen kennzeichnen Sie dann ausdrücklich mit der Überschrift: Bewertung der Ergebnisse.

\section{Eine korrekte Anwendung der Regeln der Rechtschreibung, Grammatik und Interpunktion ist gegeben}
\label{sec:rechtschreibung}

Zur formalen Korrektheit Ihres Textes zählt sprachliche Fehlerfreiheit. Dies ist ein weiterer Aspekt, unter dem Ihr Publikum die Qualität Ihrer Arbeit bewertet.

Künstliche Intelligenz hilft Ihnen dabei, Ihren Text schnell und einfach auf Orthographie, Grammatik und Interpunktion zu prüfen. Denken Sie daran, die entsprechenden Prompts nachzuweisen, Hinweise dazu finden Sie in der Praktische(n) Anleitung: Zitieren in wissenschaftlichen Arbeiten der Rheinischen Hochschule Köln. Gleichzeitig muss Ihnen klar sein, dass die Ergebnisse der KI nur Vorschläge sind, auf die Sie sich nicht verlassen können.

\section{Eine korrekte äußere Form ist vorhanden}
\label{sec:aeuszere-form}

Sie drucken Ihre Arbeit auf weißes Papier, 80 bis 110 g, DIN A4 (hochformatig) einseitig auf der rechten Seite. Der Text steht im einspaltigen Blocksatz. Große Lücken, die das Schriftbild stören, vermeiden Sie, indem Sie die letzte Zeile eines Absatzes nicht bündig zum rechten Blockrand setzen, sondern sie auslaufen (flattern) lassen. Den Seitenrändern geben Sie folgende Breiten:
\begin{itemize}[label={--}]
\item links: 3,5 cm,
\item rechts: 2,5 cm,
\item oben: 3 cm,
\item unten: 2 cm.
\end{itemize}

Sie wählen eine sachliche Groteskschrift wie z.B. Helvetica oder Arial oder eine Antiquaschrift wie z.B. Times mit folgenden Auszeichnungen:
\begin{itemize}[label={--}]
\item Fließtext: regulärer oder normaler Schriftschnitt, Schriftgröße 11 pt, Zeilenabstand 1,5 Linien, 6 pt Abstand nach dem Absatz;
\item Überschriften: linksbündig, fetter Schriftschnitt, Schriftgröße 11 pt, Zeilenabstand 1,5 Linien, 6 pt Abstand nach dem Absatz;
\item Kopfzeilen mit Seitenzahl und Fußnoten: regulärer oder normaler Schriftschnitt, Schriftgröße 9 pt, Zeilenabstand 1,5 Linien, kein Abstand nach dem Absatz.
\end{itemize}

Nach einem Absatz folgt eine Freizeile als Abstand zum nächsten Absatz oder zum nächsten Unterabschnitt. Jedes Kapitel beginnt auf einer neuen Seite.

Ihr Text hat keine Überschrift als letzte Zeile auf einer Seite. Ihr Text hat auch keine einzelne Zeile als erste oder als letzte Zeile auf einer Seite. Sollten Sie am Ende Ihrer Bearbeitung, nachdem Sie alle anderen Kontrollen durchgeführt haben, solche einzelnen Zeilen bemerken, dann sorgen Sie mit harten Zeilenumbrüchen dafür, dass dieser typographische Fauxpas behoben wird.

Akademische Grade und Titel nennen Sie an keiner Stelle, weder im Text noch in den Fußnoten oder im Literaturverzeichnis. Ob die von Ihnen interviewte Expertin eine Professorin ist oder der Verfasser eines zitierten Textes einen Doktorgrad führt, ist völlig unerheblich. Die einzigen beiden Ausnahmen für diese Regel: Falls Sie bereits einen akademischen Abschluss erworben haben, nennen Sie diesen auf dem Titel, in der Eigenständigkeitserklärung und in Ihrem Lebenslauf. Das gilt auch für Ihre Prüfer auf den Titeln.

Der Schmutztitel (Vorderdeckel, auf Karton gedruckt) enthält folgende Informationen:
\begin{itemize}[label={--}]
\item Logo und Schriftzug der Rheinischen Hochschule Köln als Abbildung,
\item Bezeichnung des Fachbereichs (fetter Schriftschnitt, 14 pt)
\item Bezeichnung des Studiengangs (regulärer Schriftschnitt, 14 pt), danach eine Freizeile
\item Hausarbeit, Projektarbeit, Bachelor-Thesis oder Master-Thesis (fetter Schriftschnitt, 14 pt)
\item Thema der Arbeit (regulärer Schriftschnitt, 14 pt), danach eine Freizeile
\item vorgelegt von (regulärer Schriftschnitt, 11 pt)
\item Ihr Vor- und Nachname (regulärer Schriftschnitt, 14 pt)
\item Ihre Matrikelnummer (regulärer Schriftschnitt, 11 pt), danach eine Freizeile
\item Sommer- oder Wintersemester
\end{itemize}

\begin{figure}[h]
\centering
\caption[Beispielhafter Schmutztitel und Haupttitel für Haus- und Abschlussarbeiten an der Rheinischen Hochschule Köln,]{Beispielhafter Schmutztitel (li.) und Haupttitel (re.) für Haus- und Abschlussarbeiten an der Rheinischen Hochschule Köln, eigene Darstellung.}
\label{fig:schmutztitel-haupttitel}
\end{figure}

Der Haupttitel (auf weißem Papier wie der gesamte Inhalt gedruckt) enthält die gleichen Informationen wie der Schmutztitel mit einem Unterschied: Auf dem Haupttitel nennen Sie nach der Matrikelnummer noch die Prüfer (regulärer Schriftschnitt, 11 pt, danach eine Freizeile).

Ihre Arbeit lassen Sie durch eine Klebebindung mit unbedrucktem Textilrücken binden.

Die Farbe des Kartons für den Vorder- und Rückdeckel richtet sich nach Ihrem Studiengang. Sie finden die korrekte Farbe für Ihre Arbeit in der folgenden Tabelle \ref{tab:einbandfarben}.

\begin{table}[H]
\label{tab:einbandfarben}
\begin{tabular}{|l|l|l|}
\hline
\textbf{Fachbereich} & \textbf{Studiengang} & \textbf{Einbandfarbe} \\
\hline
\multirow{5}{*}{Ingenieurwesen} & Elektrotechnik (B.Eng.) & Schwarz \\
\cline{2-3}
 & Informatik (B.Sc.) & Hellgrün \\
\cline{2-3}
 & Maschinenbau (B.Eng.), alle Fachrichtungen & Dunkelblau \\
\cline{2-3}
 & Prozesstechnik (B.Sc.) & Hellgrün \\
\cline{2-3}
 & Wirtschaftsingenieurwesen (B.Eng.) & Dunkelgrün \\
\cline{2-3}
 & Alle Masterstudiengänge & Dunkelblau \\
\hline
\multirow{17}{*}{\begin{tabular}[c]{@{}l@{}}Wirtschaft,\\ Psychologie \&\\ Recht\end{tabular}} & Betriebswirtschaftslehre (B.A.) & Gelb \\
\cline{2-3}
 & Compliance \& Corporate Security (LL.M.) & Rot \\
\cline{2-3}
 & Digital Transformation Management (M.A.) & Gelb \\
\cline{2-3}
 & Entrepreneurship (M.A.) & Gelb \\
\cline{2-3}
 & General Management (M.A.) & Gelb \\
\cline{2-3}
 & International Business Management (M.A.) & Gelb \\
\cline{2-3}
 & MBA International Business (MBA) & Gelb \\
\cline{2-3}
 & Nachhaltigkeitsmanagement (B.A.) & Gelb \\
\cline{2-3}
 & Psychologie (B.Sc.) & Orange \\
\cline{2-3}
 & Steuerrecht (LL.M.) & Rot \\
\cline{2-3}
 & Unternehmensmanagement (B.A.) & Gelb \\
\cline{2-3}
 & Werteorientierte Unternehmensführung (M.Sc.) & Gelb \\
\cline{2-3}
 & Wirtschaftsinformatik (B.Sc.) & Grau \\
\cline{2-3}
 & Wirtschaftsinformatik (M.Sc.) & Grau \\
\cline{2-3}
 & Wirtschaftspsychologie (B.Sc.) & Orange \\
\cline{2-3}
 & Wirtschaftspsychologie (M.Sc.) & Orange \\
\cline{2-3}
 & Wirtschaftsrecht (LL.B.) & Rot \\
\hline
\multirow{6}{*}{\begin{tabular}[c]{@{}l@{}}Medien, Marketing \&\\ Innovation\end{tabular}} & Digital Business Management (M.A.) & Weiß \\
\cline{2-3}
 & International Marketing \& Media Management (M.A.) & Weiß \\
\cline{2-3}
 & Media \& Marketing Management (B.A.) & Weiß \\
\cline{2-3}
 & Mediendesign (B.A.) & Braun oder eigene \\
\cline{2-3}
 & User Experience Design (M.A.) & Weiß oder eigene \\
\hline
\multirow{5}{*}{\begin{tabular}[c]{@{}l@{}}Medizinökonomie \&\\ Gesundheit\end{tabular}} & Erweiterte Pflegepraxis (B.Sc.) & Magenta \\
\cline{2-3}
 & Gesundheitsökonomie (M.Sc.) & Magenta \\
\cline{2-3}
 & \begin{tabular}[c]{@{}l@{}}Medizinökonomie \& Digitales Management (B.Sc.)\\ (ehem. Medizinökonomie)\end{tabular} & Magenta \\
\cline{2-3}
 & Molekulare Biomedizin (B.Sc.) & Magenta \\
\cline{2-3}
 & Physiotherapie (B.Sc.) & Magenta \\
\hline
\end{tabular}
\caption{Übersicht der Einbandfarben für Haus- und Abschlussarbeiten an der Rheinischen Hochschule Köln.}
\end{table}

\section{Die Tabellen und Abbildungen sind gut lesbar}
\label{sec:tabellen-abbildungen}

Abbildungen (z.B. Fotos, Illustrationen, Schemata, Diagramme, Zeichnungen, Pläne oder Screenshots) und Tabellen sind erwünscht, weil sie verbale Aussagen veranschaulichen bzw. einen kompakten Überblick bieten können. Abbildungen sind sogar erforderlich, wenn Ihre Aussagen ohne sie nicht völlig nachvollzogen werden können. Schemata und Grafiken können Abläufe und Zusammenhänge sichtbar machen. Es gibt keine Abbildung ohne Bezug zu Ihrem Text.

Die Größe der einzelnen Abbildung hängt davon ab, ob sie einen überwiegend illustrativen oder informativen Charakter hat.
\begin{itemize}[label={--}]
\item Mit illustrativ ist gemeint, dass das Bild vor allem verbale Aussagen anschaulich macht und das Textverständnis dadurch fördert. Eine typische illustrative Abbildung ist ein Schaubild mit einer Vielzahl aktueller Online-Medien (Blogs, Websites, Streamingplattformen, Gamingplattformen, Social-Media-Kanäle etc.). Sie wird oft zur Veranschaulichung der Aussage gezeigt, dass es gegenwärtig unübersichtlich viele Online-Medien gibt. Diese Aussage ist auch ohne die Abbildung nachvollziehbar. In einem solchen Fall sollte die Abbildung nicht größer sein als ein Viertel der Seite. Sie platzieren sie mittig auf der Seite. Abbildung \ref{fig:libelle} ist dafür ein Beispiel. – Wenn Sie feststellen, dass Texte innerhalb des Bildes dann nicht mehr lesbar wären, verzichten Sie auf diese Abbildung.
\item Mit informativ ist gemeint, dass das Bild fürs Textverständnis wichtig oder sogar unverzichtbar ist. Das kann der Fall sein, wenn Sie eine Bildanalyse durchführen oder wenn Sie Details in einem Bild zum Gegenstand Ihrer Aussagen machen. Dann zeigen Sie die Abbildung in einer passenden Größe, eventuell sogar seitenfüllend.
\end{itemize}

Ihr Publikum freut sich über einen einheitlichen gestalterischen Duktus bei Tabellen und allen Formen von Abbildungen. Einheitlichkeit ist wichtiger als Finesse im Detail, es genügt eine schlichte Darstellung mit Grauschattierungen für Flächen oder Linien.

Unmittelbar unter jeder Tabelle und Abbildung steht eine nummerierte Unterschrift. Achten Sie darauf, dass diese Bild- oder Tabellenunterschrift nicht erst auf der folgenden Seite steht. Bitte kontrollieren Sie mit der Praktische(n) Anleitung: Zitieren in wissenschaftlichen Arbeiten der Rheinischen Hochschule oder dem Chicago Manual of Style, dem Reference Guide des IEEE bzw. dem Publication Manual der APA, ob Sie Ihre Bildunterschriften korrekt geschrieben haben.

\begin{figure}[h]
\centering
\caption[Veranschaulichung des Panorama-Sichtfelds einer Libelle, die visuelle Reize ihrer Umwelt mit zwei komplexen Facettenaugen und drei einfachen Linsenaugen aufnimmt]{Veranschaulichung des Panorama-Sichtfelds einer Libelle, die visuelle Reize ihrer Umwelt mit zwei komplexen Facettenaugen und drei einfachen Linsenaugen aufnimmt, Quelle: Journal of Neuroscience 28, Nr. 11 (12. März 2008), Titelbild, Quelle: https://www.jneurosci.org/content/28/11.cover-expansion.}
\label{fig:libelle}
\end{figure}

\section{Die erforderlichen Verzeichnisse sind formal korrekt}
\label{sec:verzeichnisse}

Das Inhaltsverzeichnis enthält die Überschriften aller Bestandteile Ihrer Arbeit (evtl. Vorbemerkung, Einleitung, alle Kapitel, Abschnitte und Unterabschnitte sowie Verzeichnisse und evtl. Anhänge) mit der Zahl der Seite, an der diese beginnen. Sie können als Überschrift dafür Inhaltsverzeichnis oder kurz Inhalt verwenden. Achten Sie darauf, dass das Inhaltsverzeichnis selbst kein Eintrag im Inhaltsverzeichnis ist.

Das Abkürzungsverzeichnis enthält nur Abkürzungen jenseits des allgemeinen Sprachgebrauchs (wie Abb., etc., evtl., sog., Tab., z.B.) sowie deren Auflösung.

Wie Sie Ihr Literaturverzeichnis korrekt aufbauen, finden Sie in der Praktische(n) Anleitung: Zitieren in wissenschaftlichen Arbeiten der Rheinischen Hochschule Köln.

Alle anderen Verzeichnisse sind einfache tabellarische Listen, die Sie nach dem Muster erstellen, welches Sie in Abbildung \ref{fig:verzeichnisse} finden:

\begin{figure}[h]
\centering
\caption[Muster für die Darstellung von Abbildungs-, Tabellen-, Formel- und Symbolverzeichnissen]{Muster für die Darstellung von Abbildungs-, Tabellen-, Formel- und Symbolverzeichnissen, eigene Darstellung.}
\label{fig:verzeichnisse}
\end{figure}

\section{Die Regeln zum Umfang werden eingehalten}
\label{sec:umfang}

Der Umfang Ihrer Arbeit ergibt sich bei Haus- und Projektarbeiten aus den Vorgaben Ihrer Prüfer.
\begin{itemize}[label={--}]
\item Ihre Bachelorthesis hat einen Umfang von 135.000 Zeichen inkl. Leerzeichen (das entspricht ungefähr 60 Seiten je 2.300 Zeichen inkl. Leerzeichen).
\item Ihre Masterthesis hat einen Umfang von 185.000 Zeichen inkl. Leerzeichen (ca. 80 Seiten Text).
\end{itemize}

Damit ist der Text von der Einleitung bis zum Schluss gemeint. Titel, Verzeichnisse und Anhänge werden hierbei nicht mitgezählt. Inwiefern eine Toleranz von zehn Prozent Minder- oder Mehrumfang akzeptabel sein kann, besprechen Sie mit Ihren Prüfern.

    % Abkürzungsverzeichnis
    \phantomsection
    \addcontentsline{toc}{chapter}{Abkürzungsverzeichnis}
    \label{sec:Abkuerzungsverzeichnis}
    \printnoidxglossary[title={Abkürzungsverzeichnis}, type=main, nonumberlist]
    \newpage
    
    % Abbildungsverzeichnis
    \phantomsection
    \addcontentsline{toc}{chapter}{Abbildungsverzeichnis}
    \listoffigures
    \newpage
    
    % Tabellenverzeichnis
    \phantomsection
    \addcontentsline{toc}{chapter}{Tabellenverzeichnis}
    \listoftables
    \newpage
    
    % Promptverzeichnis (für KI-Prompts)
    \phantomsection
    \addcontentsline{toc}{chapter}{Promptverzeichnis}
    \markboth{Promptverzeichnis}{} % Manuelles Setzen der Kopfzeile
    \listofprompt
    \newpage

    % Literaturverzeichnis
    \newpage
    \phantomsection
    \addcontentsline{toc}{chapter}{Literaturverzeichnis}
    \label{chap:Literaturverzeichnis}
    \markboth{Literaturverzeichnis}{} % Manuelles Setzen der Kopfzeile
    \printbibliography[title={Literaturverzeichnis}]
    
    % Anhangsverzeichnis
    \newpage
    \phantomsection
    \addcontentsline{toc}{chapter}{Verzeichnis der Anhänge}
    \listofappendix
    \newpage
    
    % Anhänge
    \newpage
    \appendix
    % Include all appendix files
\chapter*{Anhang A: Praktische Hinweise zum Gendern}
\addcontentsline{toc}{section}{Anhang A: Praktische Hinweise zum Gendern}
\label{app:gendern}

Eine häufig verwendete Möglichkeit ist das substantivierte Partizip Präsens. Es zeigt im Plural kein Genus:
\begin{itemize}
\item die Studierenden,
\item die Lehrenden.
\end{itemize}

Bitte beachten Sie, dass das substantivierte Partizip Präsens im Singular ein Genus zeigt:
\begin{itemize}
\item der Studierende ist eindeutig maskulin,
\item die Lehrende ist eindeutig feminin.
\end{itemize}

Abstrakte Rollenbezeichnungen und Begriffe für Kollektive, Institutionen und Positionen sind eine weitere Möglichkeit für sexusindifferente Formulierungen:
\begin{itemize}
\item die Vertretung,
\item die Studiengangsleitung,
\item das Kollegium,
\item das Publikum,
\item die Lehrkräfte,
\item das Team,
\item das Präsidium.
\end{itemize}

Die Pronomina alle und niemand zeigen ebenfalls kein Genus:
\begin{itemize}
\item Alle können am Kurs teilnehmen.
\item Niemand muss sich wegen der Prüfung Sorgen machen.
\end{itemize}

Gender-Zeichen, meist ein Doppelpunkt oder Sternchen (Asterisk), stehen zwischen dem Wortstamm der männlichen Form und der weiblichen Endung:
\begin{itemize}
\item Bewerber:in
\item Teilnehmer:innen
\item Student:in
\item Expert:innen
\end{itemize}

Die Beidnennung ist eine weitere Möglichkeit, also die Nennung von maskulinen und femininen Bezeichnungen:
\begin{itemize}
\item Studentinnen und Studenten,
\item Dozentinnen und Dozenten,
\item Prüferin bzw. Prüfer,
\item Expertin bzw. Experte.
\end{itemize}
\input{anhaenge/anhang-b}
\chapter*{Anhang C: Ergänzende Hinweise für Abschlussarbeiten in den Studiengängen Wirtschaftspsychologie und Mediendesign}
\addcontentsline{toc}{section}{Anhang C: Ergänzende Hinweise für Abschlussarbeiten in den Studiengängen Wirtschaftspsychologie und Mediendesign}
\label{app:wirtschaftspsychologie-mediendesign}

In diesen beiden Studiengängen können die Abschlussarbeit unterschiedlich ausgerichtet werden. Möglich sind:
\begin{enumerate}
\item theoretisch-konzeptionelle Abschlussarbeiten,
\item empirische Arbeiten (in der Wirtschaftspsychologie verpflichtend): qualitative oder quantitative Untersuchungen,
\item konzeptionell-gestalterische Arbeiten (nur im Studiengang Mediendesign).
\end{enumerate}

Eine eindeutige Zuordnung zu einem dieser Schwerpunkte ist nicht immer möglich.

In jedem Fall ist eine umfassende und systematische Darstellung des Stands der Forschung anhand der aktuellen wissenschaftlichen Literatur (Diskurs) erforderlich. Auf diesen Status Quo bezieht sich Ihre weitere Bearbeitung, bzw. Sie wenden ihn dafür an.

Alle Arbeiten können in Kooperation mit Unternehmen geschrieben werden. Bitte stimmen Sie sich dazu mit Ihren Prüfern ab.

\section*{1. Theoretisch-konzeptionelle Arbeiten}

Wenn Sie eine theoretisch-konzeptionelle Arbeit vorlegen, analysieren Sie darin ein Thema, ohne dass Sie hierzu eigene Berechnungen durchführen.

Ihr Ausgangspunkt ist eine möglichst vollständige Erfassung der relevanten Literatur und die systematische Aufbereitung der gewonnenen Informationen. Wie Sie dieses Material strukturieren und aufbereiten (Tiefe, Differenzierung, Präzision, Prägnanz, Gewichtung), bestimmt wesentlich die Qualität Ihrer Arbeit.

Typischerweise wird von Ihnen erwartet, dass Sie die wesentliche Literatur durchdringen und sich mit ihr kritisch auseinandersetzen. Dazu zählen insbesondere Definitionen (die Sie evtl. selbst entwickeln), Klassifikationsschemata oder Typologien. Zudem wird erwartet, dass sie Begründungszusammenhänge strukturierte darstellen.

Ein zentrales Ziel der theoretisch-konzeptionellen Arbeit ist die Verbesserung, Vereinfachung und vor allem Systematisierung der Darstellung bereits bekannter, aber bisher nicht in einer einzigen Arbeit zusammengefasster Fakten oder Argumente.

Den Titel Ihrer theoretisch-konzeptionellen Arbeit können Sie beispielsweise nach diesem Schema formulieren:
\begin{itemize}
\item Entwicklung einer strategisch-konzeptionellen Empfehlung
\item Formulierung von Handlungsempfehlungen für bestimmte Fälle
\item Eingrenzung von potenziellen Entwicklungsszenarien
\item Konzeption von bestimmten Maßnahmenempfehlungen für besondere Zwecke (beispielsweise aus dem konkreten Arbeitsumfeld).
\end{itemize}

Ihre Strukturierungsleistung soll zu einem Ziel führen, das bereits im Titel anklingt, z.B.: \glqq Strategische Markt- und Wettbewerbsanalyse der Videospielbranche - unter besonderer Berücksichtigung der Neuprodukteinführung eines Konsolenherstellers\grqq{}.

\section*{2. Empirische Arbeiten}

Bei Ihrer empirischen Arbeit kann es sich entweder um eine qualitative oder quantitative Untersuchung handeln.

Qualitative Untersuchungen werden durchgeführt, wenn ein wesentliches Verständnis für eine bestimmte Forschungsfrage erlangt werden soll. Ziel der qualitativen Forschung ist das Erkennen, Beschreiben und Verstehen von Zusammenhängen. Im Vordergrund stehen die vollständige Erfassung und Interpretation aller problemrelevanten Aspekte hinsichtlich des Themas der Abschlussarbeit. Hierbei bedient man sich offener, nicht standardisierter Erhebungsverfahren (z.B. Interviews, Expertengespräche, Gruppendiskussionen, qualitative Beobachtung, qualitative Experimente), deskriptiver Aufbereitungsverfahren (z.B. Gesprächsprotokoll, Transkription) und interpretativer Auswertungsverfahren (z.B. qualitative Inhaltsanalyse, Cognitive Mapping, objektive Hermeneutik). In der Regel kommen im Rahmen der qualitativen Untersuchung Methoden zum Einsatz, die sich auf kleine Fallzahlen beschränken, keine statistischen Analysen (z.B. Signifikanztests) implizieren, relative weiche Daten produzieren und ihre Erkenntnisse auf einem verhältnismäßig niedrigen Abstraktionsniveau mittels subjektiver Interpretation gewinnen. Die gewonnenen Ergebnisse sind zwar nicht repräsentativ, sie dienen aber zur ausführlichen Stoffsammlung, um ihrerseits wieder Hypothesen quantifizieren zu können.

Quantitative Untersuchungen haben die Messung bestimmter Sachverhalte bzw. die Entdeckung von Gesetzmäßigkeiten zum Gegenstand. Es handelt sich hierbei um einen Ansatz, der theoriegeleitet ist und der sich standardisierter Erhebungsmethoden (i.d.R. schriftliche Fragebögen) bedient. Ziel dieser Standardisierung ist es, die Antworten einer Vielzahl von Befragten unmittelbar vergleichen zu können. Der Vorteil quantitativer Untersuchungen liegt darin, dass sich die Messergebnisse mit statistischen Methoden (z.B. Korrelations- und Regressionsanalysen, Varianzanalysen, Faktoren- und Clusteranalysen, Metaanalysen, Diskriminanzanalysen, MDS) unter Nutzung statistischer Kennzahlen (Mittelwert, Median, Standardabweichung, Varianz etc.) verdichten und weiterverarbeiten lassen. Die Auswertung und Analyse der Daten, die im Rahmen der quantitativen Forschung generiert werden, erfolgt mit Hilfe von statistischen Programmen (z.B. SPSS, AMOS, PLS, R). Ziel ist es, Zusammenhänge zu erkennen und daraus allgemein gültige Aussagen abzuleiten. Rückschlüsse auf die tatsächlichen Verhältnisse in der Grundgesamtheit sind möglich.

In der Wirtschaftspsychologie (B.Sc.) werden ausschließlich empirische Abschlussarbeiten durchgeführt, wobei quantitativen Ansätzen der Vorzug zu geben ist.

\section*{3. Konzeptionell-gestalterische Arbeiten}

Diese Arbeit können Sie nur im Studiengang Mediendesign vorlegen. Hierbei handelt es sich um eine innovative Lösung für eine komplexe gestalterische Aufgabe.

Ihr Ausgangspunkt ist auch hier die Durchdringung relevanter wissenschaftlicher Grundlagen (insbesondere Theorien und Beispiele aus der Designgeschichte).

Daran schließen sich evtl. strategische sowie Ihre konzeptionellen und methodischen Darstellungen an, in denen Sie den Kontext der Aufgabe und Ihren Designprozess nachvollziehbar machen.

Sowohl die wissenschaftlichen als auch die methodischen und konzeptionellen Aussagen führen zu Ihrer praktischen Umsetzung im Sinn einer Lösungsentwicklung mit gestalterischen und technischen Mitteln.
\chapter*{Anhang D: Gestalterische Hinweise für die Studiengänge Mediendesign (B.A.) und User Experience Design (M.A.)}
\addcontentsline{toc}{section}{Anhang D: Gestalterische Hinweise für die Studiengänge Mediendesign (B.A.) und User Experience Design (M.A.)}
\label{app:gestalterische-hinweise}

\section*{Schmutztitel und Haupttitel}

Auf dem Schmutztitel und dem Haupttitel stehen die gleichen Inhalte wie bei allen anderen Studiengängen (vgl. Abschnitt 6.6 in diesem Leitfaden). Ansonsten sind Sie in Ihrer Gestaltung der Titel frei. Es ist auch ausdrücklich erwünscht, dass Sie von dieser Freiheit Gebrauch machen. Falls Sie sich dafür entscheiden, Ihre Arbeit nicht zu gestalten, verwenden Sie einen braunen (Mediendesign) bzw. weißen (User Experience Design) Einband. Bitte beachten Sie, dass Sie keinen schlichten andersfarbigen Einband verwenden, weil diese bereits anderen Studiengängen vorbehalten sind.

\section*{Schriften}

Beachten Sie bei der Wahl Ihrer Schrift die Angemessenheit. Neben der akademischen Ausstrahlung ist es wichtig, dass Sie sich später als Bewerber mit Ihrer Arbeit gegen Konkurrenten durchsetzten wollen. Alle Schriften für sogenannte Mengentexte sind aufgrund ihrer Lesefreundlichkeit geeignet, beispielsweise die Thesis, Info Text oder Corporate oder Schriften nach dem dynamischen Formprinzip, wie z.B. Garamond, Minion, Weidemann, Lexicon, Optima, Officina, Frutiger, Today oder Meta.

Beachten Sie bitte auch, dass die Prüfer Ihre Arbeit innerhalb kurzer Zeit lesen werden. Die Lektüre darf auch Freude bereiten, deshalb ist die Lesbarkeit bei der Auswahl ein wichtiges Entscheidungskriterium. Eine schmal laufende Schrift (condensed) ist also nicht geeignet. Serifenschriften bieten in der Regel Vorteile, sind aber kein Muss.

\section*{Raster und Layout}

Die Bachelor-Thesis soll angemessene Seitenränder haben und nur eine Textspalte, sowohl einen toten als auch einen lebenden Kolumnentitel integrieren sowie zwingend über eine Paginierung verfügen. Abbildungen sollt angemessen groß sein. Komplexe Schaubilder oder die Herleitung Ihrer Entwürfe dürfen, sofern notwendig, seitenfüllend sein. Für Infografiken entwickeln Sie idealerweise einen durchgängigen einheitlichen Stil.

\section*{Feintypografie}

Beachten Sie bitte alle feintypografischen Regeln. Halten Sie bitte sauber definierte Abstände vor und nach Überschriften oder Absätzen usw. ein. Ein Ausgleich des Flatter- oder Blocksatzes nach Fertigstellen des Textes ist Pflicht. Tipp: Beginnen Sie damit von hinten.
    
    % Eigenständigkeitserklärung
    %\newpage
    %\chapter*{Eigenständigkeitserklärung}
\addcontentsline{toc}{section}{Eigenständigkeitserklärung}
Hiermit bestätige ich, dass ich die vorliegende Arbeit selbstständig verfasst und keine 
anderen Publikationen, Vorlagen und Hilfsmittel (z.B. künstliche Intelligenz) als die 
angegebenen benutzt habe. Alle Teile meiner Arbeit, die wortwörtlich oder dem Sinn nach 
anderen Werken entnommen sind, wurden unter Angabe der Quelle kenntlich gemacht. 
Gleiches gilt für von mir verwendete Internetquellen. Ich versichere, dass ich diese Arbeit 
oder nicht zitierte Teile daraus vorher nicht in einem anderen Prüfungsverfahren eingereicht 
habe. Mir ist bekannt, dass meine Arbeit zum Zwecke eines Plagiatsabgleichs mittels einer 
Plagiatserkennungssoftware auf eine ungekennzeichnete Übernahme von fremdem geistigen 
Eigentum sowie auf die Nutzung von künstlicher Intelligenz zur Texterstellung überprüft 
werden kann. Ich versichere, dass die elektronische Form meiner Arbeit mit der gedruckten 
Version identisch ist.


% \section*{Declaration of Originality}
% I hereby confirm that I have independently written this work and have not used any 
% publications, templates, or aids (e.g. artificial intelligence) other than those I have 
% indicated. All parts of my work which have been taken literally or correspondingly from other
% publications have been duly acknowledged. This also applies to Internet sources. I confirm
% that I have not previously submitted this work or any unquoted parts thereof in any other 
% examination procedure. I am aware that my work may be checked for plagiarism by means 
% of plagiarism recognition software, as well as for the use of artificial intelligence for text
% creation, in order to verify the integrity of its written content. I also confirm that the 
% electronic form is identical to the printed version. \\

\vspace{2cm}

\begin{center}
\begin{tabular}{p{0.45\textwidth}p{0.45\textwidth}}
    [Ort], \today & \\
    \hline
    % Place, Date & Signature \\
    Ort, Datum & Unterschrift \\
\end{tabular}
\end{center}
    
    % Lebenslauf (nur für Abschlussarbeiten)
    %\newpage
    %\chapter*{Lebenslauf}
\addcontentsline{toc}{section}{Lebenslauf}
\label{sec:lebenslauf}

\begin{tabular}{p{0.25\textwidth}p{0.7\textwidth}}
\textbf{Persönliche Daten} & \\
\\
Name: & [Vorname Nachname] \\
Geburtsdatum: & [TT.MM.JJJJ] \\
Geburtsort: & [Geburtsort] \\
Staatsangehörigkeit: & [Staatsangehörigkeit] \\
\\
\textbf{Ausbildung} & \\
\\
MM.JJJJ - heute & Studium Studiengang an der Rheinischen Hochschule Köln \\
MM.JJJJ - MM.JJJJ & Vorherige Ausbildung/Studium \\
MM.JJJJ - MM.JJJJ & Schulbildung \\
\\
\textbf{Berufliche Erfahrung} & \\
\\
MM.JJJJ - heute & Aktuelle Tätigkeit \\
MM.JJJJ - MM.JJJJ & Vorherige Tätigkeit \\
MM.JJJJ - MM.JJJJ & Praktikum/Werkstudententätigkeit \\
\\
\textbf{Kenntnisse und Fähigkeiten} & \\
\\
Sprachen: & Sprache 1 (Niveau), Sprache 2 (Niveau), ... \\
EDV-Kenntnisse: & Software 1, Software 2, ... \\
Sonstige Qualifikationen: & Qualifikation 1, Qualifikation 2, ... \\
\\
\end{tabular}

\vspace{1cm}

\begin{center}
Ort, Datum
\end{center}

\end{onehalfspace}
\end{document}
