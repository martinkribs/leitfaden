% Dokumentklasse
\documentclass[11pt,titlepage]{report} % Für längere Arbeiten mit Kapiteln und Abschnitten, geeignet für Berichte und Dokumentationen.

% Sprachunterstützung
\usepackage[ngerman]{babel} % Sorgt für sprachspezifische Regeln wie Silbentrennung und deutsche Bezeichnungen.

% Symbole und Icons
\usepackage{fontawesome} % Ermöglicht die Nutzung von Font Awesome Icons.

% Seitenlayout
\usepackage[a4paper,lmargin={3.5cm},rmargin={2.5cm},tmargin={3cm},bmargin={2cm},headsep=0.5cm,headheight=1.5cm]{geometry} % Einstellung von Seitenrändern für A4-Papier.

% Mathematische Pakete
\usepackage{amssymb} % Enthält zusätzliche mathematische Symbole.
\usepackage{amsthm} % Unterstützung für Theoreme, Definitionen und Beweise.
\usepackage{amsmath} % Verbesserte Darstellung von mathematischen Formeln.

% Grafiken und Bilder
\usepackage{graphicx} % Ermöglicht das Einfügen von Bildern und Grafiken.
\usepackage{svg} % Ermöglicht das Einfügen von SVG-Grafiken.

% Typografie
\usepackage{microtype} % Verbessert die Lesbarkeit und Silbentrennung.
\usepackage{fontspec} % Für die Verwendung von TrueType- und OpenType-Schriften.

% Hyperlinks
\usepackage[hidelinks]{hyperref}% Fügt klickbare Hyperlinks ein, ohne sie farblich hervorzuheben.

% Abkürzungen
\usepackage[footnote,printonlyused]{acronym} % Verwaltung von Abkürzungen mit Fußnotenoption.

% Kopf- und Fußzeilen
\usepackage{fancyhdr} % Ermöglicht die Gestaltung von Kopf- und Fußzeilen.

% Farben
\usepackage{xcolor} % Definiert und verwendet Farben im Dokument.

% Weitere mathematische Symbole
\usepackage{amsfonts} % Zusätzliche Schriftarten für mathematische Symbole.

% Tabellen
\usepackage{booktabs} % Verbesserte Gestaltung von Tabellen.
\usepackage{enumitem} % Anpassung von Listen und Aufzählungen.

% Literaturverwaltung
\usepackage[style=chicago-notes, backend=biber, sorting=nyt, dashed=false, giveninits=true]{biblatex} % Für Literaturverzeichnisse nach Chicago-Stil mit Fußnoten.

% Zitate
\usepackage{csquotes} % Korrekte Darstellung von Zitaten, z. B. Anführungszeichen.

% Positionierung
\usepackage{float} % Präzise Platzierung von Tabellen und Abbildungen.

% Zeilenabstand
\usepackage[onehalfspacing]{setspace} % Erzeugt 1,5-fachen Zeilenabstand.

% Glossare
\usepackage[toc=false]{glossaries-extra} % Verwaltung von Glossaren ohne Inhaltsverzeichnis-Eintrag.

% Tabellenfarben
\usepackage{colortbl} % Ermöglicht das Einfärben von Tabellenzellen.

% Tabellen- und Spaltenformatierung
\usepackage{array} % Zusätzliche Optionen für Tabellen und Spalten.

% Diagramme und Visualisierungen
\usepackage{tikz} % Erstellung von Vektorgraphiken und Diagrammen.
\usepackage{pgfplots} % Erstellung von Plots und Grafiken.

\usetikzlibrary{shapes,arrows,positioning,fit,backgrounds,calc,matrix,decorations.pathreplacing,decorations.markings,shapes.geometric,mindmap}

% PDF-Integration
\usepackage{pdfpages} % Einfügen von ganzen PDF-Dokumenten.

% Querformat
\usepackage{pdflscape} % Ermöglicht die Nutzung von Seiten im Querformat.

% Nummerierungsanpassung
\usepackage{chngcntr} % Ermöglicht Änderungen der Nummerierung von Kapiteln und Abschnitten.

% Inhaltsverzeichnis
\usepackage{tocloft} % Anpassung des Layouts des Inhaltsverzeichnisses.

% Titelgestaltung
\usepackage{titlesec} % Anpassung von Kapitel- und Abschnittstiteln.

% Ausrichtung
\usepackage{ragged2e} % Ermöglicht den Blocksatz oder Flattersatz.

\pgfplotsset{compat=1.18}

% Zähler für Abbildungen und Tabellen an Abschnitte binden
\counterwithin{figure}{section}
\counterwithin{table}{section}
\counterwithin{equation}{section}

% Pfade für Bilder und andere Ressourcen
\graphicspath{{bilder/}}

% Arial als Hauptschrift (11pt)
\setmainfont{Arial}

% Bibliographie
\addbibresource{quellen.bib}

% Farben
\definecolor{link}{HTML}{0000EE}

% Absatzformatierung
\setlength{\parindent}{0pt}           % Keine Einrückung am Absatzbeginn
\setlength{\parskip}{6pt}             % 6pt Abstand nach jedem Absatz

% Anpassung \listoffigures
\renewcommand{\cftloftitlefont}{\Large\bfseries}
\renewcommand{\cftbeforeloftitleskip}{10pt}
\renewcommand{\cftafterloftitleskip}{10pt}

% Anpassung \listoftables
\renewcommand{\cftlottitlefont}{\Large\bfseries}
\renewcommand{\cftbeforelottitleskip}{10pt}
\renewcommand{\cftafterlottitleskip}{10pt}

% Anpassung \tableofcontents
\renewcommand{\cfttoctitlefont}{\Large\bfseries}
\renewcommand{\cftbeforetoctitleskip}{10pt}
\renewcommand{\cftaftertoctitleskip}{10pt}

% Abstand nach Nummern im Abbildungs- und Tabellenverzeichnis
\setlength{\cftfignumwidth}{3em}
\setlength{\cfttabnumwidth}{3em}

% Überschriftenformatierung
\titleformat{\chapter}
  {\fontsize{11}{16.5}\bfseries\raggedright}
  {\thechapter}{1em}{}
\titlespacing*{\chapter}{0pt}{0pt}{6pt}

\titleformat{\section}
  {\fontsize{11}{16.5}\bfseries\raggedright}
  {\thesection}{1em}{}
\titlespacing*{\section}{0pt}{0pt}{6pt}

\titleformat{\subsection}
  {\fontsize{11}{16.5}\bfseries\raggedright}
  {\thesubsection}{1em}{}
\titlespacing*{\subsection}{0pt}{0pt}{6pt}

\titleformat{\subsubsection}
  {\fontsize{11}{16.5}\bfseries\raggedright}
  {\thesubsubsection}{1em}{}
\titlespacing*{\subsubsection}{0pt}{0pt}{6pt}

% Blocksatz für Fließtext
\justifying

% Bibliographie-Einstellungen für Chicago-Stil
\DeclareNameAlias{sortname}{family-given}
\DeclareDelimFormat{nameyeardelim}{\addcomma\space}
\DeclareDelimFormat{multinamedelim}{\space\addcomma\space}
\DeclareDelimFormat{finalnamedelim}{\space\&\space}
\DefineBibliographyStrings{german}{andothers = {{et\,al\adddot}},}
\renewcommand{\mkbibnamefamily}[1]{\textsc{#1}}

% Benutzerdefinierte Zitatbefehle
\newcommand{\customcite}[2][]{%
  \ifblank{#1}{%
    \footcite{#2}%
  }{%
    \footcite[#1]{#2}%
  }%
}

% Fix lange URLs
\newcommand*\oldurlbreaks{}
\let\oldurlbreaks=\UrlBreaks
\renewcommand{\UrlBreaks}{\oldurlbreaks
  \do\a\do\b\do\c\do\d\do\e\do\f\do\g\do\h\do\i\do\j\do\k\do\l\do\m\do\n\do\o\do\p\do\q\do\r\do\s\do\t\do\u\do\v
  %\do\w%
  \do\x\do\y\do\z
  \do?\do&\do_
  \do\0\do\1\do\2\do\3\do\4\do\5\do\6\do\7\do\8\do\9
  \do:\do/
  }
\urlstyle{same}

% Befehle für Links und Verweise
\newcommand{\sct}[2]{\textcolor{link}{\hyperref[#1]{#2}}}
\newcommand{\gl}[1]{\gls{#1}}

% Inhaltsverzeichnis-Einrückung
\setlength{\cftsubsecindent}{1.5em}
\setlength{\cftsubsubsecindent}{3.0em}

% Glossar/Abkürzungsverzeichnis
\makenoidxglossaries
\newglossaryentry{ba}{
  name={B.A.},
  description={Bachelor of Arts}
}
\newglossaryentry{beng}{
  name={B.Eng.},
  description={Bachelor of Engineering}
}
\newglossaryentry{bsc}{
  name={B.Sc.},
  description={Bachelor of Science}
}
\newglossaryentry{cmos}{
  name={CMOS},
  description={The Chicago Manual of Style}
}
\newglossaryentry{llb}{
  name={LL.B.},
  description={Bachelor of Laws}
}
\newglossaryentry{llm}{
  name={LL.M.},
  description={Master of Laws}
}
\newglossaryentry{ma}{
  name={M.A.},
  description={Master of Arts}
}
\newglossaryentry{mba}{
  name={MBA},
  description={Master of Business Administration}
}

\glsaddall

% Customize page numbering on right hand side
\fancypagestyle{plain}{%
  \fancyhf{} %clear all header and footer fields
  \rhead{\thepage}
}%
\fancyhf{} %clear all header and footer fields

% Seitenstil für alle Seiten
\pagestyle{fancy}
\rhead{\fontsize{9}{13.5}\selectfont\thepage} % Seitenzahl oben rechts, 9pt
\renewcommand{\headrulewidth}{0pt} % Keine Linie unter der Kopfzeile

% Anpassung der Verzeichnis-Überschriften
\renewcommand{\contentsname}{Inhaltsverzeichnis}
\renewcommand{\listfigurename}{Abbildungsverzeichnis}
\renewcommand{\listtablename}{Tabellenverzeichnis}
\renewcommand{\bibname}{Quellen- und Literaturverzeichnis}

% Setze PDF-Metadaten
\hypersetup{
  pdftitle={Titel},
  pdfauthor={Mustermensch},
  pdfsubject={Subjekt},
}

\begin{document}

% Schmutztitel (Vorderdeckel)
\begin{titlepage}
\thispagestyle{empty}
\input{zusatz/vorderseite}
\end{titlepage}

% Leeres weißes Blatt
\newpage
\thispagestyle{empty}
\quad 
\newpage

% Haupttitel
\begin{titlepage}
\thispagestyle{empty}
\begin{titlepage}
 
    \includegraphics[width=0.5\textwidth]{rh-logo}\\[1.95cm]
    
    \vspace{2cm}
    \begin{minipage}{\textwidth}
    \hspace{3cm}\begin{minipage}{\dimexpr\textwidth-3cm\relax}
    {\Large\textbf{Fachbereich Ingenieurwesen}}\\
    
    {\Large Studiengang Informatik (B.Sc.)}\\
    \\
    \\
    
    
    {\Large\textbf{Subjekt}}
    
    \parbox{\textwidth}{\Large\raggedright Titel}\\
    \\
    
    {\normalsize\fontsize{11pt}{16.5pt}\selectfont vorgelegt von}
    
    {\Large Mustermensch}
    
    {\normalsize\fontsize{11pt}{16.5pt}\selectfont Matrikelnr. 111111111}\\


    {\normalsize\fontsize{11pt}{16.5pt}\selectfont Erstprüfer: Prof. Dr. Mustermensch}

    {\normalsize\fontsize{11pt}{16.5pt}\selectfont Zweitprüfer: Prof. Dr. Mustermensch}\\
    \\
    
    {\normalsize Sommersemester 2025}
    \end{minipage}
    \end{minipage}
    
    \vfill
    
\end{titlepage}

\end{titlepage}

% Deaktiviere Einrückungen
\setlength\parindent{0pt}

\begin{onehalfspace}
    % Römische Seitennummerierung für Verzeichnisse
    \pagenumbering{roman}
    
    % Zusammenfassung (deutsch) und Abstract (englisch)
    \chapter*{Zusammenfassung}
\addcontentsline{toc}{section}{Zusammenfassung}
\label{sec:zusammenfassung}

Der vorliegende Leitfaden zum Anfertigen wissenschaftlicher Arbeiten bietet Studierenden der Rheinischen Hochschule Köln eine umfassende Orientierung für die Erstellung von Haus- und Abschlussarbeiten. Er folgt dem Bewertungsschema der Hochschule und erläutert detailliert die sechs Leistungsabschnitte: Thema und Forschungsfrage, Strukturierung, konzeptionell-theoretische Grundlagen, Durchführung, Quellen und Zitierweise sowie Form und Stil. Der Leitfaden gibt konkrete Hinweise zur formalen Gestaltung, zur inhaltlichen Strukturierung und zur wissenschaftlichen Arbeitsweise. Ergänzende Anhänge bieten spezifische Informationen für bestimmte Studiengänge sowie praktische Hilfestellungen zu Themen wie gendergerechte Sprache. Das Dokument dient als verbindliche Grundlage für die Erstellung wissenschaftlicher Arbeiten an der Rheinischen Hochschule und ergänzt die Prüfungsordnungen.

\vspace{0.5cm}

\begin{center}
\textbf{Abstract}
\end{center}

This guide for preparing academic papers provides students at Rheinische Hochschule Köln with comprehensive orientation for creating term papers and theses. It follows the university's assessment scheme and explains in detail the six performance sections: topic and research question, structuring, conceptual-theoretical foundations, implementation, sources and citation style, as well as form and style. The guide provides specific instructions on formal design, content structuring, and scientific methodology. Supplementary appendices offer specific information for certain degree programs as well as practical assistance on topics such as gender-inclusive language. The document serves as a binding basis for the preparation of academic papers at Rheinische Hochschule and complements the examination regulations.
    \newpage
    
    % Optionale Vorbemerkung
    % Auskommentieren, wenn nicht benötigt
    \chapter*{Vorbemerkung zur Schreibweise in diesem Leitfaden}
\addcontentsline{toc}{chapter}{Vorbemerkung zur Schreibweise in diesem Leitfaden}
\label{chap:vorbemerkung}

Die Debatte um das Gendern in der deutschen Sprache wird momentan engagiert geführt. Dabei überlagern sich politische, moralische und sprachwissenschaftliche Ebenen.\footnote{Der gesamte Text wurde mit folgendem Prompt bearbeitet: \promptentry{\glqq Versetze dich in die Rolle eines Lektors. Kontrolliere den folgenden Text in Bezug auf die aktuell verbindlichen Regeln der deutschen Orthographie, Grammatik und Interpunktion. Achte auch auf typographisch korrekte Anführungszeichen sowie darauf, dass Gedankenstriche anstelle von Bindestrichen (Divis) verwendet werden. Kontrolliere außerdem die Ausdrucksweise: Sie soll sachlich, nüchtern, klar, präzise und eindeutig sein. Erhalte den fachsprachlichen Duktus. Markiere alle Textstellen, die dir als Lektor unter diesen Gesichtspunkten auffallen.\grqq{}, ChatGPT-3.5, Open AI, 8.11.2024, stilistisch und inhaltlich korrigiert.}}

Unter linguistischer Perspektive geht es beim Gendern im Kern um die Frage, wie Funktionsbezeichnungen vor allem im Plural zum Ausdruck gebracht werden sollen. Die Befürworter des Genderns gehen davon aus, dass die generische Pluralform immer nur spezifisch als Ausdruck für Männer verstanden wird. Die Kritiker dieser Position gehen hingegen davon aus, dass mit der generischen Pluralform eine Aussage getroffen wird, die unabhängig ist von den konkreten Personen, welche eine Funktion ausüben.

In den beiden vorigen Sätzen wird der generische Plural \glqq Befürworter\grqq{} und \glqq Kritiker\grqq{} verwendet. Die ausschlaggebende Grundlage für diese Entscheidung, die für den gesamten Leitfaden gilt, ist der gegenwärtige Sachstand, dass das Gendern mit einem Binnen-I (\glqq VerfasserInnen\grqq{}) oder mit Sonderzeichen (z.B. \glqq Befürworter*innen\grqq{} oder \glqq Kritiker:innen\grqq{}) kein Bestandteil der deutschen Rechtschreibregelung ist.

Gleichzeitig lautet die Empfehlung dieses Leitfadens, dass Sie als diejenigen, die eine wissenschaftliche Arbeit verfassen, selbst darüber entscheiden sollen, ob Sie Ihren Text gendern oder nicht. Auf die Benotung soll Ihre Entscheidung keinen Einfluss haben.

Wofür Sie sich auch entscheiden: Es in jedem Fall erforderlich, dass Sie einheitlich schreiben, denn Uneinheitlichkeit führt zu Punktabzug.

Falls Sie möchten, können Sie Ihrer Arbeit einen spezifischen Hinweis voranstellen, zum Beispiel in einer Vorbemerkung oder in einer Fußnote, weil Sie damit möglichen Fehldeutungen vorbeugen können. Zwingend notwendig ist ein solcher Hinweis nicht.\footnote{In dieser Arbeit wird das generische Maskulinum verwendet, wenn im Mittelpunkt der Aussagen keine spezifischen Personen stehen, sondern die Funktion, die Personen ausüben. Sämtliche biologischen Geschlechter und subjektiv empfundenen Zuordnungen sind damit gleichermaßen gemeint.}

Falls Sie in Ihrem Text gendern möchten, finden Sie dazu Hinweise in Anhang A.
    \newpage
    
    % Inhaltsverzeichnis
    \phantomsection
    \addcontentsline{toc}{section}{Inhaltsverzeichnis}
    \tableofcontents
    \newpage
    
    % Abkürzungsverzeichnis
    \phantomsection
    \addcontentsline{toc}{section}{Abkürzungsverzeichnis}
    \label{sec:Abkuerzungsverzeichnis}
    \printnoidxglossary[title={Abkürzungsverzeichnis}, type=main, nonumberlist]
    \newpage
    
    % Abbildungsverzeichnis
    \phantomsection
    \addcontentsline{toc}{section}{Abbildungsverzeichnis}
    \listoffigures
    \newpage
    
    % Tabellenverzeichnis
    \phantomsection
    \addcontentsline{toc}{section}{Tabellenverzeichnis}
    \listoftables
    \newpage
    
    % Promptverzeichnis (für KI-Prompts)
    \chapter*{Promptverzeichnis}
\addcontentsline{toc}{chapter}{Promptverzeichnis}
\label{sec:promptverzeichnis}

\begin{enumerate}
\item \glqq Versetze dich in die Rolle eines Lektors. Kontrolliere den folgenden Text in Bezug auf die aktuell verbindlichen Regeln der deutschen Orthographie, Grammatik und Interpunktion. Achte auch auf typographisch korrekte Anführungszeichen sowie darauf, dass Gedankenstriche anstelle von Bindestrichen (Divis) verwendet werden. Kontrolliere außerdem die Ausdrucksweise: Sie soll sachlich, nüchtern, klar, präzise und eindeutig sein. Erhalte den fachsprachlichen Duktus. Markiere alle Textstellen, die dir unter diesen Gesichtspunkten als Lektor auffallen.\grqq{}, ChatGPT-3.5, Open AI, 15.10.2024, stilistisch und inhaltlich korrigiert.
\end{enumerate}
    \newpage

    % Hauptteil mit arabischer Seitennummerierung
    \pagenumbering{arabic}

    % Kapitel einbinden
    \foreach \i in {10,11} {%
      \edef\FileName{abschnitte/\i-kapitel}%

      \IfFileExists{\FileName}{%
        \input{\FileName}
      }
      {%
        %file does not exist
      }
	  }

    % Literaturverzeichnis
    \newpage
    \phantomsection
    \addcontentsline{toc}{chapter}{Quellen- und Literaturverzeichnis}
    \label{chap:Quellen-und-Literaturverzeichnis}
    \printbibliography[title={Quellen- und Literaturverzeichnis}]
    
    % Anhänge
    \newpage
    \appendix
    \chapter*{Anhang}
    \addcontentsline{toc}{chapter}{Anhang}
    \label{chap:Anhang}
    % Include all appendix files
\chapter*{Anhang A: Praktische Hinweise zum Gendern}
\addcontentsline{toc}{section}{Anhang A: Praktische Hinweise zum Gendern}
\label{app:gendern}

Eine häufig verwendete Möglichkeit ist das substantivierte Partizip Präsens. Es zeigt im Plural kein Genus:
\begin{itemize}
\item die Studierenden,
\item die Lehrenden.
\end{itemize}

Bitte beachten Sie, dass das substantivierte Partizip Präsens im Singular ein Genus zeigt:
\begin{itemize}
\item der Studierende ist eindeutig maskulin,
\item die Lehrende ist eindeutig feminin.
\end{itemize}

Abstrakte Rollenbezeichnungen und Begriffe für Kollektive, Institutionen und Positionen sind eine weitere Möglichkeit für sexusindifferente Formulierungen:
\begin{itemize}
\item die Vertretung,
\item die Studiengangsleitung,
\item das Kollegium,
\item das Publikum,
\item die Lehrkräfte,
\item das Team,
\item das Präsidium.
\end{itemize}

Die Pronomina alle und niemand zeigen ebenfalls kein Genus:
\begin{itemize}
\item Alle können am Kurs teilnehmen.
\item Niemand muss sich wegen der Prüfung Sorgen machen.
\end{itemize}

Gender-Zeichen, meist ein Doppelpunkt oder Sternchen (Asterisk), stehen zwischen dem Wortstamm der männlichen Form und der weiblichen Endung:
\begin{itemize}
\item Bewerber:in
\item Teilnehmer:innen
\item Student:in
\item Expert:innen
\end{itemize}

Die Beidnennung ist eine weitere Möglichkeit, also die Nennung von maskulinen und femininen Bezeichnungen:
\begin{itemize}
\item Studentinnen und Studenten,
\item Dozentinnen und Dozenten,
\item Prüferin bzw. Prüfer,
\item Expertin bzw. Experte.
\end{itemize}
\input{anhaenge/anhang-b}
\chapter*{Anhang C: Ergänzende Hinweise für Abschlussarbeiten in den Studiengängen Wirtschaftspsychologie und Mediendesign}
\addcontentsline{toc}{section}{Anhang C: Ergänzende Hinweise für Abschlussarbeiten in den Studiengängen Wirtschaftspsychologie und Mediendesign}
\label{app:wirtschaftspsychologie-mediendesign}

In diesen beiden Studiengängen können die Abschlussarbeit unterschiedlich ausgerichtet werden. Möglich sind:
\begin{enumerate}
\item theoretisch-konzeptionelle Abschlussarbeiten,
\item empirische Arbeiten (in der Wirtschaftspsychologie verpflichtend): qualitative oder quantitative Untersuchungen,
\item konzeptionell-gestalterische Arbeiten (nur im Studiengang Mediendesign).
\end{enumerate}

Eine eindeutige Zuordnung zu einem dieser Schwerpunkte ist nicht immer möglich.

In jedem Fall ist eine umfassende und systematische Darstellung des Stands der Forschung anhand der aktuellen wissenschaftlichen Literatur (Diskurs) erforderlich. Auf diesen Status Quo bezieht sich Ihre weitere Bearbeitung, bzw. Sie wenden ihn dafür an.

Alle Arbeiten können in Kooperation mit Unternehmen geschrieben werden. Bitte stimmen Sie sich dazu mit Ihren Prüfern ab.

\section*{1. Theoretisch-konzeptionelle Arbeiten}

Wenn Sie eine theoretisch-konzeptionelle Arbeit vorlegen, analysieren Sie darin ein Thema, ohne dass Sie hierzu eigene Berechnungen durchführen.

Ihr Ausgangspunkt ist eine möglichst vollständige Erfassung der relevanten Literatur und die systematische Aufbereitung der gewonnenen Informationen. Wie Sie dieses Material strukturieren und aufbereiten (Tiefe, Differenzierung, Präzision, Prägnanz, Gewichtung), bestimmt wesentlich die Qualität Ihrer Arbeit.

Typischerweise wird von Ihnen erwartet, dass Sie die wesentliche Literatur durchdringen und sich mit ihr kritisch auseinandersetzen. Dazu zählen insbesondere Definitionen (die Sie evtl. selbst entwickeln), Klassifikationsschemata oder Typologien. Zudem wird erwartet, dass sie Begründungszusammenhänge strukturierte darstellen.

Ein zentrales Ziel der theoretisch-konzeptionellen Arbeit ist die Verbesserung, Vereinfachung und vor allem Systematisierung der Darstellung bereits bekannter, aber bisher nicht in einer einzigen Arbeit zusammengefasster Fakten oder Argumente.

Den Titel Ihrer theoretisch-konzeptionellen Arbeit können Sie beispielsweise nach diesem Schema formulieren:
\begin{itemize}
\item Entwicklung einer strategisch-konzeptionellen Empfehlung
\item Formulierung von Handlungsempfehlungen für bestimmte Fälle
\item Eingrenzung von potenziellen Entwicklungsszenarien
\item Konzeption von bestimmten Maßnahmenempfehlungen für besondere Zwecke (beispielsweise aus dem konkreten Arbeitsumfeld).
\end{itemize}

Ihre Strukturierungsleistung soll zu einem Ziel führen, das bereits im Titel anklingt, z.B.: \glqq Strategische Markt- und Wettbewerbsanalyse der Videospielbranche - unter besonderer Berücksichtigung der Neuprodukteinführung eines Konsolenherstellers\grqq{}.

\section*{2. Empirische Arbeiten}

Bei Ihrer empirischen Arbeit kann es sich entweder um eine qualitative oder quantitative Untersuchung handeln.

Qualitative Untersuchungen werden durchgeführt, wenn ein wesentliches Verständnis für eine bestimmte Forschungsfrage erlangt werden soll. Ziel der qualitativen Forschung ist das Erkennen, Beschreiben und Verstehen von Zusammenhängen. Im Vordergrund stehen die vollständige Erfassung und Interpretation aller problemrelevanten Aspekte hinsichtlich des Themas der Abschlussarbeit. Hierbei bedient man sich offener, nicht standardisierter Erhebungsverfahren (z.B. Interviews, Expertengespräche, Gruppendiskussionen, qualitative Beobachtung, qualitative Experimente), deskriptiver Aufbereitungsverfahren (z.B. Gesprächsprotokoll, Transkription) und interpretativer Auswertungsverfahren (z.B. qualitative Inhaltsanalyse, Cognitive Mapping, objektive Hermeneutik). In der Regel kommen im Rahmen der qualitativen Untersuchung Methoden zum Einsatz, die sich auf kleine Fallzahlen beschränken, keine statistischen Analysen (z.B. Signifikanztests) implizieren, relative weiche Daten produzieren und ihre Erkenntnisse auf einem verhältnismäßig niedrigen Abstraktionsniveau mittels subjektiver Interpretation gewinnen. Die gewonnenen Ergebnisse sind zwar nicht repräsentativ, sie dienen aber zur ausführlichen Stoffsammlung, um ihrerseits wieder Hypothesen quantifizieren zu können.

Quantitative Untersuchungen haben die Messung bestimmter Sachverhalte bzw. die Entdeckung von Gesetzmäßigkeiten zum Gegenstand. Es handelt sich hierbei um einen Ansatz, der theoriegeleitet ist und der sich standardisierter Erhebungsmethoden (i.d.R. schriftliche Fragebögen) bedient. Ziel dieser Standardisierung ist es, die Antworten einer Vielzahl von Befragten unmittelbar vergleichen zu können. Der Vorteil quantitativer Untersuchungen liegt darin, dass sich die Messergebnisse mit statistischen Methoden (z.B. Korrelations- und Regressionsanalysen, Varianzanalysen, Faktoren- und Clusteranalysen, Metaanalysen, Diskriminanzanalysen, MDS) unter Nutzung statistischer Kennzahlen (Mittelwert, Median, Standardabweichung, Varianz etc.) verdichten und weiterverarbeiten lassen. Die Auswertung und Analyse der Daten, die im Rahmen der quantitativen Forschung generiert werden, erfolgt mit Hilfe von statistischen Programmen (z.B. SPSS, AMOS, PLS, R). Ziel ist es, Zusammenhänge zu erkennen und daraus allgemein gültige Aussagen abzuleiten. Rückschlüsse auf die tatsächlichen Verhältnisse in der Grundgesamtheit sind möglich.

In der Wirtschaftspsychologie (B.Sc.) werden ausschließlich empirische Abschlussarbeiten durchgeführt, wobei quantitativen Ansätzen der Vorzug zu geben ist.

\section*{3. Konzeptionell-gestalterische Arbeiten}

Diese Arbeit können Sie nur im Studiengang Mediendesign vorlegen. Hierbei handelt es sich um eine innovative Lösung für eine komplexe gestalterische Aufgabe.

Ihr Ausgangspunkt ist auch hier die Durchdringung relevanter wissenschaftlicher Grundlagen (insbesondere Theorien und Beispiele aus der Designgeschichte).

Daran schließen sich evtl. strategische sowie Ihre konzeptionellen und methodischen Darstellungen an, in denen Sie den Kontext der Aufgabe und Ihren Designprozess nachvollziehbar machen.

Sowohl die wissenschaftlichen als auch die methodischen und konzeptionellen Aussagen führen zu Ihrer praktischen Umsetzung im Sinn einer Lösungsentwicklung mit gestalterischen und technischen Mitteln.
\chapter*{Anhang D: Gestalterische Hinweise für die Studiengänge Mediendesign (B.A.) und User Experience Design (M.A.)}
\addcontentsline{toc}{section}{Anhang D: Gestalterische Hinweise für die Studiengänge Mediendesign (B.A.) und User Experience Design (M.A.)}
\label{app:gestalterische-hinweise}

\section*{Schmutztitel und Haupttitel}

Auf dem Schmutztitel und dem Haupttitel stehen die gleichen Inhalte wie bei allen anderen Studiengängen (vgl. Abschnitt 6.6 in diesem Leitfaden). Ansonsten sind Sie in Ihrer Gestaltung der Titel frei. Es ist auch ausdrücklich erwünscht, dass Sie von dieser Freiheit Gebrauch machen. Falls Sie sich dafür entscheiden, Ihre Arbeit nicht zu gestalten, verwenden Sie einen braunen (Mediendesign) bzw. weißen (User Experience Design) Einband. Bitte beachten Sie, dass Sie keinen schlichten andersfarbigen Einband verwenden, weil diese bereits anderen Studiengängen vorbehalten sind.

\section*{Schriften}

Beachten Sie bei der Wahl Ihrer Schrift die Angemessenheit. Neben der akademischen Ausstrahlung ist es wichtig, dass Sie sich später als Bewerber mit Ihrer Arbeit gegen Konkurrenten durchsetzten wollen. Alle Schriften für sogenannte Mengentexte sind aufgrund ihrer Lesefreundlichkeit geeignet, beispielsweise die Thesis, Info Text oder Corporate oder Schriften nach dem dynamischen Formprinzip, wie z.B. Garamond, Minion, Weidemann, Lexicon, Optima, Officina, Frutiger, Today oder Meta.

Beachten Sie bitte auch, dass die Prüfer Ihre Arbeit innerhalb kurzer Zeit lesen werden. Die Lektüre darf auch Freude bereiten, deshalb ist die Lesbarkeit bei der Auswahl ein wichtiges Entscheidungskriterium. Eine schmal laufende Schrift (condensed) ist also nicht geeignet. Serifenschriften bieten in der Regel Vorteile, sind aber kein Muss.

\section*{Raster und Layout}

Die Bachelor-Thesis soll angemessene Seitenränder haben und nur eine Textspalte, sowohl einen toten als auch einen lebenden Kolumnentitel integrieren sowie zwingend über eine Paginierung verfügen. Abbildungen sollt angemessen groß sein. Komplexe Schaubilder oder die Herleitung Ihrer Entwürfe dürfen, sofern notwendig, seitenfüllend sein. Für Infografiken entwickeln Sie idealerweise einen durchgängigen einheitlichen Stil.

\section*{Feintypografie}

Beachten Sie bitte alle feintypografischen Regeln. Halten Sie bitte sauber definierte Abstände vor und nach Überschriften oder Absätzen usw. ein. Ein Ausgleich des Flatter- oder Blocksatzes nach Fertigstellen des Textes ist Pflicht. Tipp: Beginnen Sie damit von hinten.

    % Eigenständigkeitserklärung
    \newpage
    \chapter*{Eigenständigkeitserklärung}
\addcontentsline{toc}{section}{Eigenständigkeitserklärung}
Hiermit bestätige ich, dass ich die vorliegende Arbeit selbstständig verfasst und keine 
anderen Publikationen, Vorlagen und Hilfsmittel (z.B. künstliche Intelligenz) als die 
angegebenen benutzt habe. Alle Teile meiner Arbeit, die wortwörtlich oder dem Sinn nach 
anderen Werken entnommen sind, wurden unter Angabe der Quelle kenntlich gemacht. 
Gleiches gilt für von mir verwendete Internetquellen. Ich versichere, dass ich diese Arbeit 
oder nicht zitierte Teile daraus vorher nicht in einem anderen Prüfungsverfahren eingereicht 
habe. Mir ist bekannt, dass meine Arbeit zum Zwecke eines Plagiatsabgleichs mittels einer 
Plagiatserkennungssoftware auf eine ungekennzeichnete Übernahme von fremdem geistigen 
Eigentum sowie auf die Nutzung von künstlicher Intelligenz zur Texterstellung überprüft 
werden kann. Ich versichere, dass die elektronische Form meiner Arbeit mit der gedruckten 
Version identisch ist.


% \section*{Declaration of Originality}
% I hereby confirm that I have independently written this work and have not used any 
% publications, templates, or aids (e.g. artificial intelligence) other than those I have 
% indicated. All parts of my work which have been taken literally or correspondingly from other
% publications have been duly acknowledged. This also applies to Internet sources. I confirm
% that I have not previously submitted this work or any unquoted parts thereof in any other 
% examination procedure. I am aware that my work may be checked for plagiarism by means 
% of plagiarism recognition software, as well as for the use of artificial intelligence for text
% creation, in order to verify the integrity of its written content. I also confirm that the 
% electronic form is identical to the printed version. \\

\vspace{2cm}

\begin{center}
\begin{tabular}{p{0.45\textwidth}p{0.45\textwidth}}
    [Ort], \today & \\
    \hline
    % Place, Date & Signature \\
    Ort, Datum & Unterschrift \\
\end{tabular}
\end{center}
    
    % Lebenslauf (nur für Abschlussarbeiten)
    \newpage
    \chapter*{Lebenslauf}
\addcontentsline{toc}{section}{Lebenslauf}
\label{sec:lebenslauf}

\begin{tabular}{p{0.25\textwidth}p{0.7\textwidth}}
\textbf{Persönliche Daten} & \\
\\
Name: & [Vorname Nachname] \\
Geburtsdatum: & [TT.MM.JJJJ] \\
Geburtsort: & [Geburtsort] \\
Staatsangehörigkeit: & [Staatsangehörigkeit] \\
\\
\textbf{Ausbildung} & \\
\\
MM.JJJJ - heute & Studium Studiengang an der Rheinischen Hochschule Köln \\
MM.JJJJ - MM.JJJJ & Vorherige Ausbildung/Studium \\
MM.JJJJ - MM.JJJJ & Schulbildung \\
\\
\textbf{Berufliche Erfahrung} & \\
\\
MM.JJJJ - heute & Aktuelle Tätigkeit \\
MM.JJJJ - MM.JJJJ & Vorherige Tätigkeit \\
MM.JJJJ - MM.JJJJ & Praktikum/Werkstudententätigkeit \\
\\
\textbf{Kenntnisse und Fähigkeiten} & \\
\\
Sprachen: & Sprache 1 (Niveau), Sprache 2 (Niveau), ... \\
EDV-Kenntnisse: & Software 1, Software 2, ... \\
Sonstige Qualifikationen: & Qualifikation 1, Qualifikation 2, ... \\
\\
\end{tabular}

\vspace{1cm}

\begin{center}
Ort, Datum
\end{center}

\end{onehalfspace}
\end{document}
