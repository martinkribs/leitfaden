\chapter*{Vorbemerkung zur Schreibweise in diesem Leitfaden}
\addcontentsline{toc}{chapter}{Vorbemerkung zur Schreibweise in diesem Leitfaden}
\label{chap:vorbemerkung}

Die Debatte um das Gendern in der deutschen Sprache wird momentan engagiert geführt. Dabei überlagern sich politische, moralische und sprachwissenschaftliche Ebenen.\footnote{Der gesamte Text wurde mit folgendem Prompt bearbeitet: \promptentry{\glqq Versetze dich in die Rolle eines Lektors. Kontrolliere den folgenden Text in Bezug auf die aktuell verbindlichen Regeln der deutschen Orthographie, Grammatik und Interpunktion. Achte auch auf typographisch korrekte Anführungszeichen sowie darauf, dass Gedankenstriche anstelle von Bindestrichen (Divis) verwendet werden. Kontrolliere außerdem die Ausdrucksweise: Sie soll sachlich, nüchtern, klar, präzise und eindeutig sein. Erhalte den fachsprachlichen Duktus. Markiere alle Textstellen, die dir als Lektor unter diesen Gesichtspunkten auffallen.\grqq{}, ChatGPT-3.5, Open AI, 8.11.2024, stilistisch und inhaltlich korrigiert.}}

Unter linguistischer Perspektive geht es beim Gendern im Kern um die Frage, wie Funktionsbezeichnungen vor allem im Plural zum Ausdruck gebracht werden sollen. Die Befürworter des Genderns gehen davon aus, dass die generische Pluralform immer nur spezifisch als Ausdruck für Männer verstanden wird. Die Kritiker dieser Position gehen hingegen davon aus, dass mit der generischen Pluralform eine Aussage getroffen wird, die unabhängig ist von den konkreten Personen, welche eine Funktion ausüben.

In den beiden vorigen Sätzen wird der generische Plural \glqq Befürworter\grqq{} und \glqq Kritiker\grqq{} verwendet. Die ausschlaggebende Grundlage für diese Entscheidung, die für den gesamten Leitfaden gilt, ist der gegenwärtige Sachstand, dass das Gendern mit einem Binnen-I (\glqq VerfasserInnen\grqq{}) oder mit Sonderzeichen (z.B. \glqq Befürworter*innen\grqq{} oder \glqq Kritiker:innen\grqq{}) kein Bestandteil der deutschen Rechtschreibregelung ist.

Gleichzeitig lautet die Empfehlung dieses Leitfadens, dass Sie als diejenigen, die eine wissenschaftliche Arbeit verfassen, selbst darüber entscheiden sollen, ob Sie Ihren Text gendern oder nicht. Auf die Benotung soll Ihre Entscheidung keinen Einfluss haben.

Wofür Sie sich auch entscheiden: Es in jedem Fall erforderlich, dass Sie einheitlich schreiben, denn Uneinheitlichkeit führt zu Punktabzug.

Falls Sie möchten, können Sie Ihrer Arbeit einen spezifischen Hinweis voranstellen, zum Beispiel in einer Vorbemerkung oder in einer Fußnote, weil Sie damit möglichen Fehldeutungen vorbeugen können. Zwingend notwendig ist ein solcher Hinweis nicht.\footnote{In dieser Arbeit wird das generische Maskulinum verwendet, wenn im Mittelpunkt der Aussagen keine spezifischen Personen stehen, sondern die Funktion, die Personen ausüben. Sämtliche biologischen Geschlechter und subjektiv empfundenen Zuordnungen sind damit gleichermaßen gemeint.}

Falls Sie in Ihrem Text gendern möchten, finden Sie dazu Hinweise in Anhang A.