\chapter*{Vorbemerkung}
\addcontentsline{toc}{section}{Vorbemerkung}

% Hier kann eine Vorbemerkung eingefügt werden, z.B. zum Gendern
In dieser Arbeit wird das generische Maskulinum verwendet, wenn im Mittelpunkt der Aussagen keine spezifischen Personen stehen, sondern die Funktion, die Personen ausüben. Sämtliche biologischen Geschlechter und subjektiv empfundenen Zuordnungen sind damit gleichermaßen gemeint.

% Alternativ kann hier auch ein anderer Hinweis zum Gendern stehen, z.B.:
% In dieser Arbeit werden geschlechtsneutrale Formulierungen verwendet, um alle Geschlechter gleichermaßen anzusprechen.