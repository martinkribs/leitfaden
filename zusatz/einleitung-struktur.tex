\chapter*{Einleitung}
\addcontentsline{toc}{chapter}{Einleitung}
\label{chap:einleitung}

Die scientific community – also die Gemeinschaft der Menschen, die forschen und ihre Erkenntnisse mit anderen teilen wollen – steht allen offen. Es gibt keine formale, exkludierende Zugangsbarriere.

Spätestens seit Ihrer Immatrikulation an der Rheinischen Hochschule sind Sie Teil dieser scientifc community. Herzlich willkommen!

Ihr Studium bereitet Sie nicht erst aufs Forschen vor. Es steht Ihnen frei, vom ersten Tag an Ihre Zeit an der Rheinischen Hochschule forschend zu nutzen. Das gilt nicht nur für Seminare oder Projekte. Sie können auch schon Ihre Haus- und Abschlussarbeiten als Forschung konzipieren. Es spielt dafür keine Rolle, ob Sie Ihr Forschungsinteresse auf ein äußerst kleines Feld eingrenzen. Jeder Erkenntnisgewinn ist ein Dienst an der Wissenschaft, mag er Ihnen auch gering erscheinen.

Mit der Lektüre dieses Leitfaden ersetzen Sie selbstverständlich nicht Ihre aktive Teilnahme am Modul Wissenschaftliches Arbeiten. Es handelt sich hier lediglich um einige grundlegende Hinweise von übergeordnetem, die gesamte Hochschule betreffenden Charakter. Damit sollen insbesondere solche Aspekte des wissenschaftlichen Arbeitens geklärt werden, die sich verallgemeinern lassen.

Am Ende ist die Absprache mit Ihren Prüfern verbindlich, weil die individuelle Durchführung von Forschung und Lehre durch Artikel 5 Absatz 3 des Grundgesetzes geschützt ist.

Wissenschaftliches Arbeiten ist keine Tätigkeit, der Sie sich zuwenden, wenn Sie alles andere im Studium erledigt haben. Im Kern ist das Schreiben eines wissenschaftlichen Textes die Forschung selbst. Denn Forschung zielt auf Erkenntnisgewinn, den Sie nicht für sich behalten, sondern durch Veröffentlichung mit Ihrer Fachöffentlichkeit teilen.

Dafür müssen Sie Ihre Erkenntnisse in eine schriftliche Form bringen. Und um Ihr Publikum zu überzeugen, müssen Sie Ihre Aussagen intersubjektiv nachvollziehbar formulieren und sich auf vertrauenswürdige Quellen stützen. Beide Aspekte, die schriftliche Form und Ihre Nachweisführung, werden in diesem vorliegenden Leitfaden sowie in der separaten Praktische[n] Anleitung: Zitieren in wissenschaftlichen Arbeiten der Rheinischen Hochschule behandelt.

Was für Wissenschaft grundsätzlich gilt, das trifft auch auf diesen Leitfaden zu: Wir sind alle nur Menschen, die sich irren können und denen versehentliche Fehler unterlaufen. Falls Ihnen also etwas auffällt, das Ihnen fragwürdig oder offensichtlich falsch erscheint und deshalb geändert werden sollte, freuen wir uns jederzeit über Ihren Hinweis an unser Qualitätsmanagement: qm@rh-koeln.de. Vielen Dank für Ihre Hilfe!

\chapter*{Struktur und Gültigkeit dieses Leitfadens}
\addcontentsline{toc}{chapter}{Struktur und Gültigkeit dieses Leitfadens}
\label{chap:leitfaden-struktur}

Mit jeder Hausarbeit und Abschlussarbeit führen Sie vor, dass Sie das Wissen, welches Sie sich im Studium angeeignet haben, anwenden und praxisrelevante Fragen sachgerecht in einer wissenschaftlichen Methode beantworten können. Inwieweit Ihnen dies gelungen ist, unterliegt der Bewertung durch Ihre Prüfer.

Für die Bewertung wissenschaftlicher Arbeiten verwendet die Rheinische Hochschule ein einheitliches Schema. Es enthält sechs Abschnitte mit detaillierten Hinweisen und bestimmt, wieviele Punkte Sie für welche Leistungen erhalten.

\begin{table}[h]
\label{tab:leistungsabschnitte}
\begin{tabular}{|l|l|c|}
\hline
\textbf{Nr.} & \textbf{Leistungsabschnitt} & \textbf{Punkte} \\
\hline
1. & Thema und Forschungsfrage bzw. Aufgabenstellung, Hypothesen & 10 \\
\hline
2. & Strukturierung & 10 \\
\hline
3. & Konzeptionell-theoretische Grundlagen & 20 \\
\hline
4. & Durchführung & 40 \\
\hline
5. & Quellen und Zitierweise & 10 \\
\hline
6. & Form und Stil & 10 \\
\hline
\end{tabular}
\caption{Übersicht der Leistungsabschnitte für die Bewertung von Haus- und Abschlussarbeiten an der Rheinischen Hochschule Köln.}
\end{table}

Damit Sie genau nachvollziehen können, welche Leistungen von Ihnen erwartet werden (Kategorien) und wie diese bewertet werden (Kriterien), folgt dieser Leitfaden dem Bewertungsschema.

In Anhang C finden Sie ergänzende Hinweise für die Abschlussarbeiten in den Studiengängen Mediendesign und Wirtschaftspsychologie.

Der vorliegende Leitfaden ergänzt die Prüfungsordnungen (RSPO und SPO), die im Zweifelsfall verbindlich sind.